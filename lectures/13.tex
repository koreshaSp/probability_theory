% 2023.05.23 lecture 13
\documentclass[../main.tex]{subfiles}
\begin{document}

\begin{exmpl}[управление запасами]
 Есть склад, на складе может быть максимум $ S $ единиц товара. Пусть $ \eta_i $ --- это спрос на товар в момент времени $ i $. Будем считать, что $ \eta_i $ --- независимые случайные величины. Если единиц товара в какой-то момент становится меньше $ s $, то склад заполняется до максимума. Обозначим за $ \xi_n $ количество единиц товара на складе в момент времени $ n $. Оно определяется так:
 \begin{align*}
  \xi_n = \begin{cases}
   \xi_{n-1} - \eta_n, \text{ если }  \xi_{n-1} \geqslant s, \\
   S - \eta_n, \text{ если } \xi_{n-1} < s.
  \end{cases}
 \end{align*} При этом считаем, что $ \eta_n \leqslant s $ всегда.
\end{exmpl}

\newpage
\section{Случайные блуждания.}

Возвратно = вероятность вернуться равна $ 1 $.

\begin{thm}
 Случайное блуждание по $ \Z $ возвратно тогда и только тогда, когда $ p = 1 / 2 $.
\end{thm}
\begin{proof}[\normalfont\textsc{Доказательство}]
 По критерию возвратности состояние $ 0 $ возвратно, если и только если ряд
 \begin{align}
  \sum_{n=1}^{\infty}p_{00}(n) = +\infty.
 \end{align} расходится. Но
 \begin{align*}
  &p_{00}(2n+1)= 0, \\
  &p_{00}(2n) = \binom{2n} n \cdot p^{n}(1-p)^{n} \sim \frac{4^{n}}{\sqrt{\pi n}} p^{n}(1-p)^{n}.
 \end{align*} Если $ p = 1 / 2 $, то $ p_{00} \sim \frac{1}{\sqrt{\pi n}} $, и ряд расходится. Если же $ p \neq 1 / 2 $, то $ p_{00}(2n) = \OO \left( (4p(1-p))^{n} \right) $, и ряд сходится ведь $ 4p(1-p) < 1 $.
\end{proof}

Теперь посмотрим не на блуждание по соседним точкам, а на, так называемое, \textit{произвольное} блуждание.

\begin{df}[произвольное изотропное блуждание]
 Есть $ \eta_n $ --- целозначная случайная величина все $ \eta_n $ независимы и одинаково распределены, и $ \xi_n = \eta_1 + \ldots + \eta_n $.
\end{df}

\begin{thm}
 Если $ \eta_n $ симметричные (то есть $ P(\eta_n = a) = P(\eta_n = -a) $ для всех $ a $) и имеют матожидание, то случайное блуждание возвратно.
\end{thm}
\begin{remrk*}
 Если матожидание есть, то оно обязательно равно нулю.
\end{remrk*}
\begin{proof}[\normalfont\textsc{Доказательство теоремы 1.3}]
 Надо доказать, что ряд
 \begin{align*}
  \sum_{n=1}^{\infty}p_{00}(n) = +\infty
 \end{align*} расходится.
 \begin{align*}
  p_{00}(n)&=P(\xi_n=0)=G_{\xi_n}(0) = G^{n}(0),
 \end{align*} где $ G $ --- производящая функция для $ \eta_n $ (ряд Лорана в этом случае). Проинтегрируем по контуру
 \begin{align*}
  G^{n}[0] = \frac{1}{2\pi i} \int\limits_{\left| z \right|=1}   \frac{G^{n}(z)}{z}\,dz.
 \end{align*} Тогда
 \begin{align*}
  \sum_{n=1}^{\infty}P(\xi_n=0)&=\sum_{n=1}^{\infty} \frac{1}{2\pi i} \int\limits_{\left| z \right|=1}   \frac{G^{n}(z)}{z}\,dz = \frac{1}{2\pi i} \int\limits_{\left| z \right|=1} \frac{1}{z} \sum_{n=0}^{\infty} G^{n}(z)\,dz = \\
  &= \frac{1}{2\pi i} \int\limits_{\left| z \right|=1} \frac{1}{z} \cdot \frac{1}{1 - G(z)} = \begin{bmatrix}
   z = e^{it} & dz = ie^{it}\,dt
  \end{bmatrix} = \\
 &= \frac{1}{2\pi } \int_{-\pi}^{\pi} \frac{dt}{1 - G(e^{it})} = \frac{1}{\pi} \int_{0}^{\pi} \frac{dt}{1 - G(e^{it})}.
 \end{align*} Но
 \begin{align*}
  G(1+s) = 1 + o(s).
 \end{align*} Интеграл сходится. Но очень нагло. Так не хорошо.

 Исправим.
 \begin{align*}
  \sum_{n=0}^{\infty} x^{n}P(\xi_n=0) = \sum_{n=0}^{\infty} \frac{x^{n}}{2\pi i} \int\limits_{\left| z \right|=1}   \frac{G^{n}(z)}{z}\,dz
 \end{align*} $ x \in (0,1) $. Всё сходится равномерно! И все получается! Итого
 \begin{align*}
  \sum_{n=0}^{\infty}x^{n}P(\xi_n=0) &= \frac{1}{\pi}\int_{0}^{\pi} \frac{dt}{1-xG(e^{it})} = \begin{bmatrix}
   G(e^{it}) = 1 + o(t)
  \end{bmatrix} = \\
  &= \frac{1}{\pi} \int_{0}^{\pi} \frac{dt}{1-x + o(xt)} \geqslant \frac{1}{\pi} \int_{0}^{\pi}  \frac{dt}{1-x+xt} = \frac{1}{\pi} \left.\frac{\log \left| 1-x+xt \right|}{x}\right|_0^{\pi} = \\
   &= -\frac{\log(1-x)}{\pi x} + \frac{\log(1-x+\pi x)}{\pi x}.
  \end{align*} Теперь устремим $ x \to 1+  $. Тогда левая часть стремится к ряду, а правая часть к $ +\infty $. Ряд расходится.
\end{proof}

\begin{remrk}
 Все состояния тут нулевые.
\end{remrk}
\begin{proof}[\normalfont\textsc{Доказательство}]
 Надо понять, что
 \begin{align*}
  \int\limits_{\left| z \right|=1} \frac{G^{n}(z)}{z}\,dz \to 0,\\
  = c \int_{-\pi}^{\pi} G^{n}(e^{it})\,dt
 \end{align*} Поймём, что $ G^{n}(e^{it}) \to 0 $ почти всюду. Знаем, что
 \begin{align*}
  \left| G(e^{it}) \right| = \left| \sum_{-\infty}^{+\infty} P(\eta=n)e^{int} \right| \leqslant \sum_{-\infty}^{+\infty} P(\eta = n)
 \end{align*} Равенство означает, что все аргументы одинаковы при $ P(\eta=n)>0 $.
 \begin{align*}
  \arg e^{int} = \arg e^{ikt}
 \end{align*} означает
 \begin{align*}
  nt = kt + 2\pi m, \quad m \in\Z.
 \end{align*} если $ P(\eta=n)>0 $ и $ P(\eta=k)>0 $. Тогда
 \begin{align*}
  t = \frac{2\pi m}{n - k} \in \pi \Q
 \end{align*} таких $ t $ не более чем счётно, и мера равна нулю.
\end{proof}

Посмотрим теперь на большие размерности, пусть теперь мы шагаем не по $ \Z $, а по $ \Z^{d} $. Случайное блуждание по $ \Z^{d} $. Из каждой точки мы идём в соседнюю с вероятностями $ \frac{1}{2d} $.

\begin{thm}[теорема Пойя о возвращении]
 Такое случайное блуждание возвратно тогда и только тогда, когда $ d=1 $ или $ d=2 $.
\end{thm}

Пьяная ворона назад не вернётся.

\begin{proof}[\normalfont\textsc{Доказательство}]
 Для $ d=1 $ уже всё доказано.

 Посмотрим на $ d=2 $. Повернём на $ 45 $ градусов, рисунок \eqref{fig:theorem_2d_case_poia_rotation}. Будем смотреть проекцию точки на две диагональные оси. $ \eta_n $ и $ \tilde \eta_n $. Оказывается --- это независимые блуждания по повернутым осям (по $ \Z $). Надо посмотреть на один шаг и всё будет понятно.

\begin{figure}[ht]
    \centering
	\incfig[0.4]{theorem_2d_case_poia_rotation}
    \caption{Поворот плоскости на 45 градусов.}
    \label{fig:theorem_2d_case_poia_rotation}
\end{figure}

 \begin{align*}
  P(\vec\xi_n = \vec0) = P(\eta_n=0,\;\tilde\eta_n=0)=P(\eta_n=0)P(\tilde\eta_n=0) = \begin{cases}
   0, \text{ если } n \text{ неч}  \\
   (\binom{2n}n \frac{1}{4^{n}})^{2} \sim \left( \frac{1}{\sqrt{\pi n}} \right)^{2} = \frac{1}{\pi n} \text{ иначе }
  \end{cases} 
 \end{align*}
 Ряд
 \begin{align*}
  \sum_{n=1}^{\infty}P(\vec\xi_n=\vec 0)
 \end{align*} расходится, следовательно, блуждание возвратно.

 Пусть теперь $ d = 3 $.
 \begin{align*}
  P(\vec\xi_n = \vec 0) = \begin{cases}
   0, \text{ если $ n $ нечётно}  \\
   \text{ иначе: }
  \end{cases} 
 \end{align*}
 \begin{align*}
  P(\vec\xi_{2n} = \vec 0) = \sum_{i+j \leqslant n} \binom{2n}{i,i,j,j,n-i-j,n-i-j} \frac{1}{6^{2n}}
 \end{align*} (т.е. все способы вернуться делим на суммарное число способов). В
 \begin{align*}
  \frac{(2n)!}{(i!j!(n-i-j)!)^{2}} = \left(\frac{n!}{i!j!(n-i-j)!}\right)^{2} \cdot \frac{(2n)!}{(n!)^2} = \binom {2n} n \cdot \binom{n}{i,j,n-i-j}^{2}
 \end{align*} Сумму
 \begin{align*}
  = \binom{2n} n \frac{1}{6^{2n}} \sum_{i+j \leqslant n} \binom{n}{i,j,n-i-j}^{2}
 \end{align*} Знаем
 \begin{align*}
  \sum_{i+j \leqslant n} \binom{n}{i,j,n-i-j} = 3^{n}.
 \end{align*} Тогда сумма квадратов не превсохдит суммы умноженной на максимум:
 \begin{align*}
  \leqslant \binom{2n}n \frac{1}{6^{2n}} 3^{n} \max_{i+j \leqslant n} \binom{n}{i,j,n-i-j}.
 \end{align*} Но
 \begin{align*}
  \max_{i+j \leqslant n} \sim 3^{n} \cdot \frac{3\sqrt 3}{2\pi n}
 \end{align*} Надо взять и подставить
 \begin{align*}
  \binom n {\frac{n}{3},\frac{n}{3},\frac{n}{3}}
 \end{align*} в формулу Стирлинга. Тогда
 \begin{align*}
	 \sim \binom {2n} {n} \frac 1 {2^{2n}} \frac {3 \sqrt 3} {2 \pi n} \sim \frac{C}{n\sqrt n}.
 \end{align*} Ряд сходится!

 Теперь сведём $ d \geqslant 4 $ к $ d=3 $. Будем следить на проекцию на $ \Z^{3} $: либо там происходит случайное блуждание, но с некоторой вероятностью мы стоим. Бесконечно стоять можем с вероятностью $ 0 $, поэтому все стоянки можно выкинуть, и получим случайное блуждание на $ \Z^{3} $. Чтобы вернуться в $ Z^{d} $, надо как минимум вернуться в $ \Z^{3} $.
\end{proof}
\begin{exercs*}
 Доказать, что 
 \begin{align*}
  \binom{n}{i_0,j_0,n-i_0-j_)} = \max
 \end{align*} влечёт $ n-j_0-1 \leqslant 2i_0 \leqslant n - j_0 + 1 $, и $ n-i_0 - 1 \leqslant 2j_0 \leqslant n - i_0 + 1 $.
\end{exercs*}
\begin{exercs*}
 Рассмотрим блуждание по $ \Z^{d} $ по векторам вида $ (\pm 1, \ldots, \pm 1) $ равновероятно по $ \frac{1}{2^{d}} $. Доказать, что блуждание возвратно тогда и только тогда, когда $ d = 1 $ или $ d = 2 $.

 Это проще, чем теорема Пойя.
\end{exercs*}

Задача о разорении.

 Игрок $ A $ изначально имеет $ A $ монет, а игрок $ B $ --- $ B $ монет. Они играют в орлянку с кривой монеткой: с вероятностью $ q=1-p $ первый платит второму одну монету, а с вероятностью $ p $ --- второй первому. Игра заканчивается, когда игрок разорился.

 Можно процесс представить так: целые точки от $ -A $ до $ B $. Расстояние до $ -A $ --- количество монет у первого, а расстояние до $ B $ --- количество монет у второго. На концах петли. 

 Обозначим через $ \beta_k(x) $ вероятность через $ k $ шагов оказаться в $ B $, если на нулевом шаге мы находимся в $ x $. Напишем рекурренту:
 \begin{align*}
  \beta_k(x) = p \beta_{k-1}(x + 1) + q \beta_{k-1}(x-1)
 \end{align*} верно при $ -A < x < B $. Ещё мы знаем $ \beta_{k-1}(x) \leqslant \beta_k(x) \leqslant 1$. Перейдём к пределу при $ k \to \infty $:
 \begin{align*}
  \beta(x) = \lim_{k \to \infty} \beta_k(x).
 \end{align*} Тогда получим
 \begin{align*}
  \beta(x) = p\beta(x+1) +q\beta(x-1)
 \end{align*} Кроме того, $ \beta(-A) = 0 $ и $ \beta(B) = 1 $. Нужно покумекать и решить эту рекурренту (решить в общем виде и подобрать параметры). Ответ: если $ p \neq q $, то
 \begin{align*}
  \beta(x) = \frac{(q / p)^{x} - (q / p)^{-A}}{(q / p) ^{B} - (q/ p)^{-A}}.
 \end{align*} Если же $ p = q = 1 / 2 $, то
 \begin{align*}
  \beta(x) = \frac{x+A}{A+B}.
 \end{align*}

 $ \beta(0) $ --- вероятность разорения $B$. 

 Интересно: матожидание числа раундов до конца игры большое (примерно квадратично по количеству монет, если число монет примерно одинаковое).

\end{document}

