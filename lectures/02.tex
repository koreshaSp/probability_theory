% 2023.02.21: lecture 02
\documentclass[../main.tex]{subfiles}
\begin{document}

\newpage
\section{Предельные теоремы для схем Бернулли.}

В этом параграфе мы подробно изучим конкретный процесс в элементарной теории вероятностей, называемый \textit{схемой Бернулли}.

\subsection{Схема Бернулли.}

\begin{df}[схема Бернулли]
 Рассмотрим нечестную монетку, которая с вероятностью $p \in [0,1]$ выдаёт орёл (он же успех, он же $1$ --- единица), а с вероятностью $q = 1 - p$ выдаёт решку (она же неудача, она же $0$ --- нуль). \textit{Схемой Бернулли} называется следующий эксперимент: сделаем $ n $ независимых подбрасываний такой монетки.

 Формально, элементарным исходом в схеме Бернулли является последовательность результатов подбрасываний $\omega = (x_1, \ldots, x_n)$, $x_i \in \left\{ 0,1 \right\}$. Пространство элементарных исходов --- это набор всевозможных последовательностей результатов:
 \begin{align*}
  \Omega = \left\{ \omega = (x_1, \ldots, x_n) \mid x_i \in \left\{ 0,1 \right\} \right\}.
 \end{align*} Вероятность элементарного исхода $\omega = (x_1, \ldots, x_n)$ по определению равна
 \begin{align}
  \label{eq:bernulli_schema:probability}
  P(\left\{ \omega \right\}) = p^{\# \left\{ i \mid x_i = 1 \right\}} q^{\# \left\{ i \mid x_i = 0 \right\}} = p^{\sum_{i=1}^{n} x_i} q^{n - \sum_{i=1}^{n} x_i}
 .\end{align} 
\end{df}

\begin{prop}
 В схеме Бернулли события вида $\left\{ x_i = 0 \right\}$ по всем различным $i \in [n]$ независимы в совокупности. Аналогично, события вида $\left\{ x_i = 1 \right\}$  независимы в совокупности.
\end{prop}
\begin{proof}[\normalfont\textsc{Доказательство}]
 Непосредственно следует из формулы \eqref{eq:bernulli_schema:probability}.
\end{proof}

\begin{prop}
 Вероятность того, что в схеме Бернулли случилось ровно $k$ успехов (она же вероятность события $x_1 + \ldots + x_n = k$), равна
 \begin{align*}
  \binom n k p^{k} q^{n-k}
 .\end{align*} 
\end{prop}
\begin{proof}[\normalfont\textsc{Доказательство}]
 Действительно, есть ровно $ \binom n k $ элементарных исходов схемы Бернулли с $ k $ успехами, и вероятность каждого из них равна $ p^{k}q^{n-k} $.
\end{proof}

Забегая вперёд скажем, что число успехов в схеме Бернулли имеет \textit{биномиальное распределение}.

\subsection{Полиномиальная схема.}

Схему Бернулли можно естественным образом обобщить, разрешив монетке выдавать не только $ 0 $ или $ 1 $. Такое обобщение называется \textit{полиномиальной схемой}.
\begin{df}[полиномиальная схема]
 Пусть за один раз выпадают числа $ 1,2, \ldots, m $ с соответствующими вероятностями $p_1, \ldots, p_m \in [0,1]$, $p_1 + \ldots + p_m = 1$. \textit{Полиномиальной схемой} называется эксперимент, состоящий из $ n $ независимых повторений вышеуказанного шага.

 Формально, пространство элементарных исходов имеет вид
 \begin{align*}
  \Omega = \left\{ \omega = (x_1, \ldots, x_n) \mid x_i \in [m] \right\},
 \end{align*} вероятность элементарного исхода $ \omega = (x_1, \ldots, x_n) \in \Omega $ равна
 \begin{align*}
  P(\omega) = p_1^{\# \left\{ i \mid x_i = 1 \right\}} p_2^{\# \left\{ i \mid x_i = 2 \right\}} \ldots p_m^{\# \left\{ i \mid x_i = m \right\}}
 \end{align*} 
\end{df}

\begin{prop}
 В полиномиальной схеме
 \begin{align*}
  \left\{ x_1 = k_1 \right\}, \left\{ x_2=k_2 \right\}, \ldots, \left\{ x_n = k_n \right\}
 \end{align*} для всех $ k_1, k_2, \ldots, k_n \in [m] $ независимы в совокупности.
\end{prop}

\begin{prop}\

 В полиномиальной схеме вероятность того, что числа $ 1,2,\ldots, m $ выпали $ k_1 $ раз, $ k_2 $ раз, \ldots, $ k_m $ раз соответственно ($ k_1 + k_2 + \ldots + k_m=n $), равна
 \begin{align*}
  \binom n {k_1,\ldots,k_m} p_1^{k_1} p_2^{k_2} \ldots p_m^{k_m} = \frac{n!}{k_1! \ldots k_m!} p_1^{k_1} p_2^{k_2} \ldots p_m^{k_m}.
 \end{align*}
\end{prop}

Распределение кортежа $ (k_1, k_2, \ldots, k_m) $ называется \textit{полиномиальным распределением}.

\subsection{Теорема Пуассона.}

\begin{notatn*}
 Через $S_n$ обозначим количество успехов в схеме Бернулли с $n$ испытаниями.
\end{notatn*}

Так, мы уже показали равенство
\begin{align*}
 P(S_n = k) = \binom n k p^{k} q^{n-k}.
\end{align*} 

\begin{exmpl*}
 Что больше, $P(S_{1000} = 220)$ при $p = 1 / 5$ или $P(S_{2000} = 360)$ при $p = 1 / 6$?

 Честно считать эти вероятности неприятно. Численные ответы:
 \begin{align*}
  P(S_{1000} = 220) \text{ при $p = \frac{1}{5}$ } \approx 0.008984 > 0.006625 \approx P(S_{2000} = 360) \text{ при $p = \frac{1}{6}$}
 .\end{align*}
\end{exmpl*}

Исследуем следующий вопрос. Предположим, что мы рассматриваем схему Бернулли, в которой очень много испытаний, и в которой вероятность успеха $ p $ мала. Как можно оценить вероятность $ P(S_n = k) $? На этот вопрос отвечает теорема Пуассона.

\begin{thm}[%
 Пуассона]\
 \label{theorem:poisson}

 Рассмотрим последовательность схем Бернулли $B_1, B_2, \ldots$ У схемы Бернулли $B_n$ ровно $n$ испытаний и вероятность успеха $p_n$. Пусть при этом
 \begin{align*}
  p_n \sim \frac{\lambda}{n} 
 \end{align*} для некоторого $ \lambda > 0 $. Тогда при фиксированном $k$
 \begin{align*}
  P(S_n = k) = \binom n k p_n^{k} (1-p_n)^{n-k} \to \frac{\lambda^{k}}{k!} e^{-\lambda}
 \end{align*} при $n \to \infty$.
\end{thm}
\begin{proof}
 \begin{align*}
  \binom n k p_n^{k}(1-p_n)^{n-k} &= \frac{n\ldots(n-k+1)}{k!} \left( \frac{\lambda}{n} + o(1 / n) \right)^{k} \left(1 - \frac{\lambda}{n} + o (1 / n)\right)^{n-k} \sim \\
  & \sim \frac{n^{k}}{k!} \cdot \frac{\lambda^{k}}{n^{k}} \left( 1 - \frac{\lambda}{n} + o(1 / n) \right)^{n}  = \frac{\lambda^{k}}{k!} \left( 1 - \frac{\lambda}{n} + o(1 / n) \right)^{n}.
 \end{align*} Последний множитель стремится к $e^{-\lambda}$:
 \begin{align*}
  \log \left( 1 - \frac{\lambda}{n} + o(1 / n) \right)^{n} = n \log \left( 1 - \frac{\lambda}{n} + o(1 / n) \right) = n \left( - \frac{\lambda}{n} + o(1 / n) \right)  = - \lambda + o(1).
 \end{align*} 
\end{proof}

\begin{remrk}
 Если есть равенство
 \begin{align*}
  p_n = \frac{\lambda}{n},
 \end{align*} то теорема \ref{theorem:poisson} Пуассона верна не только при фиксированном $ k $, но и при $k = k(n) = o(\sqrt{n})$.
\end{remrk}
\begin{proof}[\normalfont\textsc{Доказательство}]
 \begin{align*}
  \binom n k p_n^{k} (1-p_n)^{n-k} &= \frac{n(n-1)\ldots(n-k+1)}{k!} \left( \frac{\lambda}{n} \right)^{k}\left(1 - \frac{\lambda}{n}\right)^{n-k} = \\
  &=  \frac{\lambda^{k}}{k!} \cdot \frac{n}{n} \cdot \frac{n-1}{n} \cdot \ldots \cdot \frac{n-k+1}{n} \cdot \left( 1 - \frac{\lambda}{n} \right)^{n} \left( 1 - \frac{\lambda}{n} \right)^{-k} \sim \\
  &\sim \frac{\lambda^{k}e^{-\lambda}}{k!} \left( 1 - \frac{1}{n} \right) \ldots \left(1 - \frac{k - 1}{n}\right) \left( 1 - \frac{\lambda}{n} \right)^{-k}
 .\end{align*} Последний множитель стремится к $1$:
 \begin{align*}
  \log \left( 1 - \frac{\lambda}{n} \right)^{-k} = -k \log \left( 1 - \frac{\lambda}{n} \right) \sim \frac{\lambda k}{n} \to 0
 .\end{align*} Произведение скобочек тоже стремится к $1$:
 \begin{align*}
  1 \geqslant \left( 1 - \frac{1}{n} \right) \ldots \left( 1 - \frac{k-1}{n} \right) \geqslant 1 - \frac{1}{n} - \frac{2}{n} - \ldots - \frac{k - 1}{n} = 1- \frac{k(k-1)}{2n} \to 1,
 \end{align*} ведь $ k = o(\sqrt n) $. Действительно, верно даже более общее неравенство
 \begin{align}
  \label{equation:induction_braces_inequality}
  (1-x_1)(1-x_2)\ldots(1-x_n) \geqslant 1 - x_1 - x_2 - \ldots - x_n
 \end{align}  при $0 \leqslant x_i \leqslant 1$.
\end{proof}

\begin{proof}[\normalfont\textsc{Доказательство неравенства \eqref{equation:induction_braces_inequality}}]

 Пусть события $ A_1, \ldots, A_n $  независимы, и происходят с вероятностями $ x_1, \ldots, x_n $. Тогда по неравенству union bound \eqref{eq:union_bound}:
 \begin{align*}
  P \left( \bigcup_{k=1}^{n}A_k \right) \leqslant x_1 + x_2 +  \ldots + x_n.
 \end{align*} С другой стороны, по независимости событий $ \overline{A_1},\ldots,\overline{A_n} $  имеем
 \begin{align*}
  P \left( \overline{ \bigcup_{k=1}^{n}A_k} \right) = P \left( \bigcap_{k=1}^{n} \overline{A_k} \right) = (1-x_1)(1-x_2) \ldots (1-x_n).
 \end{align*} Совмещая вместе, получаем
 \begin{align*}
  (1-x_1)(1-x_2)\ldots(1-x_n) &= 1 - P \left( \bigcup_{k=1}^{n}A_k \right) \geqslant 1 - x_1 -x_2 - \ldots - x_n.
 \end{align*}
\end{proof}

Есть оценка на скорость сходимости в теореме Пуассона.

\begin{thm}[%
 Прохорова]
 Пусть в условиях теоремы \eqref{theorem:poisson} Пуассона есть равенство
 \begin{align*}
  p_n = \frac{\lambda}{n}.
 \end{align*} Тогда для каждого $ n \geqslant 1 $ верно
 \begin{align*}
  \sum_{k=0}^{\infty} \left| P(S_n = k) - \frac{\lambda^{k}e^{-\lambda}}{k!}\right| \leqslant \frac{2\lambda}{n} \min \left\{ 2, \lambda \right\}.
 \end{align*} 
\end{thm}

\begin{exmpl}
 Есть честная рулетка: в ней есть $ 36 $ секторов, пронумерованных числами $ 1,2, \ldots, 36 $ и $ 37 $-й сектор <<зеро>>, пронумерованный числом $ 0 $. Вероятность выпадения каждого числа равна $ p = \frac{1}{37} $.

 Допустим, мы играем в рулетку $n = 111=37\cdot3$ раз (то есть $ \lambda = np = 3 $), каждый раз ставим одну монетку (допустим, на случайное число). Если мы угадали, то получаем $37$ монет.

 Какова вероятность того, что за $111$ испытаний мы вернули ровно то, что поставили (то есть выиграли ровно $ 3 = \frac{111}{37} $ раза)? Точное вычисление:
 \begin{align*}
  P(S_{111} = 3) = \binom {111} 3 \left( \frac{1}{37} \right)^{3} \left( \frac{36}{37} \right)^{108} \approx 0.227127\ldots
 \end{align*} Приближённое вычисление по теореме \ref{theorem:poisson} Пуассона:
 \begin{align*}
  P(S_{111} = 3) \approx \frac{\lambda^{3}}{3!}e^{-\lambda} = \frac{3^{3}}{3!} e^{-3} = \frac{4.5}{e^{3}} \approx 0.224041\ldots
 \end{align*} Два знака после запятой совпали!

 Теперь оценим вероятность выиграть:
 \begin{align*}
  P(S_{111} \geqslant 4) = 1 - P(S_{111} = 0) - \ldots - P(S_{111} = 3)
 .\end{align*} Если считать точно, то получится около $0.352768\ldots$, а если по теореме \ref{theorem:poisson} Пуассона, то
 \begin{align*}
  1 - \frac{3^{0}}{0!}e^{-3} - \frac{3^{1}}{1!}e^{-3} - \frac{3^{2}}{2!}e^{-3} + \frac{3^{3}}{3!}e^{-3} = 1 - \frac{13}{e^{3}} \approx 0.352754\ldots
 \end{align*} 
\end{exmpl}

\subsection{Локальная теорема Муавра-Лапласа.}

Теорема \ref{theorem:poisson} Пуассона осмысленна, только если вероятность успеха $p$ маленькая. В случае, когда $ p $ фиксирована, применимы другие результаты такие, как следующая теорема.
\begin{thm}[%
 локальная теорема Муавра-Лапласа]
 \label{theorem:local_theorem_muavr_laplas}
 Пусть есть фиксированное число $0 < p < 1$, $q = 1 - p$. Пусть $n \to \infty$, и целое неотрицательное число $ k $ зависит от $ n $ так, что выполнено $ \left| x \right| \leqslant T $, где
 \begin{align*}
  x =\frac{k - np}{\sqrt{npq}},
 \end{align*} а $ T $ --- заведомо известная константа. Рассмотрим схемы Бернулли с $n$ испытаниями и вероятностью успеха $p$. Тогда
 \begin{align*}
  P(S_n = k) \sim \frac{1}{\sqrt{2\pi npq }}\cdot e^{-x^{2} / 2}
 \end{align*} при $n \to \infty$. При этом, эквивалентность равномерная по $ k $.
\end{thm}
\begin{proof}[\normalfont\textsc{Доказательство}]
 Заметим, что
 \begin{align*}
  k = np + x \sqrt{npq}, && n - k = nq - x \sqrt{npq}.
 \end{align*} Так как $ \left| x \right| \leqslant T $, то
 \begin{align*}
  k \geqslant np - T \sqrt{npq} \to +\infty, \\
  n - k \geqslant nq - T \sqrt{npq} \to +\infty
 .\end{align*} Обозначим
 \begin{align}
  \label{eq:proof:local_thm_muav_lap:alpha}
  \alpha = \frac{k}{n} = p + x\sqrt{\frac{pq}{n}} \to p
 .\end{align} Соответственно,
 \begin{align}
  \label{eq:proof:local_thm_muav_lap:alpha_compl}
  1 - \alpha = \frac{n - k}{n} = q - x \sqrt{\frac{pq}{n}} \to q.
 \end{align} 

 Каждая из трёх величин $n$, $k$, $n - k$ стремится к $+\infty$ при $ n \to \infty $, поэтому для выражений $ n! $, $ k! $, $ (n-k)! $ можно написать формулу Стирлинга:
 \begin{align*}
  P(S_n = k) &= \frac{n!}{k!(n-k)!} \cdot p^{k}q^{n-k} \sim \\
  &\sim \frac{n^{n}e^{-n}\sqrt{2\pi n} \cdot p^{k}q^{n-k}}{k^{k}e^{-k}\sqrt{2\pi k}(n-k)^{n-k}e^{-n+k}\sqrt{2\pi (n-k)}} = \\
  &= \frac{1}{\sqrt{2\pi}} \cdot \frac{n^{n}\sqrt{n} \cdot p^{k}q^{n-k}}{k^{k}\sqrt{k}(n-k)^{n-k} \sqrt{n-k}} = \\
  &= \frac{1}{\sqrt{2 \pi n \cdot \frac{k}{n} \cdot \frac{n-k}{n}}} \cdot \frac{p^{k}q^{n-k}}{(\frac{k}{n})^{k}(\frac{n-k}{n})^{n-k}} = \\
  &= \frac{1}{\sqrt{2 \pi n \alpha (1-\alpha)}} \cdot \frac{p^{k}q^{n-k}}{\alpha^{k}(1 - \alpha)^{n-k}} \sim \\
  &\sim \frac{1}{\sqrt{2\pi n p q}} \cdot \left(\frac{p}{\alpha}\right)^{k} \left( \frac{q}{1 - \alpha} \right)^{n-k}.
 \end{align*} Осталось доказать, что
 \begin{align*}
  \left( \frac{p}{\alpha} \right)^{k} \left( \frac{q}{1-\alpha} \right)^{n-k} \sim e^{-x^{2} / 2},
 \end{align*} что эквивалентно
 \begin{align*}
  k \log \frac{\alpha}{p} + (n-k) \log \frac{1 - \alpha}{q} \sim \frac{x^{2}}{2}
 .\end{align*} Раскроем $ \alpha $ и $ 1 - \alpha $ в левой части с помощью \eqref{eq:proof:local_thm_muav_lap:alpha} и \eqref{eq:proof:local_thm_muav_lap:alpha_compl}:
 \begin{align*}
  k \log\left(1 + x \sqrt{\frac{q}{np}}\right) + (n-k)\log \left( 1 - x \sqrt{\frac{p}{nq}} \right).
 \end{align*} Применим формулу Тейлора $\log(1 + t) = t - \frac{t^{2}}{2} + o(t^{2})$:
 \begin{align*}
  =\;&k \left[x \sqrt{\frac{q}{np}} - \frac{x^{2}}{2} \cdot \frac{q}{np} + o(1 / n)\right] + (n-k) \left[ -x \sqrt{\frac{p}{nq}}  - \frac{x^{2}}{2} \cdot \frac{p}{nq} + o(1 / n) \right] = \\
  =\;&k x \sqrt{\frac{q}{np}} - \frac{x^{2}}{2}\cdot\frac{kq}{np} - (n-k)x \sqrt{\frac{p}{nq}} - \frac{x^{2}}{2} \cdot \frac{(n-k)p}{nq} + o(1).
 \end{align*} Вспомним, что $k \sim np$ и $n-k\sim nq$:
 \begin{align*}
  &=kx \sqrt{\frac{q}{np}} - (n-k)x \sqrt{\frac{p}{nq}} - \frac{x^{2}}{2} \cdot q - \frac{x^{2}}{2} \cdot p + o(1) = \\
  &= (np + x \sqrt{npq})x \sqrt{\frac{q}{np}} - (nq - x\sqrt{npq})x\sqrt{\frac{p}{nq}} - \frac{x^{2}}{2} + o(1) = \\
  &= x\sqrt{npq} + x^{2} q - x\sqrt{npq} + x^{2}p - \frac{x^{2}}{2} + o(1) = \\
  &= \frac{x^{2}}{2} + o(1).
 \end{align*}
\end{proof}

\begin{remrk}
 Если $ \left| k-np \right| = o(n^{2 / 3}) $, то верен тот же вывод из теоремы \ref{theorem:local_theorem_muavr_laplas}.
\end{remrk}

\begin{exmpl}
 Тот же пример с рулеткой. Пусть игрок ставит на красное $n = 222$ раз. Вероятность выигрыша при одной ставке равна $p = \frac{18}{37}$, и выигрываем мы при этом две монеты.

 Тогда вероятность того, что мы выйдем в ноль равна $ P(S_{222} = 111) $. Точное вычисление:
 \begin{align*}
  P(S_{222} = 111) \approx 0.0493228...
 \end{align*} Приближённое вычисление с помощью локальной теоремы Муавра-Лапласа:
 \begin{align*}
  \frac{1}{\sqrt{2 \pi n p q}} e^{-x^2/2} \approx 0.0493950...
 \end{align*}
\end{exmpl}

\subsection{Интегральная теорема Муавра-Лапласа.}

\begin{thm}[%
 интегральная теорема Муавра-Лапласа]
 \label{theorem:intergram_theorem_muavr_laplas}
 Пусть вероятность успеха $ p \in (0,1)$  в схеме Бернулли фиксирована. Тогда для любых чисел $ a < b $ выполнено
 \begin{align}
  \label{eq:integral_theorem_muavr_lapplas}
  \lim_{n \to \infty} P\left(a < \frac{S_n - np}{\sqrt{npq}} \leqslant b\right) = \frac{1}{\sqrt{2\pi}} \int_{a}^{b} e^{-x^{2}/2}\,dx
 ,\end{align} причём сходимость равномерная по $a, b \in \R$.
\end{thm}

Сейчас мы не будем доказывать теорему \ref{theorem:intergram_theorem_muavr_laplas}. Она будет тривиальным частным случаем центральной предельной теоремы \ref{theorem:central_limit_theorem_Levi}.

\begin{notatn*}
 Есть стандартное обозначение
 \begin{align*}
  \Phi(x) = \frac{1}{\sqrt{2\pi}} \int_{-\infty}^{x} e^{-t^{2}/2}\,dt
 .\end{align*} 
\end{notatn*}

Так, интегральная теорема \ref{theorem:intergram_theorem_muavr_laplas} Муавра-Лапласа утверждает
\begin{align*}
 \lim_{n \to \infty} P \left( a < \frac{S_n-np}{\sqrt{npq}} \leqslant b \right) =\Phi(b)-\Phi(a).
\end{align*}

Есть оценка на скорость сходимости в теореме \ref{theorem:intergram_theorem_muavr_laplas}.

\begin{thm}[%
 Берри-Эссеена]
 \begin{align*}
  \sup_{x \in \R} \left| P \left( \frac{S_n - np}{\sqrt{npq}} \leqslant x \right) - \Phi(x) \right| \leqslant \frac{p^{2}+q^{2}}{\sqrt{npq}} \cdot \frac{1}{2}
 .\end{align*}
\end{thm}

\begin{remrk}
 Константу $\frac{1}{2}$ можно заменить на $0.469$.
\end{remrk}

\begin{remrk}
 Показатель степени $n$ в знаменателе (то есть $\frac{1}{2}$) нельзя уменьшить. Рассмотрим следующий пример.
\end{remrk}

\begin{exmpl}
 Пусть в схеме Бернулли $p = q = 1 / 2$ и $2n$ испытаний.
 \begin{align*}
  P(S_{2n} = n) = \binom {2n} n \left( \frac{1}{2} \right)^{2n} \sim \frac{4^{n}}{\sqrt{\pi n}} \left( \frac{1}{2} \right)^{2n} = \frac{1}{\sqrt{\pi n}}.
 \end{align*} Тогда
 \begin{align*}
  P \left( \frac{S_{2n} - 2np}{\sqrt{2npq}} \leqslant 0 \right) &= P(S_{2n} \leqslant n) = P(S_{2n} < n) + P(S_{2n} = n) = \\
  &= P(S_{2n} > n) + P(S_{2n} = n) = \frac{1 + P(S_{2n} = n)}{2} = \\
  &= \frac{1}{2} + \frac{1}{2 \sqrt{\pi n}} + o \left( \frac{1}{\sqrt{n}} \right).
 \end{align*} Так как $ \Phi(0) = \frac{1}{\sqrt{2\pi}} \int_{-\infty}^{0} e^{-t^{2} / 2}\,dt = 1 / 2 $, то
 \begin{align*}
  \left|P \left( \frac{S_{2n}-2np}{\sqrt{2npq}} \leqslant 0 \right) - \Phi(0) \right| = \frac{1}{2\sqrt{\pi n}} + o \left( \frac{1}{\sqrt n} \right).
 \end{align*} Таким образом, уменьшить показатель степени действительно нельзя.
\end{exmpl}

\begin{remrk}
 Предположим, что $ y \in \Z $ и нам необходимо приближенно вычислить $ P(S_n \leqslant y) $. По интегральной теореме \ref{theorem:intergram_theorem_muavr_laplas} Муавра-Лапласа
 \begin{align*}
  P(S_n \leqslant y) = P\left(\frac{S_n - np}{\sqrt{npq}} \leqslant \frac{y - np}{\sqrt{npq}}\right) \approx \Phi \left( \frac{y-np}{\sqrt{npq}} \right)
 .\end{align*} Но число $S_n$ также целое, как и $ y $. Поэтому, можно получить более точное вычисление, сдвинув $ y $ на $ 1 / 2 $:
 \begin{align*}
  P(S_n \leqslant y) = P\left(S_n \leqslant y + \frac{1}{2}\right) \approx \Phi \left( \frac{y + \frac{1}{2} - np}{\sqrt{npq}} \right).
 \end{align*}
\end{remrk}

\end{document}
