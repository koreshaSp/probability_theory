% 2023.04.04 lecture 08

\documentclass[../main.tex]{subfiles}

\begin{document}

\begin{exmpl}[характеристическая функция нормального распределения]
 \label{example:char_func_of_normal_distribution}
 Найдём сначала характеристическую функцию стандартного нормального распределения $ \xi \sim \Norm(0,1) $:
 \begin{align*}
  \varphi_\xi(t) &= \E e^{it\xi}  = \frac{1}{\sqrt{2\pi}} \int_{\R} e^{itx}  \cdot e^{-x^{2} / 2}\,dx = \frac{1}{\sqrt{2\pi}} \int_{\R} e^{-\frac{1}{2}(x^{2}-2itx)}\,dx  = \\
  &=\frac{1}{\sqrt{2\pi}} \int_{\R} e^{-\frac{1}{2}(x^{2}-2itx - t^{2} + t^{2})}\,dx   = \frac{e^{-t^{2} / 2}}{\sqrt{2\pi} } \int_{\R} e^{-(x-it)^{2} / 2} \,dx.
 \end{align*} Вычислим интеграл
 \begin{align*}
  I = \int_{\R} e^{-(x-it)^{2} / 2}\,dx = \lim_{R \to +\infty}  \int_{-R-it}^{R-it} e^{-z^{2} / 2}\,dz
 \end{align*} методами комплексного анализа: проинтегрируем функцию $ f(z) = e^{-z^{2} / 2} $ по контуру $ \Gamma_R $ --- границе прямоугольника с вершинами в точках $ R, -R, -R - it, R - it $.
 \begin{figure}[ht]
  \centering
  \incfig[0.9]{characteristic_function_normall}
  \caption{Контур интегрирования $ \Gamma_R $.}
  \label{fig:characteristic_function_normall}
 \end{figure}

 С одной стороны, так как $ f $ --- целая функция и не имеет особенностей внутри контура, то
 \begin{align}
  \label{eq:norm_char_func:integral_zero}
  \varointctrclockwise_{\Gamma_R} f\,dz = 0.
 \end{align} С другой стороны,
 \begin{align}
  \label{eq:norm_char_func:integral_contour}
  \varointctrclockwise_{\Gamma_R} f\,dz = \int_{-R-it}^{R-it} f\,dz + \int_{R - it}^{R} f\,dz + \int_{R}^{-R} f\,dz + \int_{-R}^{-R-it} f\,dz.
 \end{align}

 Оценим интегралы по вертикальным отрезкам при $ R \to \infty $:
 \begin{align*}
  \left| \int_{R-it}^{R} e^{-z^{2} / 2}\,dz \right| &\leqslant t \cdot \max_{-t \leqslant y \leqslant 0} \left| e^{-(R+iy)^{2} / 2} \right| = t \cdot \left| e^{-(R+\OO(1))^{2} / 2} \right| \to 0,\\
  \left| \int_{-R}^{-R-it} e^{-z^{2} / 2}\,dz \right| &\leqslant t \cdot \max_{-t \leqslant y \leqslant 0} \left| e^{-(-R+iy)^{2} / 2} \right| = t \cdot \left| e^{-(-R+\OO(1))^{2} / 2} \right| \to 0.
 \end{align*} Совмещая полученное с \eqref{eq:norm_char_func:integral_zero} и \eqref{eq:norm_char_func:integral_contour}, получаем
 \begin{align*}
  I = \lim_{R \to +\infty}  \int_{-R}^{R} e^{-x^{2} / 2}\,dx = \int_{\R} e^{-x^{2} / 2}\,dx  = \sqrt{2\pi}.
 \end{align*}

 Итого, характеристическая функция стандартного нормального распределения $ \Norm(0,1) $ равна
 \begin{align*}
  \varphi_\xi(t) = e^{-t^{2} / 2}.
 \end{align*}

 Теперь пусть $ \eta \sim \Norm(\mu, \sigma^{2}) $. Тогда $ \eta = \sigma \xi + \mu $, где $ \xi \sim \Norm(0,1) $. По свойствам \ref{prop:char_func_plus_const} и \ref{prop:char_func_times_const} предложения \ref{prop:char_func_properties} получаем, что характеристическая функция нормального распределения $ \Norm(\mu,\sigma^{2}) $ равна
 \begin{align*}
  \varphi_\eta(t) = \varphi_{\sigma \xi + \mu}(t) = e^{i \mu t} \varphi_\xi(\sigma t) = e^{i \mu t - \frac{\sigma^{2}t^{2}}{2}}.
 \end{align*}
\end{exmpl}

\subsection{Связь характеристической функции с моментами.}

\begin{thm}[гладкость характеристической функции случайной величины с моментами]
 \label{theorem:moments_imply_char_func_smoothness}
 Пусть $ \xi $ --- вещественнозначная случайная величина, имеющая $ n $-е моменты:
 \begin{align*}
  \E \left| \xi \right|^{n} < +\infty.
 \end{align*} Тогда её характеристическая функция $ \varphi_\xi(t) $ является $ C^{n} $-гладкой на $ \R $, и для каждого $ k = 0, 1, \ldots, n $ выполнено
 \begin{align*}
  \varphi_\xi^{(k)}(t) = i^{k} \cdot \E (\xi^{k} e^{it\xi})
 \end{align*} при всех $ t \in \R $. В частности,
 \begin{align*}
  \varphi_\xi^{(k)}(0) = i^{k} \cdot \E\xi^{k}.
 \end{align*}
\end{thm}
\begin{proof}[\normalfont\textsc{Доказательство}]
 Докажем индукцией по $ k $.

 База: $ \varphi_\xi^{(0)}(t) = \E e^{it\xi} $ --- определение характеристической функции.

 Переход $ k \mapsto k + 1 $, где $ k +1 \leqslant n $. Для каждой точки $ t \in \R $ вычислим производную функции $ \varphi_\xi^{(k)} $  в точке $ t $ по определению:
 \begin{align*}
  \varphi_\xi^{(k+1)}(t) &= \lim_{h \to 0} \frac{\varphi_\xi^{(k)}(t+h) - \varphi_\xi^{(k)}(t)}{h} = \lim_{h \to 0} \frac{i^{k} \cdot \E(\xi^{k}e^{i(t+h)\xi}) - i^{k} \cdot \E(\xi^{k}e^{it\xi})}{h} = \\
  &= i^{k} \lim_{h \to 0} \E \frac{\xi^{k}e^{it\xi} (e^{ih\xi} - 1)}{h} = i^{k} \lim_{h \to 0} \int_{\R} x^{k}e^{itx} \cdot \frac{e^{ihx}-1}{h}\,dP_\xi(x).
 \end{align*} Заметим, что при всех $ x \in \R $ и при $ h \to 0 $ для подынтегральной функции верно
 \begin{align*}
  x^{k}e^{itx} \cdot \frac{e^{ihx} - 1}{h}  = x^{k}e^{itx} \cdot \frac{1 + ihx + o(h) - 1}{h} = ix^{k+1}e^{itx} + o(1).
 \end{align*} Если мы обоснуем, что можно поменять местами знаки интеграла и предела, то мы получим нужное:
 \begin{align*}
  \varphi_\xi^{(k+1)}(t) = i^{k} \int_{\R} ix^{k+1}e^{itx}\,dP_\xi(x) =    i^{k+1} \cdot \E (\xi^{k+1}e^{it\xi}).
 \end{align*}

 Обосновывать мы будем по теореме Лебега о мажорируемой сходимости. Для этого нам нужно соорудить мажоранту --- функцию, которая не зависит от $ h $, суммируема по мере $ P_\xi $, и всюду ограничивает сверху следующее выражение
 \begin{align*}
  \left| x^{k}e^{itx} \cdot \frac{e^{ihx}-1}{h} \right| = \left| x \right|^{k} \cdot \left| \frac{e^{ihx}-1}{h} \right|.
 \end{align*}

 Если $ \left| hx \right| > 1 $, то
 \begin{align*}
  \left| \frac{e^{ihx}-1}{h} \right| \leqslant \frac{2}{\left| h \right|} \leqslant 2 \left| x \right|.
 \end{align*}

 Если же $ \left| h x \right| \leqslant 1 $, то можно считать, что все функции заданы на компакте, и поэтому можно написать
 \begin{align*}
  \left| \frac{e^{ihx}-1}{h} \right| = \left| \frac{1 + \OO(ihx) - 1}{h} \right| = \OO(x) \leqslant C \left| x \right|
 \end{align*} для некоторой константы $ C>0 $. Так или иначе, получаем
 \begin{align*}
  \left| x^{k}e^{itx}\frac{e^{ihx}-1}{h} \right| \leqslant C \left| x \right|^{k+1}
 \end{align*} для некоторого $ C > 0 $. Но функция $ x \mapsto  \left| x \right|^{k+1} $ суммируема по мере $ P_\xi $, ведь $ \E \left| \xi \right|^{k+1}  < +\infty$. Мажоранта найдена.

 Осталось показать непрерывность последней производной $ \varphi_\xi^{(n)}(t) $:
 \begin{align*}
  \varphi_\xi^{(n)}(t + h) - \varphi_\xi^{(n)}(t) &= i^{n} \cdot \E(\xi^{n}e^{it\xi}(e^{ih\xi} - 1)) = i^{n} \cdot \E(\xi^{n}e^{it\xi} \cdot o(1)) = o(1).
 \end{align*} Снова применяем теорему Лебега о мажорируемой сходимости, но здесь мажоранта строится проще:
 \begin{align*}
  \left| x^{n}e^{itx}(e^{ihx} - 1) \right| \leqslant 2 \left| x^{n} \right|,
 \end{align*} где функция $ x \mapsto \left| x^{n} \right| $ суммируема по мере $ P_\xi $, так как $ \E \left| \xi \right|^{n} < +\infty $.
\end{proof}

\begin{crly}[матожидание и дисперсия в терминах характеристической функции]
 Пусть вещественнозначная случайная величина $ \xi $ имеет вторые моменты. Тогда
 \begin{align*}
  \E\xi = -i \varphi'_\xi(0), &&
  \D\xi = -\varphi''_\xi(0) + \left( \varphi'_\xi(0) \right)^{2}.
 \end{align*}
\end{crly}
\begin{proof}[\normalfont\textsc{Доказательство}]
 Применим теорему \ref{theorem:moments_imply_char_func_smoothness}:
 \begin{align*}
  \varphi_\xi'(0) = i \cdot \E\xi \implies \E\xi = -i \varphi_\xi'(0),\\
  \varphi_\xi''(0) = -\E\xi^{2} \implies \E\xi^{2} = -\varphi_\xi''(0),\\
  \D\xi = \E\xi^{2}-(\E\xi)^{2} = -\varphi_\xi''(0) + (\varphi_\xi''(0))^{2}.
 \end{align*}
\end{proof}

Связь между гладкостью характеристической функции и моментами случайной величины частично работает и в другую сторону.

\begin{thm}
 \label{theorem:char_func_2diff_at_0_implies_2nd_moment}
 Если у характеристической функции вещественнозначной случайной величины $ \xi $ существует конечная вторая производная в нуле $ \varphi''_\xi(0) $, то у $ \xi $ есть конечный второй момент:
 \begin{align*}
  \E\xi^{2} < +\infty.
 \end{align*}
\end{thm}
\begin{remrk}
 \label{remark:general_char_func_2n_diff_at_0_implies_2n_moment}
 В общем виде теорема выглядит так: если существует конечная производная $ \varphi^{(2n)}_\xi(0) $, то и существуют моменты порядка $ 2n $: $ \E\xi^{2n} < +\infty $.
\end{remrk}
\begin{remrk}
 В частности, из теоремы \ref{theorem:char_func_2diff_at_0_implies_2nd_moment} следует, что если $ \varphi_\xi $ дважды дифференцируема в нуле, то функция $ \varphi_\xi $ $ C^{2} $-гладкая.
\end{remrk}
\begin{proof}[\normalfont\textsc{Доказательство теоремы \ref{theorem:char_func_2diff_at_0_implies_2nd_moment}}]\

 Перепишем крайне не очевидным образом подынтегральную функцию $ x^{2} $ в математическом ожидании:
 \begin{align*}
  \E\xi^{2} &= \int_{\R} x^{2}\,dP_\xi(x) = \int_{\R} \lim_{t \to 0} \left(\frac{\sin(tx)}{t} \right)  ^{2}dP_\xi(x),
 \end{align*} ведь $ \sin(tx) = tx + o(t) $ при $ t \to 0 $. Так как подынтегральная функция неотрицательна, то по лемме Фату
 \begin{align*}
  \E\xi^{2} \leqslant \liminf_{t \to 0} \int_{\R} \left( \frac{\sin(tx)}{t} \right)^{2}\,dP_\xi(x).
 \end{align*} Раскроем синус через комплексные экспоненты:
 \begin{align*}
  (\sin(tx))^{2} = \left( \frac{e^{itx} - e^{-itx}}{2i} \right)^{2} = -\frac{e^{2itx} + e^{-2itx}-2}{4} = \frac{2-e^{2itx}-e^{-2itx}}{4}.
 \end{align*} Тогда
 \begin{align*}
  \E\xi^{2} &\leqslant \liminf_{t \to 0} \frac{1}{4t^{2}} \int_{\R} \left( 2-e^{2itx} -e^{-2itx} \right) \,dP_\xi(x) = \\
  &=\liminf_{ t\to 0} \frac{2  - \varphi_\xi(2t) - \varphi_\xi(-2t)}{4t^{2}}.
 \end{align*} Запишем формулу Тейлора функции $ \varphi_\xi $ в точке $ 0 $:
 \begin{align*}
  \varphi_\xi(s) = 1 + \varphi'_\xi(0) \cdot s + \frac{\varphi''_\xi(0)}{2}s^{2} + o(s^{2})
 \end{align*} при $ s \to 0 $. Тогда
 \begin{align*}
  \varphi_\xi(2t) + \varphi_\xi(-2t) = 2 + \varphi''_\xi(0) \cdot 4t^{2} + o(t^{2}).
 \end{align*} Подставляя полученную оценку обратно, получаем
 \begin{align*}
  \E\xi^{2} \leqslant \liminf_{t \to 0} (-\varphi_\xi''(0) + o(1)) = -\varphi_\xi''(0).
 \end{align*}
 Общая ситуация доказывается точно так же, только много писать. Вторая производная в нуле всегда отрицательна.
\end{proof}

Общая формулировка, упомянутая в замечании \ref{remark:general_char_func_2n_diff_at_0_implies_2n_moment}, доказывается так же, но получатся больше выкладок. 

\begin{crly}
 $ \varphi_\xi''(0) \leqslant 0 $.
\end{crly}

\subsection{Формула обращения.}

Мы знаем, что характеристическая функция зависит только от распределения случайной величины. Оказывается, что верно и обратное: по характеристической функции можно однозначно восстановить распределение случайной величины. Этот результат называется \textit{формулой обращения}.

Формула обращения порождает следующий удобной метод рассуждения. Если мы хотим доказать, что некоторая случайная величина обладает таким-то распределением, то нам достаточно вычислить характеристическую функцию этой случайной величины и показать, что она совпадает с характеристической функцией желанного распределения.

\begin{thm}[формула обращения]
 \label{theorem:inversing_formula}
 Пусть $ \xi $ --- вещественнозначная случайная величина, а числа $ a < b $ --- точки непрерывности её функции распределения: $ P_\xi(\left\{ a \right\}) = P_\xi(\left\{ b \right\}) = 0 $. Тогда верна \textbf{формула обращения}
 \begin{align*}
  P(a \leqslant \xi \leqslant b) &= \frac{1}{2\pi} \lim_{T \to +\infty} \int_{-T}^{T}  \frac{e^{-iat}-e^{-ibt}}{it} \cdot \varphi_\xi(t)\,dt = \\
  &= \frac{1}{2\pi} \operatorname{p.\!v.} \int_{\R} \frac{e^{-iat}-e^{-ibt}}{it}\cdot\varphi_\xi(t)\,dt.
 \end{align*}
\end{thm}
\begin{proof}[\normalfont\textsc{Доказательство}]
 Рассмотрим сначала случай, когда $ a = -1 $, $ b = 1 $. Нужно доказать
 \begin{align*}
  P(-1 \leqslant \xi \leqslant 1) = \lim_{T \to +\infty} \frac{1}{2\pi}\int_{-T}^{T} \frac{e^{it}-e^{-it}}{it}\cdot\varphi_\xi(t)\,dt.
 \end{align*} Распишем интеграл в правой части:
 \begin{align*}
  \int_{-T}^{T} \frac{e^{it}-e^{-it}}{it}\cdot\varphi_\xi(t)\,dt &= \int_{-T}^{T} \int_\R \frac{e^{it}-e^{-it}}{it} \cdot e^{itx}\,dP_\xi(x)\,dt.
 \end{align*} Мы хотим воспользоваться теоремой Фубини и переставить интегралы местами. Для этого нам достаточно показать ограниченность выражения
 \begin{align*}
  \left| \frac{e^{it}-e^{-it}}{it} \cdot e^{itx} \right| = \left| \frac{e^{it}-e^{-it}}{it} \right|,
 \end{align*} ведь меры конечные ($ P_\xi $ --- вероятностная мера, а внешний интеграл берётся по ограниченному подмножеству $ \R $). 

 Если $ t > 1 $, то
 \begin{align*}
  \left| \frac{e^{it}-e^{-it}}{it} \right| \leqslant \frac{2}{\left| t \right|} < 2.
 \end{align*} 

 Если же $ t \leqslant 1 $, то
 \begin{align*}
  \left| \frac{e^{it}-e^{-it}}{it} \right| = \left| \frac{1 + it - 1 - (-it) + o(t)}{it} \right| = \left| 2 + o(1) \right|
 \end{align*} --- ограничено, так как $ t $ лежит в компакте  $ [-1,1] $.

 Так или иначе получаем
 \begin{align*}
  \int_{-T}^{T} \frac{e^{it}-e^{-it}}{it} \cdot \varphi_\xi(t)\,dt = \int_{\R} \int_{-T}^{T} \frac{e^{it}-e^{-it}}{it}  \cdot e^{itx}\,dt\,dP_\xi(x).
 \end{align*}

 Далее обозначим и перепишем внутренний интеграл:
 \begin{align*}
  \Phi_T(x) &= \int_{-T}^{T} \frac{e^{it}-e^{-it}}{it} \cdot e^{itx}\,dt = \int_{-T}^{T} \left(\int_{-1}^{1} e^{itu}\,du \right) e^{itx}\,dt = \int_{-T}^{T} \int_{-1}^{1} e^{it(u+x)}\,du\,dt.
 \end{align*} Так как $ \left| e^{it(u+x)} \right| = 1 $, то по теореме Фубини
 \begin{align*}
  \Phi_T(x) &= \int_{-1}^{1} \int_{-T}^{T} e^{it(u+x)}\,dt\,du = \int_{-1}^{1} \frac{e^{iT(u+x)}-e^{-iT(u+x)}}{i(u+x)}\,du = \\
  &= 2 \int_{-1}^{1} \frac{\sin(T(u+x))}{u+x}\,du = \begin{bmatrix}
   v = T(u+x), & dv = T\,du \\
   u = v / T - x, & du = dv / T
  \end{bmatrix} = \\
  &= 2 \int_{T(x-1)}^{T(x+1)} \frac{\sin v}{v / T}\,\frac{dv}{T} = 2 \int_{T(x-1)}^{T(x+1)}  \frac{\sin v}{v}\,dv.
 \end{align*}

 Исследуем теперь для числа $ x \in \R $ предел
 \begin{align*}
  \lim_{T \to +\infty}  \Phi_T(x) = \lim_{T \to +\infty} \int_{T(x-1)}^{T(x+1)}  \frac{2\sin v}{v}\,dv.
 \end{align*} 

 \begin{itemize}
  \item Если $x > 1$, то оба конца $ T(x - 1) $ и $ T(x+1) $ стремятся к $ +\infty $. Так как интеграл
   \begin{align*}
    \int_{1}^{+\infty} \frac{\sin v}{v}\,dv
   \end{align*} сходится, то его хвост может стремится только к нулю. Симметрично, если $ x < -1 $, то оба конца $ T(x-1) $ и $ T(x+1) $ стремятся к $ -\infty $, значит хвост интеграла стремится к нулю.
  \item Пусть $-1 < x < 1 $. Тогда
   \begin{align*}
    \lim_{T \to +\infty} \int_{T(x-1)}^{T(x+1)}  \frac{2 \sin v}{v}\,dv = \int_{-\infty}^{+\infty} \frac{2\sin v}{v}\,dv = 4 \int_{0}^{\infty} \frac{\sin v}{v}\,dv = 2 \pi.
   \end{align*}

  \item Пусть $ x = 1 $. Тогда
   \begin{align*}
    \lim_{T \to +\infty} \int_{T(x-1)}^{T(x+1)} \frac{2\sin v}{v}\,dv = \int_{0}^{\infty} \frac{2\sin v}{v} = \pi.
   \end{align*} Аналогично для $ x = -1 $:
   \begin{align*}
    \lim_{T \to +\infty} \int_{T(x-1)}^{T(x+1)} \frac{2\sin v}{v}\,dv = \int_{-\infty}^{0} \frac{2\sin v}{v}\,dv = \pi.
   \end{align*}
 \end{itemize}

 Таким образом,
 \begin{align*}
  \lim_{T \to +\infty} \Phi_T(x) = \begin{cases}    0, \text{ если } x < -1 \text{ или } x > 1  \\ 
   \pi, \text{ если } x = \pm 1 \\
  2\pi, \text{ если } -1 < x < 1   \end{cases}
  = 2\pi \cdot \Ind_{[-1,1]}(x) 
 \end{align*} $ P_\xi $-почти всюду, так как по условию $P_{\xi}(\{-1\}) = P_{\xi}(\{1\}) = 0$.

 Теперь, если мы обоснуем следующий предельный переход
 \begin{align}
  \label{eq:inversing_formula:limit_swap_int}
  \lim_{T \to +\infty} \int_{\R} \Phi_T(x)\,dP_\xi(x) = \int_{\R} \left(\lim_{T \to +\infty}   \Phi_T(x) \right)dP_\xi(x),
 \end{align} то мы получим то, что нужно:
 \begin{align*}
  \frac{1}{2\pi} \lim_{T \to +\infty} \int_{-T}^{T} \frac{e^{it}-e^{-it}}{it} \cdot \varphi_\xi(t)\,dt &= \frac{1}{2\pi} \lim_{T\to +\infty} \int_{\R} \Phi_T(x)\,dP_\xi(x) = \\
  &= \frac{1}{2\pi} \int_{\R} \left( \lim_{T \to +\infty} \Phi_T(x)  \right) dP_\xi(x) = \\
  &= \frac{1}{2\pi} \int_{\R} 2\pi \cdot \Ind_{[-1,1]}(x)\,dP_\xi(x) = P_\xi([-1,1]) = \\
  &= P(-1 \leqslant \xi \leqslant 1).
 \end{align*}

 Обосновывать \eqref{eq:inversing_formula:limit_swap_int} будем теоремой Лебега о мажорируемой сходимости. Так как мера $ P_\xi $ конечная, достаточно показать ограниченность $ \left| \Phi_T(x) \right| $. Эквивалентно, нужно показать ограниченность интегралов вида
 \begin{align*}
  \int_{a}^{b} \frac{\sin v}{v}\,dv, \qquad a,b\in\R.
 \end{align*} Так как интегралы  $ \int_{0}^{\infty} \frac{\sin v}{v}\,dv  $ и $ \int_{-\infty}^{0} \frac{\sin v}{v}\,dv  $ сходятся, то интегралы вида
 \begin{align*}
  \int_{0}^{w} \frac{\sin v}{v}\,dv, \qquad w \in \R
 \end{align*}  ограничены некоторой константой $ C $. Но тогда интегралы ограничены и по любому отрезку:
 \begin{align*}
  \left| \int_{a}^{b} \frac{\sin v}{v}\,dv \right| \leqslant \left| \int_{a}^{0} \frac{\sin v}{v}\,dv \right| + \left| \int_{0}^{b} \frac{\sin v}{v}\,dv \right| \leqslant 2C.
 \end{align*}

 Нам осталось свести общий случай к $ a = -1 $, $ b = 1 $. Рассмотрим \begin{align*}
  \xi = \frac{b-a}{2}\cdot\eta + \frac{a+b}{2}.
 \end{align*} Тогда $ a \leqslant \xi \leqslant b $ равносильно $ -1 \leqslant \eta \leqslant 1 $. Значит,
 \begin{align*}
 P(a \leqslant \xi \leqslant b) &= P(-1 \leqslant \eta \leqslant 1) = \lim_{T \to +\infty} \frac{1}{2\pi} \int_{-T}^{T} \frac{e^{it}-e^{-it}}{it}\cdot\varphi_\eta(t)\,dt. \end{align*} Нам осталось доказать равенство
 \begin{align}
  \label{eq:inversing_formula:a_b_to_pm_1}
  \lim_{T \to +\infty} \int_{-T}^{T} \frac{e^{-iat} - e^{-ibt}}{it} \cdot \varphi_\xi(t)\,dt = \lim_{T \to +\infty} \int_{-T}^{T} \frac{e^{it}-e^{-it}}{it} \cdot \varphi_\eta(t)\,dt.
 \end{align} Для этого раскроем $\varphi_{\xi}(t) $ по свойствам характеристической функции:
 \begin{align*}
  \varphi_\xi(t) = e^{it \frac{a+b}{2}}\varphi_\eta\left(\frac{b-a}{2} \cdot t\right)
 \end{align*}
 и выполним преобразования:	
 \begin{align*}
  \int_{-T}^{T} \frac{e^{-iat} - e^{ibt}}{it} \cdot \varphi_{\xi}(t)\,dt &= \int_{-T}^{T} \frac{e^{-iat}-e^{-ibt}}{it}e^{i t \frac{a+b}{2}} \cdot \varphi_\eta \left( \frac{b-a}{2} \cdot t \right)\,dt = \\
  &= \int_{-T}^{T} \frac{e^{it \frac{b-a}{2}} -e^{-i t \frac{b-a}{2}}}{it} \varphi_\eta \left( \frac{b-a}{2} \cdot t \right)dt = \\
  &= \begin{bmatrix}
   s =  \frac{b-a}{2} \cdot t, & ds = \frac{b-a}{2}\,dt \\
   t = \frac{2}{b-a} \cdot s, & dt = \frac{2}{b-a}\,ds
  \end{bmatrix} = \\
  &= \int_{- \frac{b-a}{2}T}^{\frac{b-a}{2}T} \frac{e^{is}-e^{-is}}{is}\cdot\varphi_\eta(s)\,ds.
 \end{align*} Так как $ \frac{b-a}{2} > 0 $, то устремив $ T \to +\infty $ мы получим равенство \eqref{eq:inversing_formula:a_b_to_pm_1}.
\end{proof}

\begin{crly}[характеристическая функция определяет распределение]
 \label{corollary:char_func_defines_distribution}
 Пусть $ \xi,\eta $ --- случайные величины. Если их характеристические функции совпадают: $ \varphi_\xi = \varphi_\eta $, то они одинаково распределены: $ P_\xi = P_\eta $.
\end{crly}
\begin{proof}[\normalfont\textsc{Доказательство}]
 Введём обозначение для множества точек разрыва функций распределения $ \xi $ и $ \eta $:
 \begin{align*}
  B = \left\{ x\in\R \mid P_\xi(\left\{ x \right\}) > 0 \text{ или } P_\eta(\left\{ x \right\}) > 0 \right\}.
 \end{align*} Множество $ B $ не более, чем счётно (например, так как $ \sum_{x \in B} P(\xi = x) + \sum_{x \in B} P(\eta = x) \leqslant 2 $).

 По формуле обращения (теорема \ref{theorem:inversing_formula}) для любой ячейки $ (a,b] $, концы которого не являются точками разрыва, верно
 \begin{align*}
  P_\xi \left(a, b\right] = P_\eta \left(a, b\right], \qquad a,b \notin B.
 \end{align*}

 По единственности из теоремы Каратеодори о стандартном продолжении равенство мер $ P_\xi = P_\eta $ достаточно проверить на ячейках:
 \begin{align*}
  P_\xi \left(a, b\right] = P_\eta \left(a, b\right], \qquad a,b \in \R.
 \end{align*} Поэтому, нам осталось лишь разобраться с точками разрыва.

 Докажем равенство мер для ячейки $ \left(a, b\right]   $, где $ a \in B $ и  $ b \notin B $. Рассмотрим точки $ a_n \notin B $, такие, что $ a_n > a $ и $ a_n \to a $. Так как
 \begin{align*}
  \left(a, b\right] = \bigcup_{n=1}^{\infty} \left(a_n, b\right],
 \end{align*} то по непрерывности меры сверху
 \begin{align*}
  P_\xi \left(a, b\right] = \lim_{n \to \infty} P_\xi \left(a_n, b\right]    = \lim_{n \to \infty} P_\eta \left(a_n, b\right] = P_\eta \left(a, b\right].  
 \end{align*} Равенство мер доказано на ячейках с $ a \in \R $ и  $ b \notin B $.

 Аналогично докажем равенство мер для ячейки  $ \left(a, b\right]   $, где $ a \in \R $ и  $ b \in B $. Возьмём точки  $ b_n \notin B $ такие, что  $ b_n > b $ и  $ b_n \to b $ . Так как
 \begin{align*}
  \left(a, b\right] = \bigcap_{n=1}^{\infty}\left(a, b_n\right],
 \end{align*} то по непрерывности меры снизу
 \begin{align*}
  P_\xi \left(a, b\right]  = \lim_{n \to \infty} P_\xi \left(a, b_n\right]    = \lim_{n \to \infty} P_\eta \left(a, b_n\right] = P_\eta \left(a, b\right].  
 \end{align*}
\end{proof}

\begin{crly}
 Пусть $ \xi $ --- случайная величина с суммируемой по мере Лебега функцией распределения:
 \begin{align*}
  \int_{\R} \left| \varphi_\xi(t) \right|dt < +\infty. 
 \end{align*}
 Тогда $ \xi $ абсолютно непрерывна, и её плотность равна
 \begin{align}
  \label{eq:density_of_rv_with_sum_char_func}
  p_\xi(x) = \frac{1}{2\pi} \int_{\R} e^{-itx}\cdot\varphi_\xi(t)\,dt. 
 \end{align}
\end{crly}
Выражение \eqref{eq:density_of_rv_with_sum_char_func} называют \textit{преобразованием Фурье} функции $ \varphi_\xi(t) $.
\begin{proof}[\normalfont\textsc{Доказательство}]
 Пусть $ a < b $ --- точки непрерывности функции распределения $ \xi $. Из абсолютной сходимости интеграла $ \int_{\R} \left| \varphi_\xi(t) \right|dt  $ следует абсолютная сходимость интеграла
 \begin{align*}
  \int_{\R} \left|\frac{e^{-iat}-e^{-ibt}}{it} \cdot \varphi_\xi(t) \right|dt,
 \end{align*} поскольку добавленный множитель ограничен:
 \begin{align*}
  &\left| \frac{e^{-iat} - e^{-ibt}}{it} \right| \leqslant \frac{2}{\left| t \right|} < 2 &\text{при } \left| t \right| > 1,\\
  &\left| \frac{e^{-iat} - e^{-ibt}}{it} \right| = \left| \frac{1 + \OO(t) - 1 + \OO(t)}{t} \right| = \OO(1) &\text{при } t \in [-1, 1].
 \end{align*} Поэтому, по теореме \ref{theorem:inversing_formula} формула обращения принимает следующий вид:
 \begin{align*}
  P(a \leqslant \xi \leqslant b) = \frac{1}{2\pi} \int_{\R} \frac{e^{-iat}-e^{-ibt}}{it}\cdot\varphi_\xi(t)\,dt. 
 \end{align*}

 Нам нужно проверить равенство
 \begin{align*}
  P(a \leqslant \xi \leqslant b) = \int_{a}^{b} p_\xi(x)\,dx
 \end{align*} где $ p_\xi(x) $ взята из условия теоремы (равенство \eqref{eq:density_of_rv_with_sum_char_func}). Проверим:
 \begin{align*}
  \int_{a}^{b} \frac{1}{2\pi} \int_{\R} e^{-itx}\cdot\varphi_\xi(t)\,dt\,dx &= \frac{1}{2\pi}\int_{\R}  \left(\int_{a}^{b} e^{-itx}\,dx \right) \cdot \varphi_\xi(t)\,dt = \\
  &= \frac{1}{2\pi} \int_{\R} \frac{e^{-iat}-e^{-ibt}}{it} \cdot \varphi_\xi(t)\,dt = \\
  &= P(a \leqslant \xi \leqslant b).
 \end{align*} Интегралы были переставлены по теореме Фубини: функция $ e^{-itx} \cdot \varphi_\xi(t) $ суммируема при $ t \in \R $ и $ x \in [a,b] $, так как $ \varphi_\xi(t) $ суммируема по условию, а $ e^{-itx} $ ограничена единицей.

 Отметим, что из уже доказанного следует отсутствие точек разрыва функции распределения (поэтому ограничение на $ a $ и $ b $ не имеет силы). Действительно, мы можем покрыть точку разрыва малым отрезком $ [a,b] $, и интеграл плотности по нему будет стремится к нулю, то есть вероятность этой точки разрыва на самом деле равна нулю.
\end{proof}

\begin{thm}[сумма независимых нормальных случайных величин]
 \label{theorem:sum_of_independent_normal_rvs}
 Пусть нормальные случайные величины $ \xi_k \sim \Norm(\mu_k, \sigma_k^{2}) $ независимы. Пусть среди коэффициентов $ c_1, c_2, \ldots, c_n $ есть хотя бы один ненулевой, и пусть $ \mu_0 \in \R $ --- число. Тогда следующая сумма случайных величин распределена нормально:
 \begin{align*}
  \xi = \mu_0 + c_1 \xi_1 + \ldots + c_n \xi_n \sim \Norm(\mu, \sigma^{2}),
 \end{align*} где
 \begin{align*}
  \mu = \mu_0 + \sum_{k=1}^{n}c_k\mu_k, &&
  \sigma^{2} = \sum_{k=1}^{n}c_k^{2}\sigma_k^{2}.
 \end{align*}
\end{thm}
\begin{proof}[\normalfont\textsc{Доказательство}]
 Вычислим характеристическую функцию $ \xi $:
 \begin{align*}
  \varphi_\xi(t) &= \varphi_{\mu_0 + c_1\xi_1 + \ldots + c_n\xi_n}(t) = e^{i\mu_0 t} \prod_{k=1}^{n}\varphi_{\xi_k}(c_k t) = e^{i\mu_0 t} \prod_{k=1}^{n} e^{i\mu_k (c_k t) - \frac{\sigma_k^{2}(c_k t)^{2}}{2}} = \\
  &= \exp \left[ i \left( \mu_0 + \sum_{k=1}^{n}c_k\mu_k \right) t - \frac{t^{2}}{2} \left( \sum_{k=1}^{n}c_k^{2}\sigma_k^{2} \right) \right] = \\
  &= e^{i\mu t - \frac{\sigma^{2}t^{2}}{2}}.
 \end{align*} Таким образом, характеристическая функция $ \xi $ совпадает с характеристической функцией нормального распределения $ \Norm(\mu,\sigma^{2}) $. По следствию \ref{corollary:char_func_defines_distribution} $ \xi \sim \Norm(\mu,\sigma^{2}) $.
\end{proof}

Доказательство теоремы \ref{theorem:sum_of_independent_normal_rvs} с помощью характеристических функций оказалось крайне простым и красивым: вычислили характеристическую функцию, проверили. Если же доказывать её, вычисляя свёртку плотностей (как мы это сделали в примере \ref{example:sum_of_two_independent_normal_rvs}), то мы получим огромную выкладку.

\end{document}

