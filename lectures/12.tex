% 2023.05.16 lecture 12
\documentclass[../main.tex]{subfiles}
\begin{document}

\newpage
\section{Цепи Маркова.}

\subsection{Однородные марковские цепи.}

Второй дискретный случайный процесс, который мы изучим --- это \textit{марковская цепь}.

\begin{df}[цепь Маркова]
 Пусть $ Y $ --- не более, чем счётное множество, которые мы будем называть \textit{фазовым пространством}. Мы будем часто считать, что $ Y \subset \R $ (или даже $ Y \subset \N $), при этом рассматривая $ Y $ как измеримое пространство $ (Y,2^{Y}) $, тогда функции $ \xi\colon\,\Omega\to Y $ можно называть случайными величинами. Элементы $ Y $ будем называть \textit{состояниями}.

 Пусть $ (\Omega,\F,P) $ --- вероятностное пространство. Последовательность случайных величин $ \{\xi_n \colon\,\Omega\to Y\}_{n = 0}^{\infty} $ называется \textit{цепью Маркова}, если:
 \begin{align*}
  P(\xi_n = a_n \mid \xi_{n-1} = a_{n-1},\, \ldots,\, \xi_0 = a_0) = P(\xi_n = a_n \mid \xi_{n-1} = a_{n-1})
 \end{align*} для любых состояний $ a_0, a_1, \ldots, a_n \in Y $ таких, что $ P(\xi_{n-1} = a_{n-1},\, \ldots,\, \xi_0 = a_0) > 0 $.

 Говоря проще, случайная величина $ \xi_n $ не зависит от $\xi_0, \xi_1, \ldots, \xi_{n-2} $, но может зависеть от $ \xi_{n-1} $.
\end{df}

\begin{exmpl}[случайное блуждание по $ \Z $]
 \label{exmpl:random_walk_z}
 Пусть фазовое пространство $ Y = \Z $. Пусть случайные величины $ \eta_1, \eta_2, \ldots $ независимы, и распределены следующим образом:
 \begin{align*}
  \eta_k = \begin{cases}
   1 &\text{ с вероятностью } p, \\
   -1 &\text{ с вероятностью } 1-p.
  \end{cases} 
 \end{align*} Пусть
 \begin{align*}
  \xi_n := \eta_1 + \eta_2 + \ldots + \eta_n.
 \end{align*} Тогда $\{\xi_{n}\}_{n=0}^{\infty} $ --- цепь Маркова. Действительно, $ \xi_n $ зависит лишь от $ \xi_{n-1} $ и от $ \eta_n $:
 \begin{align*}
  \xi_n = \eta_n + \xi_{n-1}.
 \end{align*} Эта цепь Маркова называется \textit{случайным блужданием по $ \Z $}.
\end{exmpl}

\begin{exmpl}
 \label{exmpl:random_walk_machine}
 Пусть есть прибор, у которого бывают два состояния: <<работает>> и <<не работает>>. Каждый день может случиться следующее. Если прибор работает, то с вероятностью $ p $ он сломаться, а с вероятностью $ 1-p $ он продолжит работать. Если прибор не работает, то с вероятностью $ q $ его могут починить, а с вероятностью $ 1-q $ он останется сломанным.

 Система выше может быть описана следующей цепью Маркова: фазовое пространство $ Y = \left\{ 0,1 \right\} $ ($ 1 $, если прибор работает, и $ 0 $ --- если нет), $ \xi_0 = 1 $, и
 \begin{align*}
  \xi_{n+1} = \begin{cases}
   0, \text{ с вероятностью $ p $, если $ \xi_n = 1 $},  \\
   1, \text{ с вероятностью $ 1-p $, если $ \xi_n = 1 $},  \\
   1, \text{ с вероятностью $ q $, если $ \xi_n = 0 $},  \\
   0, \text{ с вероятностью $ 1-q $, если $ \xi_n = 0 $}.
  \end{cases} 
 \end{align*}
\end{exmpl}

\begin{remrk}
 Цепь Маркова однозначно определяется начальным распределением $ \pi_0 := P_{\xi_0} $ и \textit{функциями перехода}
 \begin{align*}
  p_n(a,b) := P(\xi_n = b \mid \xi_{n-1} = a), \quad n \geqslant 1.
 \end{align*}
\end{remrk}

\begin{df}
 Цепь Маркова называется \textit{однородной}, если функции перехода не зависят от $ n $:
 \begin{align*}
  p_n(a, b) = p_{ab}
 \end{align*} для любого $ n \geqslant 1 $ и любых состояний $ a,b\in Y $.
\end{df}

Далее мы будем изучать лишь однородные цепи Маркова. В них мы будем говорить не о функциях перехода, а об одной \textit{функции перехода} $ p \colon\, Y \times Y \to [0,1] $.

\begin{exmpl*}
 В случайном блуждании по $ \Z $ (примере \ref{exmpl:random_walk_z}) функция перехода имеет вид:
 \begin{align*}
  \begin{cases}
   p_{ab} = 0, \text{ если } \left| a-b \right| \neq 1, \\
   p_{a,a+1} = p, \\
   p_{a,a-1} = 1-p.
  \end{cases} 
 \end{align*}

 В примере \ref{exmpl:random_walk_machine}:
 \begin{align*}
  p_{11} = 1-p, && p_{10} = p, && p_{01} = q, && p_{00} = 1-q.
 \end{align*}
\end{exmpl*}

\begin{df}
 Последовательность
 \begin{align*}
  \xi_0 = a_0,\; \xi_1 = a_1,\; \ldots,\; \xi_n = a_n
 \end{align*}
 называется \textit{траекторией} цепи Маркова.
\end{df}

\begin{thm}
 Вероятность траектории цепи Маркова равна
 \begin{align*}
  P(\xi_0 = a_0,\; \xi_1 = a_1,\; \ldots,\; \xi_n = a_n) = \pi_0(a_0) \cdot  p_{a_0 a_1} \cdot p_{a_1 a_2} \cdot  \ldots \cdot p_{a_{n-1}a_n}.
 \end{align*}
 \begin{proof}[\normalfont\textsc{Доказательство}]
  По индукции. База очевидна (определение $ \pi_0 $). Переход $ n-1 \to n $ тоже очевиден:
  \begin{align*}
   &P(\xi_0 = a_0,\; \ldots,\; \xi_n = a_n) =\\
   =\;&P(\xi_n = a_n \mid \xi_{n-1} = a_{n-1},\;\ldots,\;\xi_0 = a_0)\cdot P(\xi_{n-1} = a_{n-1},\;\ldots,\;\xi_0=a_0) = \\
   =\;&P(\xi_n = a_n \mid \xi_{n-1} = a_{n-1}) \cdot P(\xi_{n-1} = a_{n-1},\; \ldots,\; \xi_0 = a_0) = \\
   =\;&p_{a_{n-1} a_n} \cdot \pi_0(a_0) \cdot p_{a_0 a_1} \cdot p_{a_1 a_2} \cdot \ldots \cdot p_{a_{n-2} a_{n-1}}.
  \end{align*}
 \end{proof}
\end{thm}

\begin{df}
 Будем говорить, что $ \pi \colon\, Y \to [0,1] $ --- \textit{распределение} на $ Y $, если
 \begin{align*}
  \sum_{y \in Y} \pi(y) = 1.
 \end{align*}
\end{df}

\begin{thm}[о существовании цепи Маркова]
 \label{theorem:markov_chain_exists}
 Пусть $ Y $ --- фазовое пространство, $ \pi_0 \colon\, Y \to [0,1] $ --- распределение на $ Y $, и функция $ p \colon\, Y \times Y \to [0,1] $ такова, что для каждого состояния $ a \in Y $ $ p_a \colon\, Y\to[0,1] $ --- распределение на $ Y $.

 Тогда существуют вероятностное пространство $ (\Omega,\F,P) $ и однородная цепь Маркова $ \left\{\xi_n\colon\,\Omega\to Y \right\}_{n=0}^{\infty} $ с начальным распределением $ \pi_0 $ и функцией перехода $ p $.
\end{thm}

Теорема \ref{theorem:markov_chain_exists} чисто техническая, как и её доказательство, поэтому приводить его мы не будем.

\subsection{Эргодическая теорема Маркова.}

\begin{conventn}
 \label{convention:linalg_in_markov_chain}
 В цепях Маркова бывает удобно использовать сущности линейной алгебры (вектора и матрицы). Для этого мы полагаем фазовое пространство равным $ Y = [n] $ или $ Y = \N $. Тогда распределения $ \pi_n = P_{\xi_n} $ мы будем рассматривать как вектора-строки $ \pi_n \in \R^{1 \times Y} $ (возможно счётной длины, если $ Y = \N $), а функцию перехода $ p $ мы будем рассматривать как матрицу $ P \in \R^{Y \times Y} $
 \begin{align*}
  P = \left( p_{ab} \right)_{(a,b)\in Y \times Y}.
 \end{align*} Матрица $ P $ также может быть счётной ($ \N \times \N $).

 В теории вероятностей принято использовать вектора-строки вместо векторов-столбцов, к этому нужно привыкнуть.
\end{conventn}

\begin{thm}
 В обозначениях из соглашения \ref{convention:linalg_in_markov_chain} верно
 \begin{align*}
  \pi_n = \pi_0 P^{n}.
 \end{align*}
\end{thm}
\begin{proof}
 Легко проверяется по индукции. База очевидна. Переход:
 \begin{align*}
  \pi_n(b) &= \sum_{a \in Y} \pi_{n-1}(a) \cdot p_{ab} \implies \pi_n = \pi_{n-1} P.
 \end{align*}
\end{proof}

\begin{df}[стационарное распределение]
 Распределение $ \pi \colon\, Y \to [0,1] $ на $ Y $ называется \textit{стационарным}, если $ \pi P = \pi $.

 Иными словами, стационарное распределение --- это неподвижная точка линейного оператора $ P $.
\end{df}

\begin{exmpl}
 Стационарное распределение может не существовать.

 Рассмотрим случайное блуждание на $ \Z $ с вероятностью $ p = 1 / 2 $. Предположим, что есть стационарное распределение $ \pi \colon\, \Z \to [0,1] $. Тогда для любой точки $ y \in \Z $ верно
 \begin{align*}
  \pi(y) = \frac{1}{2}\pi(y-1) + \frac{1}{2}\pi(y+1) \implies \pi(y) - \pi(y-1) = \pi(y+1) - \pi(y).
 \end{align*} То есть, величина $ \alpha(y) = \pi(y+1) - \pi(y) $  --- это константа $ \alpha \in \R $. Если $ \alpha > 0 $, то $ \pi(n) = \pi(0) + \alpha n \to \infty $, что невозможно. Аналогично получаем противоречие при $ \alpha < 0 $. Если же $ \alpha = 0 $, то $ \pi(y) $ --- константа, что тоже невозможно. Значит, у случайного блуждания по $ \Z $ не может быть стационарного распределения.
\end{exmpl}

\textit{Эргодическая теорема Маркова} утверждает, что при достаточно слабых условиях на цепь Маркова, стационарное распределение существует, единственно, и может быть приближено на практике очень простым итеративным способом.

\begin{thm}[эргодическая теорема Маркова]
 \label{theorem:ergodic_theorem_markov}
 Пусть есть конечная (то есть с конечным фазовым пространством $ Y $) однородная цепь Маркова, функция перехода которой не обнуляется: $ p_{ab} > 0 $ для всех состояний $ a,b\in Y $.

 Тогда у этой цепи Маркова существует единственное стационарное распределение $ \pi\colon\,Y\to[0,1] $. При этом, для произвольного начального состояния $ a \in Y $ выполнено предельное соотношение
 \begin{align}
  \label{eq:ergodic_theorem_markov:lim}
  \pi(b) = \lim_{n \to \infty} P(\xi_n = b \mid \xi_0 = a)
 \end{align} при всех $ b \in Y $. Более того, сходимость предела \eqref{eq:ergodic_theorem_markov:lim} экспоненциально быстрая: существуют константы $ q < 1 $ и $ C> 0 $, зависящие лишь от цепи Маркова, такие, что
 \begin{align}
  \label{eq:ergodic_theorem_markov:convergence_bound}
  \left| \pi(b) - P(\xi_n = b \mid \xi_0 = a) \right| \leqslant C q^{n}
 \end{align} при всех $ a,b\in Y $.
\end{thm}

Удивительная особенность эргодической теоремы Маркова состоит в том, что совершенно не важно, из какого начального состояния $ a \in Y $ мы стартуем: итерируя переходы в цепи Маркова множество раз мы в всё равно в пределе получим единственное стационарное распределение $ \pi $. Кроме того, это теорема предоставляет мощную оценку на скорость сходимости \eqref{eq:ergodic_theorem_markov:convergence_bound}, которая позволяет теоретически обосновать приближение стационарного распределения вышеуказанным итеративным методом.

Эргодическая теорема Маркова имеет множество приложений в биологии и прочих прикладных областях.

Уже из формулировки теоремы \ref{theorem:ergodic_theorem_markov} понятно, что в доказательстве мы будем применять теорему о сжимающем отображении, которая была изучена в первом семестре математического анализа --- ведь в этом результате мы говорим о существовании и единственности неподвижной точки, и о возможности приближения этой неподвижной точки методом простых итераций.

\begin{proof}[\normalfont\textsc{Доказательство теоремы \ref{theorem:ergodic_theorem_markov}}]
 Обозначим через $n = \# Y $ количество состояний в фазовом пространстве. Рассмотрим полное метрическое пространство $ \R^{n} $ с $ L^{1} $-нормой (также известной как \textit{манхэттенской} нормой):
 \begin{align*}
  \left\| x \right\| = \left| x_1 \right| + \left| x_2 \right| + \ldots + \left| x_n \right|.
 \end{align*} Рассмотрим следующее подмножество $ \R^{n} $:
 \begin{align*}
  S = \left\{ x \in \R^{n} \Mid \sum_{j=1}^{n}x_j = 1,\; x_j \geqslant 0 \right\}
 \end{align*} --- это в точности множество всевозможных распределений $ \pi \colon\,Y \to [0,1] $, записанных как вектора-строки из $ \R^{n} $. Заметим, что $ S $ тоже является полным пространством как замкнутое подмножество полного пространства $ \R^{n} $.

 Рассмотрим матрицу функции переходов $ P = \left( p_{ab} \right)_{(a,b) \in Y \times Y} $ как линейный оператор $ P \colon\,S \to S $:
 \begin{align*}
  P(x) = x P,
 \end{align*} где $ x \in \R^{n} $ --- вектор-строка. (Хоть $ P $ и линейный оператор, мы будем писать $ P(x) $ со скобками, дабы прояснить, что мы не умножаем на матрицу слева.) Ясно, что оператор $ P $ действительно переводит $ S $ в $ S $, так как функция перехода переводит распределение на $ Y $ в распределение на $ Y $.

 Оказывается, оператор функции перехода $ P $ является сжимающим отображением на полном пространстве $ S $ распределений на $ Y $. Докажем это. Обозначим
 \begin{align*}
  \delta := \min \left\{ p_{ab} \mid a,b\in Y \right\} \in (0, 1 / n].
 \end{align*} Рассмотрим произвольные точки $ x,y \in S $ и рассмотрим их разность $z=x-y \in \R^{n}$, $ z=(z_1,\ldots,z_n) $. Так как суммы координат $ x $ и $ y $ равны $ 1 $, то
 \begin{align*}
  z_1 + \ldots + z_n = 0.
 \end{align*} Оценим расстояние между образами точек $ x $ и $ y $:
 \begin{align*}
  \left\| P(x)-P(y) \right\| &= \left\| P(z) \right\| = \sum_{k=1}^{n} \left| \sum_{j=1}^{n}z_jp_{jk} \right| = \sum_{k=1}^{n} \left| \sum_{j=1}^{n}z_j(p_{jk} - \delta) + \delta \sum_{j=1}^{n}z_j \right| = \\
  &= \sum_{k=1}^{n} \left| \sum_{j=1}^{n} z_j (p_{jk}-\delta) \right| \leqslant \sum_{k=1}^{n}\sum_{j=1}^{n} \left| z_j \right|(p_{jk}-\delta) = \\
  &= \sum_{j=1}^{n}\left| z_j \right| \sum_{k=1}^{n} (p_{jk} - \delta) = \sum_{j=1}^{n} \left| z_j \right| \cdot (1 - n\delta) = (1 - n\delta) \cdot \left\| z \right\| = \\
  &= (1-n\delta)\cdot \left\| x-y \right\|.
 \end{align*}

 Таким образом, $ P \colon\,S \to S $ --- сжимающее отображение на полном пространстве $ S $ с коэффициентом сжатия $ q := 1 - n \delta \in [0, 1) $. По теореме о сжимающем отображении для любой начальной точки $ x_0 \in S $ последовательность
 \begin{align*}
  x_0,\; P(x_0),\;P(P(x_0)),\;P(P(P(x_0))),\;\ldots
 \end{align*} стремится к единственной неподвижной точке $ x^{\ast} \in S $, $ P(x^{\ast}) = x^{\ast} $. При этом, скорость сходимости удовлетворяет оценке \eqref{eq:ergodic_theorem_markov:convergence_bound}.
\end{proof}

В эргодической теореме Маркова можно ослабить условие на марковскую цепь: достаточно того, чтобы функция перехода не обнулялась лишь при возведении в степень $ m $ для некоторого $ m \geqslant 1 $.

\begin{crly}
 Пусть марковская цепь конечна, и существует число $ m \geqslant 1$ такое, что
 \begin{align*}
  p_{ab}(m) := P(\xi_m = b \mid \xi_0 = a) > 0
 \end{align*} для всех $ a,b \in Y $. Тогда верно то же заключение теоремы \ref{theorem:ergodic_theorem_markov}: существует единственное стационарное распределение $ \pi \colon\,Y \to [0,1] $, и верны \eqref{eq:ergodic_theorem_markov:lim} и \eqref{eq:ergodic_theorem_markov:convergence_bound}.
\end{crly}
\begin{proof}[\normalfont\textsc{Доказательство}]
 Рассмотрим цепь Маркова на том же вероятностном пространстве и фазовом пространстве $ Y $, которая за один шаг применяет сразу $ m $ шагов исходной цепи. То есть,
 \begin{align*}
  \hat p_{ab} := p_{ab}(m)
 \end{align*} для всех $ a,b \in Y$, где $ \hat p $ --- функция перехода новой цепи Маркова. Так как теперь её функция перехода не обнуляется, то по эргодической теореме Маркова \ref{theorem:ergodic_theorem_markov} существует единственное стационарное распределение $ \pi \colon\,Y \to [0,1] $: $ \hat p(\pi) = \pi $.

 Покажем, что $ \pi $ также является стационарным распределением для исходной цепи. Предположим, что $ p(\pi) \neq \pi $. Тогда повторные применения функции перехода $ p $ к распределению $ \pi $ образуют цикл длины не более $ m $:
 \begin{align*}
  \pi = \pi_0 \xmapsto{p} \pi_1 \xmapsto{p} \ldots \xmapsto{p} \pi_{k-1} \xmapsto p \pi_k = \pi, \quad k \leqslant m.
 \end{align*} Но тогда любое другое распределение $ \pi_j $ из цикла также будет неподвижной точкой функции перехода $ \hat p $, а этого не может быть по единственности.

 Оценка на скорость сходимости \eqref{eq:ergodic_theorem_markov:convergence_bound} верна отдельно для каждой из $ m $ подпоследовательностей с шагом $ m $:
 \begin{align*}
  \left| \pi(b) - P(\xi_{jm + r} = b \mid \xi_0 = a) \right| \leqslant C q^{j}.
 \end{align*} Тогда их можно соединить в одну оценку для всей последовательности:
 \begin{align*}
  \left| \pi(b) - P(\xi_n = b \mid \xi_0 = a) \right| \leqslant C q^{n / m} = C \left( q^{1 / m} \right)^{n}.
 \end{align*} Стремление \eqref{eq:ergodic_theorem_markov:lim} непосредственно следует из \eqref{eq:ergodic_theorem_markov:convergence_bound}.
\end{proof}

\subsection{Критерий возвратности. Теорема солидарности.}

\begin{notatn*}
 Обозначим
 \begin{align*}
  p_{ab}(m) = P(\xi_m = b \mid \xi_0 = a)
 \end{align*} --- вероятность перейти в состояние $ b \in Y $ за $ m $ шагов, начав в состоянии $ a \in Y$. В частности,
 \begin{align*}
  p_{ab}(0) = \Ind_{a = b}.
 \end{align*}
\end{notatn*}

\begin{df}
 Скажем, что состояние $ b \in Y $ \textit{достижимо} из состояния $ a \in Y $, если $ p_{ab}(m) > 0 $ для некоторого $ m \geqslant 0 $.

 Достижимость эквивалентна наличию траектории
 \begin{align*}
  a = a_0 \to a_1 \to \ldots \to a_{m-1} \to a_m = b,
 \end{align*} для которой $ p_{a_0 a_1} > 0,\; \ldots,\; p_{a_{m-1} a_m} > 0 $, то есть достижимости в ориентированном графе, состоящем из переходов марковской цепи с ненулевой вероятностью. Указанный граф далее будем называть \textit{графом переходов}.
\end{df}

\begin{df}
 Скажем, что состояния $ a \in Y $ и $ b \in Y $ \textit{сообщающиеся}, если как $ a $ достижимо из $ b $, так и $ b $ достижимо из $ a $.

 Это определение эквивалентно определению сильной связности в графе переходов.
\end{df}

\begin{df}
 Скажем, что состояние $ a \in Y $ \textit{существенное}, если для любого состояния $ b \in Y $, достижимого из $ a $, $ a $ и $ b $ сообщающиеся.
\end{df}

\begin{remrk*}
 В конечной марковской цепи всегда есть хотя бы одно существенное состояние: в конденсации конечного графа переходов есть хотя бы одна вершина-сток, и все состояния из этой компоненты сильной связности являются существенными.

 В бесконечной марковской цепи это неверно. Можно рассмотреть блуждание по $ \N $, где мы всегда идём вперёд.
\end{remrk*}

\begin{notatn*}
 Для состояния $ a \in Y $ и числа $ n \geqslant 1 $ обозначим
 \begin{align*}
  f_a(n) := P(\xi_n=a\mid \xi_{n-1} \neq a,\; \ldots,\; \xi_1 \neq a,\; \xi_0 = a)
 \end{align*} --- вероятность впервые вернуться в $ a $ за $ n $ шагов, если начать в $ a $.
 % Для $ n = 0 $ положим
 % \begin{align*}
 %  f_a(0) = 0.
 % \end{align*}
\end{notatn*}

\begin{notatn*}
 Для состояния $ a\in Y $ обозначим
 \begin{align*}
  F_a := \sum_{n=1}^{\infty} f_a(n)
 \end{align*} --- вероятность вернуться в состояние $ a $, начав из него же.
\end{notatn*}

\begin{df}
 Будем называть состояние $ a \in Y $ \textit{возвратным}, если вероятность вернуться в него равна единице: $ F_a = 1 $. В противном случае ($ F_a < 1 $) будем называть состояние $ a $  \textit{невозвратным}.
\end{df}

\begin{df}
 Будем говорить, что состояние $ a \in Y $ \textit{нулевое}, если
 \begin{align*}
  \lim_{n \to \infty} p_{aa}(n)  = 0.
 \end{align*}
\end{df}

\begin{thm}[критерий возвратности]
 \label{theorem:back_criteria}
 Состояние $ a \in Y $  возвратно тогда и только тогда, когда следующий ряд расходится
 \begin{align*}
  \sum_{n=0}^{\infty} p_{aa}(n) = +\infty.
 \end{align*} Более того, если $ a $  невозвратно, то
 \begin{align}
  \label{eq:back_criteria:F_a}
  F_a = \frac{P_a - 1}{P_a},
 \end{align} где
 \begin{align*}
  P_a = \sum_{n=0}^{\infty} p_{aa}(n).
 \end{align*}
\end{thm}
\begin{proof}[\normalfont\textsc{Доказательство}]
 Рассмотрим следующие формальные степенные ряды:
 \begin{align*}
  \mathcal P(z) = \sum_{n=0}^{\infty} p_{aa}(n) \cdot z^{n}, && \F(z) = \sum_{n=1}^{\infty}f_a(n) \cdot z^{n}.
 \end{align*} Напомним при этом, что $ p_{aa}(0) = 1 $.

 Заметим следующую связь между этими рядами:
 \begin{align*}
  p_{aa}(n) = \sum_{k=1}^{n} f_a(k) \cdot p_{aa}(n-k), \quad n \geqslant 1.
 \end{align*} Действительно, событие <<прийти из $ a $ в $ a $ за $ n $ шагов>> состоит из события <<впервые вернуться назад в $ a $ за $ k \geqslant 1 $ шагов>> и события <<прийти из $ a $ в $ a $ за $ n-k $ шагов>>. Следовательно,
 \begin{align*}
  \mathcal P(z) = \F(z) \cdot \mathcal P(z) + 1,
 \end{align*} или же
 \begin{align*}
  \mathcal F(z) = \frac{\mathcal P(z)-1}{\mathcal P(z)},
 \end{align*} где пока-что имеется в виду равенство между формальными степенными рядами. Но так как оба ряда $ \mathcal P(z) $ и $ \F(z) $ сходятся при $ \left| z \right| < 1 $ (коэффициенты ограничены единицей, и при $ \left| z \right| < 1 $ ряды ограничиваются сверху суммой геометрической прогрессии), то
 \begin{align*}
  \lim_{z \to 1-}  \mathcal F(z) = \lim_{z \to 1-} \frac{\mathcal P(z) -1}{\mathcal P(z)}.
 \end{align*} Так как ряд $ \F(z) $ сходится и при  $ z = 1 $, то левая часть просто равна  $ \F(1) $, то есть  $ F_a $:
 \begin{align*}
  F_a = \lim_{z \to 1-} \frac{\mathcal P(z) - 1}{\mathcal P(z)}.
 \end{align*}

 Если ряд $ \mathcal P(1) $ сходится, то мы получаем
 \begin{align*}
  F_a = \frac{\mathcal P(1) - 1}{\mathcal P(1)} = \frac{P_a - 1}{P_a},
 \end{align*} а если он расходится, то
 \begin{align*}
  F_a = \lim_{z \to 1-} \frac{1 - 1 / \mathcal(P(z))}{1} = 1.
 \end{align*}
\end{proof}

\begin{crly}
 Всякое невозвратное состояние является нулевым.
\end{crly}
\begin{proof}[\normalfont\textsc{Доказательство}]
 Если состояние $ a \in Y $ невозвратно, то по теореме \ref{theorem:back_criteria} ряд сходится:
 \begin{align*}
  \sum_{n=0}^{\infty} p_{aa}(n) < +\infty.
 \end{align*} Поэтому, его члены стремятся к нулю: $ p_{aa}(n) \to 0 $, то есть состояние $ a $ нулевое.
\end{proof}

\begin{thm}[теорема солидарности]
 Пусть $ a, b \in Y $ --- пара сообщающихся состояний. Тогда состояния $ a $ и $ b $ одновременно либо возвратные, либо невозвратные, а также они одновременно либо нулевые, либо не нулевые.
\end{thm}
\begin{proof}[\normalfont\textsc{Доказательство}]
 Пусть состояния $ a, b \in Y $ сообщающиеся: для некоторых $ i,j \geqslant 0 $ выполнено
 \begin{align*}
  p_{ab}(i) > 0, && p_{ba}(j) > 0.
 \end{align*} Тогда для всякого $ n \geqslant 0 $ выполнено неравенство
 \begin{align}
  \label{eq:theorem:solidarity:ineq}
  p_{aa}(n+i+j) \geqslant p_{ab}(i) \cdot p_{bb}(n) \cdot p_{ba}(j).
 \end{align} Если состояние $ a $ нулевое, то $ p_{aa}(n+i+j) \to 0 $, и тогда по неравенству \eqref{eq:theorem:solidarity:ineq} $ p_{bb}(n) \to 0 $, то есть $ b $ также нулевое.

 Аналогично для возвратности. Из неравенства \eqref{eq:theorem:solidarity:ineq} следует
 \begin{align}
  \label{eq:theorem:solidarity:ineq_series}
  \sum_{n=0}^{\infty} p_{aa}(n+i+j) \geqslant p_{ab}(i) \cdot p_{ba}(j) \cdot \sum_{n=0}^{\infty}p_{bb}(n).
 \end{align} Если $ a $ невозвратное, то по критерию возвратности (теорема \ref{theorem:back_criteria}) ряд
 \begin{align*}
  \sum_{n=0}^{\infty} p_{aa}(n+i+j)
 \end{align*} сходится, и по неравенству \eqref{eq:theorem:solidarity:ineq_series} сходится и ряд
 \begin{align*}
  \sum_{n=0}^{\infty}p_{bb}(n),
 \end{align*} что означает невозвратность $ b $ по критерию возвратности.
\end{proof}

\end{document}

