% 2023.03.21: lecture 06
\documentclass[../main.tex]{subfiles}
\begin{document}

\begin{remrk}[геометрический смысл ковариации]
 \label{remark:cov_geometric_intuition}
 У ковариации есть очень интересный геометрический смысл. 

 Рассмотрим пространство \textit{несмещённых} случайных величин $ \xi $ с конечным вторым моментом $ \E\xi^{2} < \infty $ и нулевым матожиданием:
 \begin{align*}
  L^{2}_0 = \left\{ \xi \mid \E\xi = 0,\; \E\xi^{2} < \infty \right\}.
 \end{align*}

 Ковариация двух таких величин $ \xi,\eta\in L^{2}_0 $ будет иметь вид
 \begin{align*}
  \cov(\xi,\eta) = \E(\xi\eta).
 \end{align*}

 Оказывается, что пространство $ L^{2}_0 $ является гильбертовым пространством с ковариацией в качестве скалярного произведения. Действительно,
 \begin{enumerate}
  \item $ \cov(\xi,\xi) = \D\xi \geqslant 0 $.
  \item Если $ \cov(\xi, \xi) = 0 $, то $ \D\xi = 0 $, и случайная величина равна нулю почти всюду ($ \E\xi = 0$).
  \item $ \cov(\xi,\eta) = \cov(\eta,\xi) $.
  \item $ \cov(a \cdot \xi_1 + b \cdot \xi_2, \eta) = a \cdot \cov(\xi_1, \eta) + b \cdot \cov(\xi_2,\eta) $.
 \end{enumerate}

 В этом пространстве  $ L^{2}_0 $  норма случайной величины равна корню из дисперсии:
 \begin{align*}
  \left\| \xi \right\|_{L^{2}_0} = \sqrt{\cov(\xi,\xi)} = \sqrt{\D\xi},
 \end{align*} то есть, так называемому, \textit{стандартному отклонению}. Коэффициент корреляции
 \begin{align*}
  \rho(\xi,\eta) = \frac{\cov(\xi,\eta)}{\sqrt{\D\xi} \cdot \sqrt{\D\eta}}
 \end{align*} можно понимать как косинус угла между случайными величинами $ \xi $ и $ \eta $. В частности, ортогональность в пространстве $ L^{2}_0 $ --- это некоррелированость случайных величин.

 Очень часто математики ищут скалярные произведения, так как гильбертовы пространства --- это всегда очень хорошо.

\end{remrk}

\subsection{Примеры применения моментов в вероятностном методе.}

\begin{exmpl}[независимые множества в графе]
 Пусть в неориентированном графе $ G = (V,E) $ $ n $ вершин и не более $ \frac{nd}{2} $ рёбер, где $ d \geqslant 1 $. Тогда в $ G $ можно выбрать независимое множество на $ \frac{n}{2d} $ вершинах.
\end{exmpl}
\begin{proof}[\normalfont\textsc{Доказательство}]
 Возьмём случайное множество вершин $ S \subset V $: каждая вершина $ v \in V $ независимо от других попадёт в $ S $ с вероятностью $ p \in (0,1) $, которую мы подберём позже.

 Пусть $ \eta $ --- это случайная величина, равная количеству элементов во множестве $ S $. Тогда $ \E\eta = np $, поскольку $ \eta $ можно выразить как сумму $ n $ индикаторных случайных величин $ \eta_x = \Ind_{\left\{ x \in S \right\}} $, $ x \in V $, для которых верно $ \E\eta_x = P(x \in S) = p $.

 Аналогично, рассмотрим случайную величину $ \xi $, равную количеству рёбер, оба конца которых лежат в $ S $. $ \xi $ можно выразить как сумму индикаторов
 \begin{align*}
  \xi_{xy} = \Ind_{\left\{ x \in S,\, y \in S \right\}}
 \end{align*} по всем рёбрам $ xy \in E $. Так как $ \E\xi_{xy} = p^{2} $ для всех рёбер, и $ \left| E \right| \leqslant \frac{nd}{2} $, то
 \begin{align*}
  \E\xi \leqslant \frac{nd}{2} \cdot p^{2}.
 \end{align*}

 Теперь рассмотрим разность этих величин $ \eta-\xi $. Её матожидание можно оценить снизу так:
 \begin{align*}
  \E(\eta - \xi) \geqslant np - \frac{nd}{2} \cdot p^{2}.
 \end{align*} Максимизируем правую часть по $ p \in (0,1) $. Для этого продифференцируем её:
 \begin{align*}
  \left(np - \frac{nd}{2}\cdot p^{2}\right)'_p = n - ndp.
 \end{align*} Приравняем производную к нулю:
 \begin{align*}
  n - ndp = 0 \iff p = \frac{1}{d}.
 \end{align*} Для вероятности $ p = \frac{1}{d} $ получаем
 \begin{align*}
  \E(\eta-\xi) \geqslant n \cdot \frac{1}{d} - \frac{nd}{2} \cdot \frac{1}{d^{2}} = \frac{n}{2d}.
 \end{align*} Таким образом, найдётся реализация (то есть подмножество $ S \subset V $), на которой $ \eta - \xi \geqslant \frac{n}{2d} $. Возьмём это множество $ S $. Для каждого ребра $ xy \in E$, оба конца которых лежат в $ S $, выкинем из $ S $ любой из концов $ x $ или $ y $. Так как $ \eta-\xi \geqslant \frac{n}{2d} $, то после выбрасываний останется независимое множество из хотя бы $ \frac{n}{2d} $ вершин.
\end{proof}

Рассмотрим следующую иллюстрацию применения моментов случайных величин  в теории чисел.

\begin{problem}
 Обозначим за $ \nu(k) $ количество различных простых делителей натурального числа $ k $. Выберем случайное число $ k $ равновероятно среди чисел $ 1, 2, \ldots, n $. Какое будет $ \nu(k) $?
\end{problem}
\begin{thm}[Харди-Рамануджана]
 Пусть $ \omega(n) \to +\infty $ при $ n \to \infty $. Тогда
 \begin{align*}
  P( \left| \nu(k) - \log\log n \right| > \omega(n) \sqrt{\log\log n}) \to 0
 \end{align*}  при $ n \to \infty $.
\end{thm}
\begin{proof}[\normalfont\textsc{Доказательство (Туран)}]
 Рассмотрим число $ M := \sqrt[10] n $. Для каждого простого числа $ p \leqslant M $ рассмотрим индикаторную случайную величину $ \xi_p $:
 \begin{align*}
  \xi_p(k) = \begin{cases}
   1, \text{ если } k \ \vdots \ p, \\
   0, \text{ иначе. }
  \end{cases}
 \end{align*} Рассмотрим сумму этих индикаторов:
 \begin{align*}
  \xi = \sum_{p \leqslant M} \xi_p.
 \end{align*} Здесь и далее для сумм мы подразумеваем, что $ p $ пробегает простые числа. Так, $ \xi(k) $ равно числу различных простых делителей $ k $, не превосходящих $M$.

 Заметим, что
 \begin{align}
  \label{eq:hardy_ramanujan_xi_and_nu}
  \xi(k) \leqslant \nu(k) < \xi(k) + 10.
 \end{align} Это так, потому что у числа $ k $ не может быть $ 10 $ различных простых делителей $ q_1, q_2, \ldots, q_{10} $, больших $ M $. Ведь иначе $ k $ должно делиться на $ q_1 \cdot \ldots \cdot q_{10} $, однако
 \begin{align*}
  q_1 \cdot \ldots \cdot q_{10} > M^{10} = n.
 \end{align*}

 Таким образом, можно забыть про величину $ \nu(k) $ и оценивать поведение $ \xi(k) $ при $ n \to \infty $.

Найдём матожидание индикатора $ \xi_p $ для простого $ p \leqslant M$:
 \begin{align*}
  \E\xi_p = \frac{\left\lfloor n / p \right\rfloor }{n} = \frac{1}{p} + \OO(1 / n).
 \end{align*} Тогда
 \begin{align*}
  \E\xi = \sum_{p \leqslant M} \E\xi_p = \sum_{p \leqslant M}\left(\frac{1}{p} + \OO ( 1/n ) \right) = \log\log M + \OO(1) = \log\log n + \OO(1).
 \end{align*} Здесь мы воспользовались следующим фактом, который был оставлен без доказательства.
 \begin{lm}[ряд обратных простых]
  \begin{align*}
   \sum_{\substack{p \leqslant n}} \frac{1}{p} = \log\log n + \OO(1).
  \end{align*}
 \end{lm}

 Теперь вычислим дисперсию с помощью формулы \eqref{eq:variance_sum_many}:
 \begin{align*}
  \D\xi = \sum_{p \leqslant M} \D\xi_p + 2 \sum_{p < q \leqslant M} \cov(\xi_p,\xi_q).
 \end{align*}
 Найдём дисперсию индикаторов:
 \begin{align*}
  \D\xi_p &= \E\xi_p^{2} - (\E\xi_p)^{2} = \E\xi_p - (\E\xi_p)^{2} = \frac{1}{p} - \OO( 1 / n) + \left( \frac{1}{p} + \OO(1 / n) \right)^{2} = \\
  &= \frac{1}{p} - \frac{1}{p^{2}} + \OO \left( 1 / n \right).
 \end{align*} Их сумма будет равна
 \begin{align*}
  \sum_{p \leqslant M}\D\xi_p &= \sum_{p \leqslant M} \frac{1}{p} - \sum_{p \leqslant M} \frac{1}{p^{2}} + \sum_{p \leqslant M} \OO(1 / n) = \sum_{p \leqslant M} \frac{1}{p} + \OO(1) = \log\log n + \OO(1),
 \end{align*} так как ряд обратных квадратов, как известно, сходится.

 Оценим ковариацию величин $ \xi_p $ и $ \xi_q $ для простых $ p < q \leqslant M $:
 \begin{align*}
  \cov(\xi_p,\xi_q) &= \E(\xi_p\xi_q) - \E\xi_p \cdot \E\xi_q = \frac{\left\lfloor n / (pq) \right\rfloor }{n} - \frac{\left\lfloor n / p \right\rfloor }{n} \cdot \frac{\left\lfloor n / q \right\rfloor }{n} = \\
  &= \frac{1}{pq} + \OO(1 / n) - \left(\frac{1}{p} + \OO(1 / n)\right) \left(\frac{1}{q} + \OO(1 / n)\right) = \OO(1 / n).
 \end{align*} Тогда сумма ковариаций не превосходит
 \begin{align*}
  \sum_{p < q \leqslant M} \cov(\xi_p,\xi_q) \leqslant M^{2} \cdot \OO(1 / n) = \OO(1).
 \end{align*} Собирая всё вместе, получаем
 \begin{align}
  \label{eq:hardy_ramanujan_E_xi} \E\xi = \log\log n + \OO(1), \\
  \label{eq:hardy_ramanujan_D_xi} \D\xi = \log\log n + \OO(1).
 \end{align}

 Наконец, перейдём теперь к оценке вероятности события из условия. Так как выполнено \eqref{eq:hardy_ramanujan_xi_and_nu}  и \eqref{eq:hardy_ramanujan_E_xi}, то существует $ C > 0 $ такое, что при достаточно больших $ n $ выполнено вложение
 \begin{align*}
  \left\{\left| \nu(k) - \log\log n \right| > \omega(n) \sqrt{\log\log n} \right\} \subset \left\{ \left|\xi(k) - \E\xi \right| > \omega(n)\sqrt{\log\log n} - C \right\}.
 \end{align*} Оценим теперь вероятность с помощью неравенства Чебышёва и оценкой \eqref{eq:hardy_ramanujan_D_xi}:
 \begin{align*}
  P( \left| \xi - \E\xi \right| > \omega(n)\sqrt{\log\log n} - C) &\leqslant \frac{\D\xi}{\left( \omega(n)\sqrt{\log \log n} - C \right)^{2}} = \\
  &= \frac{\log\log n + \OO(1)}{(\omega(n)\sqrt{\log \log n} - C)^{2}} \to 0.
 \end{align*} 
\end{proof}
\begin{remrk}
 На самом деле верно
 \begin{align*}
  \lim_{n \to \infty} \frac{1}{n} \cdot \# \left\{ k \Mid a \leqslant \frac{\nu(k)- \log\log n}{\sqrt{\log \log n}} \leqslant b \right\} = \frac{1}{\sqrt{2\pi}} \int_{a}^{b} e^{-t^{2} / 2}\,dt.
 \end{align*} Доказательство ровно такое же, нужно сослаться только не на Чебышёва, а на ЦПТ.
\end{remrk}

\newpage
\section{Сходимость случайных величин.}

\subsection{Виды сходимостей.}

Сначала докажем техническую, но полезную саму по себе теорему.

\begin{thm}
 Пусть $ \xi_1, \xi_2, \xi_3, \ldots $ --- независимые случайные величины. Пусть есть много функций $ f_i \colon\;\R^{n_i} \to \R $, измеримых по Борелю. Тогда случайные величины
 \begin{align*}
  f_1(\xi_1, \ldots, \xi_{n_1}),\; f_2(\xi_{n_1 + 1}, \ldots, \xi_{n_1 + n_2}),\; f_3(\xi_{n_1 + n_2 + 1}, \ldots, \xi_{n_1 + n_2 + n_3}),\; \ldots
 \end{align*} независимы.
\end{thm}
\begin{proof}[\normalfont\textsc{Доказательство}]
 Сначала докажем для независимых случайных величин
 \begin{align*}
  \xi_1, \ldots, \xi_m, \eta_1, \ldots \eta_n
 \end{align*} и двух измеримых по Борелю функций $ f \colon\,\R^{m}\to\R $, $ g \colon\,\R^{n}\to\R $, что случайные величины $ f(\vec \xi) $ и $ g(\vec \eta) $ независимы. Для этого нужно доказать, что для любых борелевских подмножеств $ \tilde A, \tilde B \subset \R $ события
 \begin{align*}
  \left\{ f(\vec\xi) \in \tilde A \right\}, \; \left\{ g(\vec\eta)\in\tilde B \right\}
 \end{align*} независимы. Но левое событие --- это $ \vec\xi \in f^{-1}(\tilde A) $, а правое событие --- $ \vec\eta \in g^{-1}(\tilde B) $, где прообразы $ f^{-1}(\tilde A), g^{-1}(\tilde B) $ --- борелевские множества.

 Поэтому, достаточно доказать независимость событий $ \vec\xi \in A $ и $ \vec\eta \in B $ для любых борелевских множеств $ A \subset\R^{m} $ и $ B \subset \R^{n} $. А это верно, поскольку совместное распределение $ P_{(\vec\xi,\vec\eta)} $ по теореме \ref{theorem:random_variable_independence_through_distribution_measure_eq} является произведением координатных распределений (в частности, $ P_{(\vec\xi,\vec\eta)} = P_{\vec\xi} \times P_{\vec\eta} $):
 \begin{align*}
  P(\vec\xi \in A,\,\vec\eta\in B) &= P((\vec\xi, \vec\eta) \in A \times B) = P_{(\vec\xi,\vec\eta)}(A \times B) = P_{\vec\xi}(A) \cdot P_{\vec\eta}(B) = \\
 &= P(\vec\xi \in A) \cdot P(\vec\eta \in B).
 \end{align*}
\end{proof}

\begin{df}[сходимость почти наверное]
 \label{def:converge_almost_everywhere}
 Последовательность случайных величин $ \{\xi_{n}\}_{n=1}^{\infty}  $ сходится к случайной величине $ \xi $  \textit{почти наверное}, если последовательность чисел $ \{\xi_{n}(\omega)\}_{n=1}^{\infty}  $  сходится к числу $ \xi(\omega) $ для почти всех точек $ \omega\in\Omega $:
 \begin{align*}
  P \left[ \omega\in\Omega\Mid \lim_{n \to \infty} \xi_n(\omega) = \xi(\omega) \right] = 1.
 \end{align*} Обозначение: $ \xi_n\to\xi $ п.н.
\end{df}

\begin{df}[сходимость в среднем порядка $ r $]
 \label{def:converge_average_ord_r}
 Последовательность случайных величин $ \{\xi_{n}\}_{n=1}^{\infty}  $ сходится к случайной величине $ \xi $   \textit{в среднем порядка $ r \geqslant 1 $}, если
 \begin{align*}
  \lim_{n \to \infty} \E \left| \xi_n-\xi \right|^{r} = 0.
 \end{align*}
\end{df}
\begin{remrk*}
 Сходимость в среднем порядка $ r \geqslant 1 $ --- это в точности сходимость в пространстве Лебега $ L^{r}(\Omega,P) $.
\end{remrk*}

\begin{df}[сходимость по вероятности]
 \label{def:converge_probability}
 Последовательность случайных величин $ \{\xi_{n}\}_{n=1}^{\infty}  $ сходится к случайной величине $ \xi $ \textit{по вероятности}, если для любого числа $ \eps > 0 $ выполнено
 \begin{align*}
  \lim_{n \to \infty} P(\left| \xi_n-\xi \right| \geqslant \eps) = 0.
 \end{align*} Обозначение: $ \xi_n \xrightarrow{P} \xi $.
\end{df}

Отметим, что для определений \ref{def:converge_almost_everywhere}, \ref{def:converge_average_ord_r}, \ref{def:converge_probability} требуется, чтобы случайные величины $ \xi_1, \xi_2, \ldots, \xi $ были заданы на одном вероятностном пространстве. В следующем виде сходимости такого требования уже нет.

\begin{df}[сходимость по распределению]
 Последовательность случайных величин $ \{\xi_{n}\}_{n=1}^{\infty}  $ сходится к случайной величине $ \xi $  \textit{по распределению}, если для каждой точки непрерывности $ x \in \R $ функции распределения $ F_\xi $ верно
 \begin{align*}
  \lim_{n \to \infty} F_{\xi_n}(x) = F_\xi(x).
 \end{align*}

 Сходимость по распределению также называют \textit{слабой сходимостью}.
\end{df}

Изучим связь между этими видами сходимости.

\begin{prop}
 Из сходимости почти наверное следует сходимость по вероятности.
\end{prop}
\begin{proof}[\normalfont\textsc{Доказательство}]
 Возьмём любой $ \eps > 0 $. Для $ n \geqslant 1 $ рассмотрим событие
 \begin{align*}
  B_n = \left\{ \sup_{k \geqslant n} \left| \xi_k - \xi \right| \geqslant \eps \right\}.
 \end{align*} Мы докажем $ P(B_n) \to 0 $, из этого уж точно последует $ P(\left| \xi_n-\xi \right| \geqslant \eps) \to 0 $. Рассмотрим событие
 \begin{align*}
  B = \bigcap_{n=1}^{\infty} B_n.
 \end{align*} В каждой точке $ \omega \in B $ не может быть сходимости $ \xi_n(\omega) \to \xi(\omega) $, ведь $ \left| \xi_k(\omega) - \xi(\omega) \right| \geqslant \eps $ для бесконечного многих $ k $. Раз так, то $ P(B) = 0 $. Но так как множества $ B_n $ убывают:
 \begin{align*}
  B_1 \supset B_2 \supset \ldots
 \end{align*} то по непрерывности конечной меры снизу имеем
 \begin{align*}
  \lim_{n \to \infty} P(B_n) = P(B) = 0.
 \end{align*}
\end{proof}

\begin{prop}
 Из сходимости в среднем порядка $ r \geqslant 1 $ следует сходимость по вероятности.
\end{prop}
\begin{proof}[\normalfont\textsc{Доказательство}]
 Действительно, для любого $ \eps > 0 $
  \begin{align*}
  P(\left| \xi_n-\xi \right| \geqslant \eps) \leqslant \frac{\E \left| \xi_n - \xi \right|^{r}}{\eps^{r}} \to 0
 \end{align*} по обобщённому неравенству Маркова \eqref{eq:general_markov_ineq} для функции $ x \mapsto x^{r} $.
\end{proof}

\begin{prop}
 Из сходимости по вероятности следует сходимость по распределению.
\end{prop}
\begin{proof}[\normalfont\textsc{Доказательство}]
 Пусть $ x \in \R $ --- точка непрерывности $ F_\xi $. Докажем отдельно оценки для верхнего и нижнего пределов.
 \begin{itemize}
  \item Возьмём любой $ \eps > 0 $. Заметим следующее включение событий:
   \begin{align*}
    \left\{ \xi_n \leqslant x \right\} \subset \left\{ \xi \leqslant x + \eps \right\} \cup \left\{ \left| \xi_n-\xi \right| \geqslant \eps \right\}.
   \end{align*} Припишем с обеих сторон вероятности:
   \begin{align*}
    F_{\xi_n}(x) \leqslant F_{\xi}(x + \eps) + P(\left| \xi_n-\xi \right| \geqslant \eps).
   \end{align*} Так как есть сходимость по вероятности, то
   \begin{align*}
    \limsup_{n \to \infty} F_{\xi_n}(x) \leqslant F_\xi(x + \eps).
   \end{align*} Устремляя $ \eps \to 0 $, получаем
   \begin{align*}
    \limsup_{n \to \infty} F_{\xi_n}(x) \leqslant F_\xi(x).
   \end{align*}
  \item Возьмём любой $ \eps > 0 $. Заметим включение
   \begin{align*}
    \left\{ \xi_n > x \right\} \subset \left\{ \xi > x - \eps \right\} \cup \left\{ \left| \xi_n-\xi \right| \geqslant \eps \right\}.
   \end{align*} Припишем вероятности и перейдём к дополнению:
   \begin{align*}
    P(\xi_n > x) &\leqslant P(\xi > x - \eps) + P(\left| \xi_n-\xi \right| \geqslant \eps) \\
    \implies 1 - P(\xi_n > x) &\geqslant 1 - P(\xi > x - \eps) - P(\left| \xi_n-\xi \right| \geqslant \eps) \\
    \iff F_{\xi_n}(x) &\geqslant F_\xi(x - \eps) - P(\left| \xi_n-\xi \right| \geqslant \eps).
   \end{align*} Так как есть сходимость по вероятности, то
   \begin{align*}
    \liminf_{n \to \infty} F_{\xi_n}(x) \geqslant F_\xi(x - \eps).
   \end{align*} Устремляя $ \eps \to 0 $ и пользуясь непрерывностью в точке $ x $, получаем
   \begin{align*}
    \liminf_{n \to \infty} F_{\xi_n}(x) \geqslant F_\xi(x).
   \end{align*}
 \end{itemize} Итого,
 \begin{align*}
  \lim_{n \to \infty} F_{\xi_n}(x) = F_\xi(x).
 \end{align*}
\end{proof}

Приведём теперь отрицательные примеры, показывающие, что других следствий не может быть.

\begin{exmpl}
 Из сходимости почти наверное не следует сходимость в среднем порядка $ r \geqslant 1 $.

 Пусть нам дано число $ r \geqslant 1 $. Рассмотрим вероятностное пространство $ [0,1] $ с мерой Лебега. Рассмотрим случайные величины
 \begin{align*}
  \xi_n = n^{1 / r} \cdot \Ind_{[0,1 / n]}.
 \end{align*} Очевидно, есть сходимость к нулю почти всюду (на самом деле везде, кроме точки $ 0 $). Но сходимости в среднем порядка $ r $ нет:
 \begin{align*}
  \E \left| \xi_n - 0 \right|^{r} = n \cdot \E\Ind_{[0, 1 / n]}^{r} = n \cdot \E\Ind_{[0,1 / n]} = 1 \not\rightarrow 0.
 \end{align*}
\end{exmpl}

Автоматически получаем, что из сходимости по вероятности не следует сходимость в среднем порядка $ r $.

\begin{exmpl}
 Из сходимости в среднем порядка $ r \geqslant 1 $ не следует сходимость почти наверное.

 Рассмотрим вероятностное пространство $ [0,1] $ с мерой Лебега. Рассмотрим последовательность случайных величин $\xi_n$ как на рисунке \ref{fig:function_sequence_converges_by_measure_but_not_in_any_point}: \begin{align*}
  &\xi_1 = \Ind_{\left[0,1\right]},\\
  &\xi_2 = \Ind_{\left[0, \frac{1}{2}\right]}, &\xi_3 = \Ind_{\left[ \frac{1}{2}, 1 \right]},\\
  &\xi_4 = \Ind_{\left[ 0,\frac{1}{4} \right]}, &\xi_5 = \Ind_{\left[ \frac{1}{4},\frac{1}{2} \right]}, &&\xi_6 = \Ind_{\left[ \frac{1}{2},\frac{3}{4} \right]}, &&\xi_7=\Ind_{\left[ \frac{3}{4},1 \right]}, \\
  &\qquad\vdots
 \end{align*}
 Тогда есть сходимость к нулю в среднем любого порядка $ r \geqslant 1 $:
 \begin{align*}
  \E \left| \xi_n - 0 \right|^{r} = \E \xi_n \to 0,
 \end{align*} так как длины отрезков стремятся к нулю. Но ни в одной точке $ \omega \in [0,1] $ нет сходимости $ \xi_n(\omega) \to 0 $: в каждой точке бесконечно много раз повстречается значение $ 0 $ и значение $ 1 $.
\end{exmpl}

Автоматически получаем, что из сходимости по вероятности не следует сходимость в среднем порядка $ r $.

\begin{figure}[ht]
 \centering
 \incfig{function_sequence_converges_by_measure_but_not_in_any_point}
 \caption{Последовательность случайных величин $\xi_n$, сходящаяся к $0$ по вероятности, но не сходящаяся ни в одной точке отрезка $[0,1]$.}
 \label{fig:function_sequence_converges_by_measure_but_not_in_any_point}
\end{figure}

Говорить о том, что из сходимости по распределению следует сходимость по вероятности, некорректно: ведь в сходимости по распределению случайные величины последовательности могут быть заданы на разных вероятностных пространствах. Однако, можно придумать пример, где последовательность величин, заданных на одном вероятностном пространстве, сходится по распределению, но не сходится по вероятности (упражнение).

\end{document}

