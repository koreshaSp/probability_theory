% 2023.04.11 lecture 09
\documentclass[../main.tex]{subfiles}
\begin{document}

\newpage
\section{Сходимость по распределению.}

Напомним, что \textit{сходимостью по распределению} (или \textit{слабой сходимостью}) случайных величин $ \xi_n $ к величине $ \xi $ называется сходимость функций распределения $ F_{\xi_n}(x) \to F_\xi(x) $ в точках непрерывности $ x $ функции $ F_\xi $.

В этом параграфе основным результатом будет большая теорема о равносильных формулировках сходимости по распределению. Начнём с подготовительных утверждений.

\begin{remrk}
 \label{remark:sum_with_continuous_rv_is_continuous}
 Пусть cлучайные величины $ \xi $ и $ \eta $ независимы, и $ \eta $ непрерывна. Тогда $ \xi + \eta $ непрерывна.
\end{remrk}
\begin{proof}[\normalfont\textsc{Доказательство}]
 Докажем непрерывность в точке $ a \in \R $, вычислив свёртку распределений:
 \begin{align*}
  P_{\xi+\eta}(\left\{ a \right\}) = P_\xi \ast P_\eta(\left\{ a \right\}) = \int_{\R} P_{\eta}(\left\{ a-x \right\}) dP_\xi(x) = \int_{\R} 0\,dP_\xi = 0, 
 \end{align*} где $ P_\eta(\left\{ y \right\}) = 0 $ по непрерывности $ \eta $.
\end{proof}

Удивительно, что непрерывность второй случайной величины $ \xi $ здесь не требуется.

\begin{remrk}
 \label{remark:open_set_partitioned_into_cells_with_countable_bad_points}
 Пусть $ U \subset \R $ --- открытое множество, а $ D $  --- не более, чем счётное множество <<плохих>> точек. Тогда $ U $ можно разбить на счётное число ячеек
 \begin{align*}
  U = \bigsqcup_{n=1}^{\infty} \left(a_n, b_n\right],
 \end{align*} концы которых <<хорошие>>: $ a_n, b_n \notin D $.
\end{remrk}
\begin{proof}[\normalfont\textsc{Доказательство}]
 Берём ячейки с шагом $ 1 $. Какие-то попали целиком, а какие-то не влезли. Потом берём шаг $ 1 / 2 $ и так далее.

 Если заботимся о концах: делим не совсем пополам, а примерно (так чтобы точка деления не лежала в $ D $).
\end{proof}

\begin{remrk}
 \label{remrk:lim_f_n_using_f_n_a_minnus_f_n_b}
 Пусть $ F_n $ и $ F $ --- функции распределения такие, что для любых $ a, b \in \R $:
 \begin{align*}
  \lim_{n \to \infty} (F_n(b) - F_n(a)) = F(b)  - F(a).
 \end{align*} Тогда для любого $ x \in \R $
 \begin{align*}
  \lim_{n \to \infty} F_n(x) = F(x).
 \end{align*}
\end{remrk}
\begin{proof}
 Возьмём $ x \in\R $. Тогда для любой точки $ a < x $ выполнено
 \begin{align*}
  \lim_{n \to \infty} (F_n(x) - F(x) - (F_n(a)-F(a))) = 0.
 \end{align*} Так как 
\end{proof}
\begin{proof}[\normalfont\textsc{Доказательство}]
 От функций $ F_n $ и $ F $ нужна лишь ограниченность отрезком $ [0,1] $ и существование пределов на бесконечностях, равных $ 0 $ (на $ -\infty $) и $ 1 $ (на $ +\infty $).

 Возьмём любой $ \eps > 0 $. Так как $ F(a) \to 0 $ при $ a \to -\infty $, то существует $ a \in \R $ такое, что $ F(a) < \eps $. Аналогично, так как $ F(b) \to 1 $ при $ b \to +\infty $, то существует $ b \in \R $ такое, что $ F(b) > 1 - \eps $. Зафиксируем эти $ a $, $ b $ раз и навсегда.

 Так как $ F(b) - F(a) > 1 - 2\eps $, то для $ \eps $ существует $ N $ такое, что для всех $ n > N $ выполнено
 \begin{align*}
  \left|(F_n(b) - F_n(a)) - (F(b)-F(a)) \right| < \eps,
 \end{align*} из чего следует $ F_n(b) - F_n(a) > 1 - 3\eps $ и $ F_n(a) < 3\eps $ при $ n > N $.

 Рассмотрим любую точку $ x \in \R $. Тогда для $ \eps $ существует $ N' $ такое, что при всех $ n > N' $ выполнено
 \begin{align*}
  \left|(F_n(x) - F_n(a)) - (F(x)-F(a)) \right| < \eps.
 \end{align*} Следовательно,
 \begin{align*}
  \left|F_n(x) - F(x) \right| \leqslant \left| (F_n(x) - F_n(a)) - (F(x) - F(a)) \right| + F_n(a) + F(a) < 5\eps
 \end{align*} при $ n > \max(N,N') $.
\end{proof}

\begin{prop}
 У функции распределения не более, чем счётно много точек разрыва.
\end{prop}
\begin{proof}[\normalfont\textsc{Доказательство}]
 Уже было пояснено ранее, в доказательстве следствия \ref{corollary:char_func_defines_distribution}. Другое доказательство: в каждом интервале разрыва можно выбрать рациональное число.
\end{proof}

\begin{df}
 Множество $ B \subset \R $ \textit{регулярно} относительно распределения $ P_\xi $, если мера его границы нулевая: $ P_\xi(\mathrm{Cl}\,B \setminus \mathrm{Int}\,B) = 0 $.

 В этом случае $ P(\xi \in A) = P(\xi \in B) $, если $ \mathrm{Int}\,B \subset A \subset \mathrm{Cl}\,B $.

\end{df}

Наконец-то мы готовы приступить к основной теореме параграфа.

\begin{thm}[о сходимости по распределению]
 \label{theorem:weak_convergence}
 Пусть $ \xi, \xi_1, \xi_2, \ldots $ --- последовательность случайных величин, $ F, F_1, F_2, \ldots $ --- их функции распределения, и $ \varphi, \varphi_1, \varphi_2, \ldots $ --- их характеристические функции соответственно. Следующие условия равносильны.
 \begin{enumerate}
  \item \label{i1:weak_convergence} $ \xi_n $ сходится к $ \xi $ по распределению:
   \begin{align*}
    \lim_{n \to \infty} F_n(x) = F(x)
   \end{align*} для любой точки непрерывности $ x $ функции распределения $ F $.
  \item \label{i2:weak_convergence} Для любого открытого множества $ U \subset \R $
   \begin{align*}
    \liminf_{n \to \infty} P(\xi_n \in U) \geqslant P(\xi \in U).
   \end{align*}
  \item \label{i3:weak_convergence} Для любого замкнутого множества $ A \subset \R $
   \begin{align*}
    \limsup_{n \to \infty} P(\xi_n \in A) \leqslant P(\xi \in A).
   \end{align*}
  \item \label{i4:weak_convergence} Для любого регулярного (относительно $ P_\xi $) борелевского множества $ B \subset \R$
   \begin{align*}
    \lim_{n \to \infty} P(\xi_n \in B) = P(\xi \in B).
   \end{align*}
  \item \label{i5:weak_convergence} Для любого регулярного (относительно $ P_\xi $) борелевского множества $ B \subset\R $
   \begin{align*}
    \lim_{n \to \infty} \E \Ind_B(\xi_n) = \E \Ind_B(\xi).
   \end{align*}
  \item \label{i6:weak_convergence} Для любой непрерывной ограниченной вещественнозначной функции $ f \in C(\R) $
   \begin{align*}
    \lim_{n \to \infty} \E f(\xi_n) = \E f(\xi).
   \end{align*}
  \item \label{i7:weak_convergence} Характеристические функции $ \varphi_n(t) $ сходятся к характеристической функции $ \varphi(t) $ всюду в $ \R $:
   \begin{align*}
    \lim_{n \to \infty} \varphi_n(t) = \varphi(t),
   \end{align*} для любой точки $ t \in \R $.
 \end{enumerate}
\end{thm}
\begin{proof}[\normalfont\textsc{Доказательство}]
 Докажем сначала простые следствия и эквивалентности.
 \begin{itemize}
  \item Условие \ref{i2:weak_convergence} равносильно условию \ref{i3:weak_convergence}. Здесь достаточно перейти к дополнению: для открытого $ U \subset \R$ взять замкнутое $ A = \R \setminus U $. Тогда $ P(\xi_n \in A) = 1 - P(\xi_n \in U) $ для всех $ n $, и
   \begin{align*}
    \limsup_{n \to \infty} P(\xi_n \in A) = 1 - \liminf_{n \to \infty} P(\xi_n \in U).
   \end{align*} Тогда если
   \begin{align*}
    \liminf_{n \to \infty} P(\xi_n \in U) \geqslant P(\xi \in U),
   \end{align*} то
   \begin{align*}
    \limsup_{n \to \infty} P(\xi_n\in A) \leqslant 1 - P(\xi \in U) = P(\xi\in A).
   \end{align*} Наоборот, если
   \begin{align*}
    \limsup_{n \to \infty} P(\xi_n \in A) \leqslant P(\xi \in A),
   \end{align*} то
   \begin{align*}
    \liminf_{n \to \infty} P(\xi_n \in U) \geqslant 1 - P(\xi \in A) = P(\xi \in U).
   \end{align*}

  \item Из условий \ref{i2:weak_convergence}, \ref{i3:weak_convergence} следует условие \ref{i4:weak_convergence}. Для регулярного борелевского множества $ B \subset \R $ возьмём открытое множество $ U = \mathrm{Int}\,B $ и замкнутое множество $ A =\mathrm{Cl}\,B $. Пользуясь условиями \ref{i2:weak_convergence}, \ref{i3:weak_convergence}, запишем цепочку неравенств:
   \begin{align*}
    P(\xi \in U) &\leqslant \liminf_{n \to \infty} P(\xi_n \in U) \leqslant \\
    &\leqslant \liminf_{n \to \infty} P(\xi_n \in B) \leqslant \limsup_{n \to \infty} P(\xi_n \in B) \leqslant \\
    &\leqslant \limsup_{n \to \infty} P(\xi_n \in A) \leqslant P(\xi \in A).
   \end{align*} Но так как множество $ B $ регулярное, то $ P(\xi\in U) = P(\xi\in B) = P(\xi\in A) $, и все неравенства выше --- равенства. Тогда верхний предел равен нижнему, и мы получаем
   \begin{align*}
    \lim_{n \to \infty} P(\xi_n \in B) = P(\xi\in B).
   \end{align*}

  \item Условие \ref{i5:weak_convergence} --- это то же условие \ref{i4:weak_convergence}, но по-другому записанное:
   \begin{align*}
    \E \Ind_B (\xi) = P(\xi \in B), && \E \Ind_B(\xi_n) = P(\xi_n \in B).
   \end{align*}

  \item Из условия \ref{i6:weak_convergence} следует условие \ref{i7:weak_convergence}:
   \begin{align*}
    \varphi_n(t) &= \E e^{it\xi_n} = \E \cos(t\xi_n) + i \cdot \E\sin(t\xi_n) \to \\
    &\to \E \cos(t\xi) + i \cdot \E\sin(t\xi) = \E e^{it\xi} = \varphi(t).
   \end{align*} $ x \mapsto \cos(tx) $ и $ x \mapsto \sin(tx) $ --- те самые непрерывные ограниченные вещественнозначные функции.
 \end{itemize}

 Осталось доказать три сложных следования.

 \begin{itemize}
  \item Из условия \ref{i1:weak_convergence} следует условие \ref{i2:weak_convergence}. Обозначим за $ D $ не более чем счётное множество точек разрыва функции распределения $ F $. Пользуясь замечанием  \ref{remark:open_set_partitioned_into_cells_with_countable_bad_points}, разобьём открытое множество $ U $ на счётное число ячеек:
   \begin{align*}
    U = \bigsqcup_{k=1}^{\infty} \left(a_k, b_k\right],
   \end{align*} концы которых являются точками непрерывности $ F $: $ a_n, b_n \notin D $.

   Для каждого конечного  $ m \geqslant 1 $  имеем
   \begin{align*}
    P(\xi_n \in U) &\geqslant P \left[ \xi_n\in \bigsqcup_{k=1}^{m} \left(a_k, b_k\right]   \right] = \\
    &= \sum_{k=1}^{m} P(\xi_n \in \left(a_k, b_k\right]  ) = \sum_{k=1}^{m} (F_n(b_k) - F_n(a_k)).
   \end{align*} Пририсуем пределы с обеих сторон:
   \begin{align*}
    \liminf_{n \to \infty} P(\xi_n \in U) \geqslant \lim_{n \to \infty} \sum_{k=1}^{m}(F_n(b_k) - F_n(a_k)) = \sum_{k=1}^{m} F(b_k)-F(a_k),
   \end{align*} где последнее равенство верно из-за сходимости по распределению и из-за того, что $ a_k,b_k \notin D $. Так как это верно для любого $ m $, то
   \begin{align*}
    \liminf_{n \to \infty} P(\xi_n\in U) \geqslant \sum_{k=1}^{\infty}F(b_k)-F(a_k) = P \left( \xi \in \bigsqcup_{k=1}^{\infty} \left(a_k, b_k\right]   \right) = P(\xi\in U).
   \end{align*}

  \item Из условия \ref{i5:weak_convergence} следует условие \ref{i6:weak_convergence}. Идея доказательства естественная: приблизить непрерывную ограниченную функцию простой функцией --- линейной комбинацией индикаторных функцией, множества которых являются борелевскими и регулярными относительно $ P_\xi $. Однако само доказательство длинновато из-за технических деталей.

   Пусть $ f \in C(\R)$ --- непрерывная ограниченная вещественнозначная функция. Рассмотрим следующее множество плохих точек:
   \begin{align*}
    D = \left\{ x \in \R \mid P(f(\xi) = x) > 0 \right\}.
   \end{align*} Множество $ D $ не более, чем счётно, так как
   \begin{align*}
    \sum_{x \in D} P(f(\xi) = x) \leqslant 1.
   \end{align*}

   Пусть функция $ f $ ограничена числом $ M > 0 $:  $ \left| f \right| < M $, причём $ M \notin D $ и $ -M \notin D $ (если всё-таки $ M $ или $ -M $ попало в $ D $, то увеличим $ M $ чуть-чуть). Разобьем отрезок $ [-M,M] $ на $ m $ маленьких отрезочков с разрезами по точкам
   \begin{align*}
    -M = t_0,\;t_1,\;t_2,\;\ldots,\;t_{m-1},\;t_m=M
   \end{align*} так, чтобы $ t_j \notin D $ для всех $ 0 \leqslant j \leqslant m $, и чтобы длины отрезков были малы:
   \begin{align*}
    t_j - t_{j-1} < \eps = \frac{2M}{m}
   \end{align*} для всех $ 1 \leqslant j \leqslant m $.
   \begin{figure}[ht]
    \centering
    \incfig[0.8]{7_points_theorem_cut_m}
    \caption{Разбиение отрезка $[-M, M]$ на малые отрезки с хорошими концами.}
    \label{fig:7_points_theorem_cut_m}
   \end{figure}

   Разобьем вещественную прямую на следующие $ m $ множеств:
   \begin{align*}
    \R = \bigsqcup_{j=1}^{m} B_j, &&
    B_j=\left\{x\in\R\mid t_{j-1} < f(x) \leqslant t_j\right\}.
   \end{align*} Множества $ B_j $ покрывают всю вещественную прямую, так как $ f $ ограничена числом $ M $. При этом, множества $ B_j $ борелевские как непрерывные прообразы борелевских множеств.

   Проверим, что каждое множество $ B_j $ регулярное относительно распределения $ P_\xi $. Для этого рассмотрим множества
   \begin{align*}
    U_j &= \left\{ x \in \R \mid t_{j-1} < f(x) < t_j \right\},\\
    A_j &= \left\{ x \in \R \mid t_{j-1} \leqslant f(x) \leqslant t_j  \right\}.
   \end{align*} Множество $ U_j $ открыто как непрерывный прообраз открытого, а множество $ A_j $ замкнуто как непрерывный прообраз замкнутого. Поэтому, верна цепочка включений
   \begin{align*}
    U_j \subset \mathrm{Int}\,B_j \subset B_j \subset \mathrm{Cl}\,B_j \subset A_j.
   \end{align*} Тогда
   \begin{align*}
    P_\xi(\mathrm{Cl}\,B_j \setminus \mathrm{Int}\,B_j) \leqslant P_\xi(A_j \setminus U_j) = P(f(\xi) \in \left\{ t_{j-1},t_j \right\}) = 0,
   \end{align*} так как $ t_{j-1},t_j \notin D $. Регулярность $ B_j $ проверена.

   Построим следующую простую функцию
   \begin{align*}
    g(x) = \sum_{k=1}^{m} t_{j-1}\Ind_{B_j}(x),
   \end{align*} приближающую $ f $. По её построению верно
   \begin{align*}
    \left| f(x)-g(x) \right| < \eps
   \end{align*} всюду в $ \R $. Следовательно,
   \begin{align*}
    \E \left| f(\xi_n)-g(\xi_n) \right| < \eps, && \E \left| f(\xi)-g(\xi) \right| < \eps.
   \end{align*}

   Раз все $ B_j $ регулярные и борелевские, то по условию \ref{i5:weak_convergence}
   \begin{align*}
    \lim_{n \to \infty} \E \Ind_{B_j}(\xi_n) = \E \Ind_{B_j}(\xi)
   \end{align*} для каждого $ j $. По линейности математического ожидания
   \begin{align*}
    \lim_{n \to \infty} \E g(\xi_n) = \E g(\xi).
   \end{align*}

   Теперь применим $ 3\eps $-приём:
   \begin{align*}
    &\left| \E f(\xi_n) - \E f(\xi) \right| \leqslant \\
    \leqslant\;&\left| \E f(\xi_n) - \E g(\xi_n) \right| + \left| \E g(\xi_n) - \E g(\xi) \right| + \left| \E g(\xi) - \E f(\xi) \right| < 3\eps
   \end{align*} при достаточно больших $ n $. Устремляя $ m \to \infty $  ($ \eps = \frac{2M}{m} \to 0 $), получаем
   \begin{align*}
    \lim_{n \to \infty} \E f(\xi_n)=\E f(\xi).
   \end{align*}
 \end{itemize}

 Осталось одно, самое сложное следование.

 \begin{itemize}
  \item Из условия \ref{i7:weak_convergence} следует условие \ref{i1:weak_convergence}. 

   Чтобы перейти от характеристических функций к функциям распределения, мы хотели бы воспользоваться формулой обращения (теорема \ref{theorem:inversing_formula}):
   \begin{align*}
    P(\xi \in \left(a, b\right]) = \frac{1}{2\pi} \lim_{T \to +\infty} \int_{-T}^{T} \frac{e^{-iat}-e^{-ibt}}{it} \cdot \varphi_\xi(t)\,dt,
   \end{align*} где $ P(\xi=a)=P(\xi=b)=0 $. Проблема здесь состоит в том, что интеграл не настоящий, а в смысле главного значения. Решать проблему будем так: добавим ко всем случайным величинам некоторую малую случайную величину так, чтобы с одной стороны значения поменялись не сильно, а с другой стороны характеристическая функция улучшилась.

   Добавим ко всем величинам случайную величину $ \eta \sim \Norm(0,\sigma^{2}) $, не зависящую от всех $ \xi, \xi_1, \xi_2, \ldots $, где $ \sigma $ очень мало (подбирать его мы будем позже). Формальное доказательство возможности взять такую независимую величину очень занудно, поэтому мы не будем его приводить. Тогда
   \begin{align*}
    &\varphi_{\xi_n + \eta}(t) = \varphi_{\xi_n}(t) \cdot \varphi_\eta(t) = \varphi_{\xi_n}(t) \cdot e^{-\frac{\sigma^{2}t^{2}}{2}},\\
    &\varphi_{\xi+\eta}(t)=\varphi_{\xi}\cdot\varphi_\eta(t)=\varphi_\xi(t)\cdot e^{-\frac{\sigma^{2}t^{2}}{2}}
    %,\\ &\lim_{n \to \infty} \varphi_{\xi_n+\eta}(t)=\varphi_{\xi+\eta}(t)
   .\end{align*} Также, случайные величины $ \xi + \eta $ и $ \xi_n + \eta $ непрерывны по замечанию \ref{remark:sum_with_continuous_rv_is_continuous}, поэтому при использовании формулы обращения нам не будет нужна оговорка о точках непрерывности. Наконец, заметим, что интеграл в формуле обращения величин $ \xi_n $
   \begin{align*}
    \int_{\R} \frac{e^{-iat}-e^{-ibt}}{it} \cdot \varphi_{\xi_n}(t) \cdot e^{-\frac{\sigma^{2}t^{2}}{2}}\,dt
   \end{align*} сходится абсолютно. Действительно, интеграл
   \begin{align*}
    \int_{\R} e^{-\frac{\sigma^{2}t^{2}}{2}}\,dt
   \end{align*} абсолютно сходится, а остальные множители ограничены:
   \begin{align*}
    \left| \varphi_{\xi_n}(t) \right| \leqslant 1, && \left| \frac{e^{-iat}-e^{-ibt}}{it} \right| \leqslant C.
   \end{align*} Поэтому, в формуле обращения можно вместо интеграла в смысле главного значения писать обычный интеграл: для любого $ n $ и любых $ a < b $ верно
   \begin{align*}
    P(a \leqslant \xi_n + \eta \leqslant b) = \frac{1}{2\pi} \int_{\R}   \frac{e^{-iat}-e^{-ibt}}{it} \cdot \varphi_{\xi_n}(t) \cdot e^{-\frac{\sigma^{2}t^{2}}{2}}\,dt,
   \end{align*} и те же самые рассуждения справедливы для $ \xi $:
   \begin{align*}
    P(a\leqslant\xi+\eta\leqslant b)=\frac{1}{2\pi}\int_\R\frac{e^{-iat}-e^{-ibt}}{it}\cdot\varphi_\xi(t)\cdot e^{-\frac{\sigma^{2}t^{2}}{2}}\,dt.
   \end{align*}

   Так как $ \varphi_{\xi_n}(t) \to \varphi_\xi(t) $ поточечно, то по теореме Лебега о мажорируемой сходимости 
   \begin{align}
    \label{eq:weak_convergence:7_implies_1:lim_of_P_a_b}
    \lim_{n \to \infty} P(a \leqslant \xi_n+\eta \leqslant b) = P(a \leqslant \xi+\eta \leqslant b),
   \end{align} где мажоранта --- это суммируемая функция $ t \mapsto C e^{-\frac{\sigma^{2}t^{2}}{2}} $.

   Равенство \eqref{eq:weak_convergence:7_implies_1:lim_of_P_a_b} можно записать в терминах функций распределения:
   \begin{align*}
    \lim_{n \to \infty} (G_n(b)-G_n(a)) = G(b)-G(a)
   \end{align*} для любых $ a,b \in \R $, где $ G_n $ и $ G $ --- функции распределения случайных величин $ \xi_n+\eta $ и $ \xi+\eta $ соответственно. Тогда по замечанию \ref{remrk:lim_f_n_using_f_n_a_minnus_f_n_b} для любого $ x \in \R $ выполнено
   \begin{align*}
    \lim_{n \to \infty} G_n(x)=G(x).
   \end{align*} Мы доказали сходимость по распределению величин $ \xi_n+\eta $ к величине $ \xi+\eta $. Теперь нам осталось избавиться от $ \eta $, то есть перейти от $ G $ к $ F $.

   Пусть $ x \in \R $ --- точка непрерывности функции распределения $ F $. Зафиксируем число $ \eps > 0 $. Выберем по непрерывности $ \delta > 0 $ для $ \eps $ такое, что
   \begin{align*}
    \left| F(y)-F(x) \right| < \eps
   \end{align*} при $ x - 2\delta \leqslant y \leqslant x + 2\delta $. Оценим теперь $ F_n(x) $ отдельно сверху и снизу.

   Оценка сверху:
   \begin{align*}
    F_n(x)&=P(\xi_n \leqslant x) \leqslant P(\xi_n + \eta \leqslant x + \delta) + P(\left| \eta \right| \geqslant \delta) = \\
    &= G_n(x+\delta) + P(\left| \eta \right| \geqslant \delta).
   \end{align*} По неравенству Чебышева \eqref{equation:chebyshev_inequality}:
   \begin{align*}
    F_n(x) &\leqslant G_n(x + \delta) + \frac{\D\eta}{\delta^{2}} < G(x+\delta) + \eps + \frac{\sigma^{2}}{\delta^{2}} = \\
    &= P(\xi+\eta \leqslant x + \delta) + \eps + \frac{\sigma^{2}}{\delta^{2}} \leqslant \\
    &\leqslant P(\xi \leqslant x + 2\delta) + P(\left| \eta \right| \geqslant \delta) + \eps + \frac{\sigma^{2}}{\delta^{2}} \leqslant \\
    &\leqslant F(x+2\delta) + \eps +  \frac{2\sigma^{2}}{\delta^{2}} < F(x) + 2\eps + \frac{2\sigma^{2}}{\delta^{2}}
   \end{align*} при достаточно больших $ n $.

   Оценка снизу полностью аналогична:
   \begin{align*}
    F_n(x) &= P(\xi_n \leqslant x) \geqslant P(\xi_n + \eta \leqslant x - \delta) - P(\left| \eta \right| \geqslant \delta) \geqslant \\
    &\geqslant G_n(x-\delta) - \frac{\sigma^{2}}{\delta^{2}} > G(x-\delta)-\eps-\frac{\sigma^{2}}{\delta^{2}} = \\
    &= P(\xi+\eta \leqslant x - \delta)-\eps-\frac{\sigma^{2}}{\delta^{2}} \geqslant \\
    &\geqslant P(\xi \leqslant x - 2\delta) - P(\left| \eta \right| \geqslant \delta) - \eps - \frac{\sigma^{2}}{\delta^{2}} \geqslant \\
    &\geqslant F(x-2\delta) - \eps - \frac{2\sigma^{2}}{\delta^{2}} > F(x) - 2\eps - \frac{2\sigma^{2}}{\delta^{2}}
   \end{align*} при больших $ n $.

   Итого, для каждого $ \eps > 0 $ можно выбрать достаточно малое $ \delta > 0 $, и по нему выбрать достаточно малое $ \sigma > 0 $  так, чтобы для достаточно больших $ n $ 
   \begin{align*}
    \left| F_n(x)-F(x) \right| < 2\eps + \frac{2\sigma^{2}}{\delta^{2}} < 3\eps.
   \end{align*} Сходимость $ \xi_n $ к $ \xi $ по распределению доказана.
 \end{itemize}
\end{proof}

\newpage
\section{Центральная предельная теорема.}

Предположим, что у нас есть большое количество слабо зависимых и примерно одинаково распределённых случайных величин:
\begin{align*}
 \xi_1, \xi_2, \xi_3, \ldots, \xi_n,
\end{align*} где $ n $ стремится к бесконечности. Например, эти величины являются результатами независимых повторений одного и того же эксперимента. Однако, в отличие от постановки закона больших чисел (параграф \ref{subsection:law_of_large_numbers}), нас интересует не стремление среднего результата к ожидаемому, а то, как распределена сумма результатов
\begin{align*}
 S_n = \xi_1 + \xi_2 + \ldots + \xi_n.
\end{align*}

\textit{Центральные предельные теоремы (ЦПТ)} утверждают, что при некоторых условиях распределение суммы $ S_n $ стремится к нормальному распределению. Разные формы ЦПТ в основном варьируют условия на распределения величин $ \xi_1, \xi_2, \ldots $ В самой простой форме (ЦПТ в форме Поля Лев\'{и}) величины $ \xi_n $ должны иметь вторые моменты, и должны быть одинаково распределены. В других формах ЦПТ (например, ЦПТ в форме Линденберга) условие на распределения величин более сложные, но более слабые.

\subsection{ЦПТ в форме Поля Лев\'{и}.}

Прежде, чем мы перейдём к ЦПТ в самой простой форме, нам потребуется техническая теорема, позволяющая получить равномерность в поточечной сходимости функций.

\begin{thm}[равномерность слабой сходимости к непрерывному распределению]
 \label{theorem:even_weak_convergence_to_continuous_distr_func}
 Пусть $ F \in C(\R)$ --- непрерывная функция распределения, и последовательность функций распределения $ F_1, F_2, \ldots $ сходится к $ F $ поточечно: $ F_n \to F $. Тогда эта сходимость равномерная: $ F_n \rightrightarrows F $.
\end{thm}
\begin{proof}[\normalfont\textsc{Доказательство}]
 Для большого $ m $ найдём точки $ t_j $ такие, что $ F(t_j) = \frac{j}{m} $, где $ j = 1, 2, \ldots, m - 1 $ (это можно сделать, потому что $ F $ непрерывна и стремится к $ 0 $ и $ 1 $ при $ x \to -\infty $ и $ x \to +\infty $ соответственно). Кроме того, положим $ t_0 = -\infty $ и $ t_m = +\infty $, и будем считать, что
 \begin{align*}
  F(t_0) = F_n(t_0) = 0, && F(t_m) = F_n(t_m) = 1,
 \end{align*} на самом деле подразумевая пределы функций $ F $ и $ F_n $ при $ t \to t_0 $ или $ t \to t_m $.

 Зафиксируем $ \eps > 0 $. В каждой точке $ t_j $ верна поточечная сходимость, поэтому при $ n > N $ верно
 \begin{align*}
  \left| F_n(t_j) - F(t_j) \right| < \eps / 2
 \end{align*} одновременно для всех $ t_j $.

 Рассмотрим любую точку $ t \in \R $: для неё существует единственное $ j \in \left\{ 1,\ldots,m \right\} $ такое, что $ t_{j-1} < t \leqslant t_j $. Тогда при $ n > N $
 \begin{align*}
  F_n(t) &\leqslant F_n(t_j) < F(t_j) + \frac{\eps}{2} < F(t) + \frac{\eps}{2} + \frac{1}{m},\\
  F_n(t) &\geqslant F_n(t_{j-1}) > F(t_{j-1}) - \frac{\eps}{2} > F(t) - \frac{\eps}{2} - \frac{1}{m}.
 \end{align*} Тогда при $ \frac{1}{m} < \frac{\eps}{2} $ получаем
 \begin{align*}
  \left| F_n(t)-F(t) \right| < \eps
 \end{align*} при $ n > N $ для всех $ t \in \R $. Равномерность доказана.
\end{proof}

\begin{thm}[ЦПТ в форме Поля Лев\'{и}]
 \label{theorem:central_limit_theorem_Levi}
 Пусть случайные величины~$ \xi_1, \xi_2, \ldots $  одинаково распределены и независимы, $ \mu = \E \xi_i $~--- их матожидание, $ \sigma^{2} = \D \xi_i > 0 $~--- их дисперсия. Пусть
 \begin{gather*}
  S_n = \xi_1 + \xi_2 + \ldots + \xi_n \text{ --- их частичная сумма, а}\\
  Z_n = \frac{S_n - \E S_n}{\sqrt{\D S_n}} = \frac{S_n - \mu n}{\sigma \sqrt n}\text{ --- та же частичная сумма,}
 \end{gather*} но несмещённая (её матожидание равно нулю) и нормированная (её дисперсия равна единице). Тогда $ Z_n $ слабо сходится к стандартному нормальному распределению~$ \Norm(0,1) $, причём сходимость равномерная по $ x\in\R $:
 \begin{align*}
  F_{Z_n}(x) \rightrightarrows \Phi(x).
 \end{align*} Здесь
 \begin{align*}
  \Phi(x) = \frac{1}{\sqrt{2\pi}} \int_{-\infty}^{x} e^{-t^{2} / 2}\,dt \text{ --- функция распределения }\Norm(0,1).
 \end{align*}
\end{thm}
\begin{remrk*}
 Поль Лев\'{и} (1886 - 1971) --- не тот, что и Л\'{е}ви (Беппо Леви, 1875 - 1961) из теоремы в теории меры о перестановке знака предела и интеграла Лебега.
\end{remrk*}
\begin{remrk*}
 Заключение ЦПТ можно сформулировать в терминах $ S_n $, а не $ Z_n $: ЦПТ утверждает, что распределение $ S_n $ стремится к нормальному распределению $ \Norm(n\mu, n\sigma^{2}) $.
\end{remrk*}
\begin{proof}[\normalfont\textsc{Доказательство теоремы \ref{theorem:central_limit_theorem_Levi}}]
 По предыдущей теореме \ref{theorem:even_weak_convergence_to_continuous_distr_func} нам достаточно лишь доказать поточечную сходимость $ F_{Z_n} \to \Phi $ (не заботясь о равномерности), а по теореме \ref{theorem:weak_convergence} о сходимости по распределению нам достаточно доказать, что характеристические функции $ \varphi_{Z_n}(t) $ поточечно сходятся к характеристической функции стандартного нормального распределения, то есть к функции $ t \mapsto e^{-t^{2} / 2} $ (см. пример \ref{example:char_func_of_normal_distribution}). Так и поступим.

 Запишем формулу Тейлора в окрестности точки $ u = 0 $ для характеристической функции несмещённого распределения величин $ \xi_i $:
 \begin{align*}
  \varphi(u) &:= \varphi_{\xi_i - \mu}(u) = 1 + \varphi_{\xi_i - \mu}'(0) \cdot u + \frac{1}{2}\varphi_{\xi_i - \mu}''(0) \cdot u^{2} + o(u^{2}) = \\
  &= 1 - i\cdot\E(\xi_i-\mu) \cdot u - \frac{\E(\xi_i - \mu)^{2}}{2} \cdot u^{2} + o(u^{2}) = \\
  &= 1 - \frac{\sigma^{2}u^{2}}{2} + o(u^{2}).
 \end{align*} 

 Пусть теперь $ t \in \R $ --- произвольная точка. Тогда при $ n \to \infty $
 \begin{align*}
  \varphi_{Z_n}(t) &= \varphi_{S_n - \mu n} \left( \frac{t}{\sigma\sqrt n} \right) = \prod_{k=1}^{n}\varphi_{\xi_k - \mu} \left( \frac{t}{\sigma \sqrt n} \right) = \left( \varphi \left( \frac{t}{\sigma\sqrt n} \right) \right)^{n} = \\
  &= \left( 1 - \frac{t^{2}}{2n} + o(1 / n) \right)^{n} = \exp \left[ n \log \left( 1 - \frac{t^{2}}{2n} + o(1 / n) \right) \right] = \\
  &= \exp \left[ n \left( - \frac{t^{2}}{2n} + o(1 / n) \right) \right] = \exp \left[ -\frac{t^{2}}{2} + o(1) \right] = e^{-t^{2} / 2} + o(1),
 \end{align*} что и требовалось доказать.
\end{proof}

\subsection{Следствия ЦПТ.}

Оказывается, интегральная теорема Муавра-Лапласа \ref{theorem:intergram_theorem_muavr_laplas}, которая тогда была оставлена без доказательства, является тривиальным следствием из ЦПТ.

\begin{proof}[\normalfont\textsc{Доказательство теоремы \ref{theorem:intergram_theorem_muavr_laplas}}]
 Пусть $ \xi_1, \xi_2, \ldots $ --- схема Бернулли с вероятностью успеха $ p \in (0,1) $, $ S_n = \xi_1 + \ldots + \xi_n $ --- число успехов среди $ n $ испытаний. Случайные величины $ \xi_1, \xi_2, \ldots $ независимы, одинаково распределены ($ P(\xi_i = 1) = p $ и $ P(\xi_i = 0) = 1 - p $), имеют матожидание $ \E\xi_i = p $ и дисперсию $ \D\xi_i = pq $. Тогда по центральной предельной теореме \ref{theorem:central_limit_theorem_Levi}
 \begin{align*}
  P \left[ \frac{S_n - np}{\sqrt{npq}} \leqslant x \right] \rightrightarrows \Phi(x).
 \end{align*} Или же
 \begin{align*}
  P \left[ a < \frac{S_n-np}{\sqrt{npq}} \leqslant b \right] \rightrightarrows \Phi(b) - \Phi(a).
 \end{align*}
\end{proof}

Тем же самым методом, которым мы доказали ЦПТ, то есть методом характеристический функций, можно доказать более широкий вариант теоремы \ref{theorem:poisson} Пуассона.

\begin{thm}[Пуассона]
 \label{theorem:poissonx}
 Пусть для каждого $ n \geqslant 1 $ есть набор из $ n $ бернуллиевских случайных величин
 \begin{align*}
  \xi_{n1} \sim \Bernoulli(p_{n1}), && \xi_{n2}\sim\Bernoulli(p_{n2}), && \ldots, && \xi_{nn} \sim \Bernoulli(p_{nn}),
 \end{align*} где $ p_{n1}, p_{n2}, \ldots, p_{nn}\in(0,1) $, которые к тому же независимы между собой. Их сумму обозначим
 \begin{align*}
  S_n = \xi_{n1} + \xi_{n 2} + \ldots + \xi_{n n}.
 \end{align*} Пусть
 \begin{align}
  \label{eq:theorem:poissonx:lim_of_max_of_p_n_k}
  \lim_{n \to \infty} \max_{k=1}^{n} p_{nk} = 0.
 \end{align} Наконец, пусть
 \begin{align}
  \label{eq:theorem:poissonx:lim_of_sum_of_p_n_k}
  \lim_{n \to \infty} \sum_{k=1}^{n} p_{nk} = \lambda,
 \end{align} где $ \lambda > 0 $ --- некоторое число. Тогда для каждого $ k \geqslant 0 $
 \begin{align}
  \label{eq:theorem:possionx:convergence}
  \lim_{n \to \infty} P(S_n = k) = \frac{e^{-\lambda}\lambda^{k}}{k!}.
 \end{align}
\end{thm}

\begin{remrk*}
 В стандартной версии теоремы Пуассона (теорема \ref{theorem:poisson}), вероятности успеха бернуллиевских величин в одном наборе были одинаковы:
 \begin{align*}
  p_{n 1} = p_{n 2} = \ldots = p_{n n} = p_n.
 \end{align*} Тогда условие \eqref{eq:theorem:poissonx:lim_of_sum_of_p_n_k} эквивалентно условию $ p_n \sim \lambda / n $, и из него автоматически следует условие \eqref{eq:theorem:poissonx:lim_of_max_of_p_n_k}.
\end{remrk*}

\begin{remrk}[характеристическая функция распределения Пуассона]
 \label{remark:poisson_char_func}
 В доказательстве теоремы \ref{theorem:poissonx} нам потребуется характеристическая функция распределения Пуассона, поэтому найдём её заранее. Пусть $ \xi \sim \Poisson(x) $. Тогда
 \begin{align*}
  \varphi_\xi(t) &= \E e^{it\xi} = \sum_{k=0}^{\infty}e^{itk} \cdot \frac{e^{-\lambda}\lambda^{k}}{k!} = e^{-\lambda} \sum_{k=0}^{\infty} \frac{(e^{it}\lambda)^{k}}{k!} = e^{-\lambda} \cdot e^{e^{it}\lambda} = \exp \left[ (e^{it}-1)\lambda \right].
 \end{align*}
\end{remrk}

\begin{proof}[\normalfont\textsc{Доказательство теоремы \ref{theorem:poissonx}}]
 Условие \eqref{eq:theorem:possionx:convergence} означает слабую сходимость величин $ S_n $ к распределению Пуассона $ \Poisson(\lambda) $ (так как все величины дискретные, то можно проигнорировать оговорку о точках непрерывности: скажем, рассмотрим $ k + 1 / 2 $). Как мы и поступили в доказательстве ЦПТ, пользуясь теоремой \ref{theorem:weak_convergence}, докажем, что характеристические функции сумм $ \varphi_{S_n}(t) $ поточечно сходятся к характеристической функции распределению Пуассона, которую мы заранее вычислили в замечании \ref{remark:poisson_char_func}.

 Запишем характеристическую функцию бернуллиевской случайной величины $ \xi_{n k} $:
 \begin{align*}
  \varphi_{\xi_{n k}}(t) = \E e^{it \xi_{n k}} = (1 - p_{n k}) + e^{it} p_{n k} = 1 + (e^{it} - 1)p_{n k}.
 \end{align*} Тогда характеристическая функция суммы равна
 \begin{align*}
  \varphi_{S_n}(t) = \prod_{k=1}^{n} \varphi_{\xi_{n k}}(t) = \prod_{k=1}^{n} (1 + (e^{it} - 1)p_{n k}).
 \end{align*} Проверим, что она стремится к характеристической функции распределения Пуассона $ t \mapsto \exp \left[ (e^{it}-1)\lambda \right] $, что равносильно
 \begin{align*}
  \log \varphi_{S_n}(t) \to (e^{it} - 1)\lambda.
 \end{align*} Прологарифмируем:
 \begin{align*}
  \log \varphi_{S_n}(t) = \sum_{k=1}^{n}\log (1 + (e^{it} - 1)p_{nk}).
 \end{align*} Но так как по условию \eqref{eq:theorem:poissonx:lim_of_max_of_p_n_k} $ p_{n k} \to 0 $ при $ n \to \infty $, то по формуле Тейлора для логарифма
 \begin{align*}
  \log(1 + (e^{it} - 1)p_{n k}) = (e^{it} - 1)p_{n k} + \OO(p_{n k}^{2}).
 \end{align*} Тогда по условию \eqref{eq:theorem:poissonx:lim_of_sum_of_p_n_k}
 \begin{align*}
  \log \varphi_{S_n}(t) &= \sum_{k=1}^{n} \left( (e^{it}-1)p_{nk} +\OO(p_{nk}^{2}) \right) = (e^{it} - 1) \sum_{k=1}^{n}p_{nk} + \sum_{k = 1}^{n} \OO(p_{nk} ^{ 2}) = \\
  &= (e^{it}-1)\lambda + o(1) + \OO \left[ \sum_{k=1}^{n}p_{n k}^{2} \right].
 \end{align*} Осталось оценить остаток:
 \begin{align*}
  \sum_{k=1}^{n}p_{nk}^{2} \leqslant \sum_{k=1}^{n} p_{nk} \cdot \max_{k=1}^{n} p_{n k} = (\lambda + o(1)) \cdot o(1) = o(1).
 \end{align*} Таким образом, мы получили
 \begin{align*}
  \lim_{n \to \infty} \log \varphi_{S_n}(t) = (e^{it}-1)\lambda,
 \end{align*} что завершает доказательство.
\end{proof}

\subsection{Обобщения классической ЦПТ.}

\newcommand{\Lind}{\ensuremath \mathrm{Lind}}

\begin{thm}[%
 ЦПТ в форме Линденберга]
 \label{theorem:central_limit_theorem_Linderberg}
 Пусть случайные величины $ \xi_1, \xi_2, \ldots $  независимы, имеют матожидание $ \mu_k = \E \xi_k $ и конечную дисперсию $\sigma_k^{2} = \D \xi_k > 0$. Обозначим $ S_n = \xi_1 + \ldots + \xi_n $, и
 \begin{align*}
  D_n^{2} := \D S_n = \sum_{k=1}^{n} \sigma_k^{2}.
 \end{align*} Для числа $ \eps > 0 $ обозначим
 \begin{align*}
  \Lind(\eps, n) := \frac{1}{D_n^{2}} \sum_{k=1}^{n} \E \left[ (\xi_k-\mu_k)^{2} \cdot \Ind_{\left\{ \left| \xi_k \right| \geqslant \eps \cdot D_n \right\}} \right].
 \end{align*} Предположим, что для случайных величин $ \xi_1,\xi_2,\ldots $ выполняется \emph{условие Линденберга:} для любого $ \eps > 0 $ верно
 \begin{align}
  \label{eq:lindenberg_condition}
  \lim_{n \to \infty} \Lind(\eps, n) = 0.
 \end{align} Тогда верно заключение ЦПТ:
 \begin{align*}
  P \left[ \frac{S_n - \E S_n}{\sqrt{\D S_n}} \leqslant x \right] \rightrightarrows \Phi(x).
 \end{align*}
\end{thm}

Доказывать теорему \ref{theorem:central_limit_theorem_Linderberg} мы не будем, потому что там слишком много счёта.

\begin{remrk}
 Покажем, что классическая версия ЦПТ (теорема \ref{theorem:central_limit_theorem_Levi}) является частным случаем ЦПТ в форме Линденберга. Для этого нам нужно проверить условие Линденберга \eqref{eq:lindenberg_condition} для независимых одинаково распределённых случайных величин с конечной дисперсией.

 Пусть $ \xi_1, \xi_2, \ldots $ независимы, одинаково распределены, $ \mu = \E \xi_i $ и $ \sigma^{2} = \D \xi_i > 0 $. Тогда
 \begin{align*}
  D_n^{2} = \D S_n = n\sigma^{2}.
 \end{align*} Для любого $ \eps > 0 $
 \begin{align*}
  \Lind(\eps,n) &= \frac{1}{n\sigma^{2}} \cdot \sum_{k=1}^{n}\E \left[ (\xi_k-\mu)^{2} \cdot \Ind_{\left\{ \left| \xi_k \right| \geqslant \eps \cdot \sigma \sqrt n \right\}} \right] = \frac{1}{\sigma^{2}} \E \left[ (\xi_1 - \mu)^{2} \cdot \Ind_{\left\{ \left| \xi_1 \right| \geqslant \eps\sigma \sqrt n \right\}} \right].
 \end{align*} Тогда нам нужно показать, что
 \begin{align*}
  \int\limits_{\left\{ \left| \xi_1 \right| \geqslant \eps \sigma \sqrt n \right\}}  (\xi_1-\mu)^{2}\,dP \to 0.
 \end{align*} Но это верно по абсолютной непрерывности интеграла Лебега, так как по неравенству Маркова \eqref{equation:markov_inequality}
 \begin{align*}
  P(\left| \xi_1 \right| \geqslant \eps \sigma \sqrt n) \leqslant \frac{\E \left| \xi_1 \right|}{\eps \sigma \sqrt n} \to 0.
 \end{align*}
\end{remrk}

\begin{thm}[%
 ЦПТ в форме Ляпунова]
 \label{theorem:central_limit_theorem_Lyapunov}
 Пусть cлучайные величины $ \xi_1, \xi_2, \ldots $ независимы, имеют матожидание $ \mu_k = \E \xi_k $ и конечную дисперсию $ \sigma_k^{2} = \D \xi_k > 0 $. Предположим, что для некоторого $ \delta > 0 $ выполнено
 \begin{align*}
  L(\delta, n) = \frac{1}{D_n^{2+\delta}} \sum_{k=1}^{n} \E \left| \xi_k - a_k \right|^{2 + \delta} \to 0,
 \end{align*} где $ D_n $ и $ S_n $ --- как в теореме \ref{theorem:central_limit_theorem_Linderberg}. Тогда верно заключение ЦПТ:
 \begin{align*}
  P \left[ \frac{S_n - \E S_n}{\sqrt{\D S_n}} \leqslant x \right] \rightrightarrows \Phi(x).
 \end{align*}
\end{thm}
\begin{proof}[\normalfont\textsc{Вывод из ЦПТ в форме Линденберга}]
 Достаточно проверить условие Линденберга \eqref{eq:lindenberg_condition}.

 Заметим неравенство
 \begin{align*}
  \Ind_{\left\{ \left| x \right| \geqslant \eps D_n \right\}} \leqslant \left( \frac{\left| x \right|}{\eps D_n} \right)^{\delta}.
 \end{align*} Подставим его:
 \begin{align*}
  \Lind(\eps,n) &= \frac{1}{D_n^{2}} \sum_{k=1}^{n}\E \left[ (\xi_k-\mu_k)^{2} \cdot \Ind_{\left\{ \left| \xi_k \right| \geqslant \eps D_n \right\}} \right] \leqslant\\
  &\leqslant \frac{1}{D_n^{2}} \sum_{k=1}^{n}\E \left[ \frac{\left| \xi_k-\mu_k \right|^{2 + \delta}}{\eps^{\delta} D_n^{\delta}} \right] = \frac{L(\delta,n)}{\eps^{\delta}} \to 0.
 \end{align*}
\end{proof}

\subsection{Скорость сходимости в ЦПТ.}

Приведём без доказательства известные науке оценки на скорость сходимости в центральных предельных теоремах.

\begin{thm}[оценка для ЦПТ в форме Ляпунова]
 В условиях ЦПТ в форме Ляпунова (теорема \ref{theorem:central_limit_theorem_Lyapunov}) при $ 0 < \delta \leqslant 1 $ выполнено
 \begin{align*}
  \left| P \left[ \frac{S_n-\E S_n}{\sqrt{\D S_n}} \leqslant x\right] - \Phi(x) \right| \leqslant C_\delta \cdot L(\delta, n),
 \end{align*} где $ C_\delta $ зависит лишь от $ \delta $.
\end{thm}
\begin{remrk}
 Для одинаково распределённых случайных величин верно
 \begin{align*}
  L(\delta,n) &= \frac{1}{(n\sigma^{2})^{1 + \delta / 2}} \cdot n \E \left| \xi_1 - \mu \right|^{2 + \delta} = \frac{\E \left| \xi_1 - \mu \right|^{2 + \delta}}{n^{\delta / 2} \cdot \sigma^{2 + \delta}}.
 \end{align*} Поэтому, здесь верно
 \begin{align}
  \label{eq:clt_velocity_ineq_2_plus_delta}
  \left| P \left[ \frac{S_n- n\mu}{\sigma \sqrt{n}} \leqslant x \right] - \Phi(x) \right| \leqslant C_\delta \cdot \frac{\E \left| \xi_1 - \mu \right|^{2 + \delta}}{n^{\delta / 2} \cdot \sigma^{2 + \delta}}.
 \end{align}
\end{remrk}

Неравенство \eqref{eq:clt_velocity_ineq_2_plus_delta} при $ \delta = 1 $ называется \textit{теоремой Берри-Эссеена}.

\begin{thm}[Берри-Эссеена]
 Пусть случайные величины $ \xi_1, \xi_2, \ldots $ независимы и одинаково распределены.  Тогда
 \begin{align*}
  \left| P \left[ \frac{S_n - na}{\sigma \sqrt n} \leqslant x \right] - \Phi(x) \right| \leqslant C \cdot \frac{\E \left| \xi_1 - a \right|^{3}}{\sqrt n \cdot \sigma^{3}}.
 \end{align*}
\end{thm}

Что известно про константы?

\begin{itemize}
 \item Эссеен (1956): $ C \geqslant \frac{3 + \sqrt{10}}{6 \sqrt{2\pi}} \approx 0.40973 $.
 \item Шевцова (2014): $ C \leqslant 0.469 $.
 \item 2018: для схем Бернулли $ C \leqslant 0.4009 $. До нижней оценки не хватает $ 0.002 $!
 \item Для схем Бернулли с $ p = 1/2 $: $ C = \frac{1}{\sqrt{2\pi}} $.

 \item Для общего случая (распределения могут быть разные)  известно $ C_1 \leqslant 0.5583 $.
\end{itemize}

Чисто для справки приведём ещё два результата.

\begin{thm}[Хартмана-Винтнера, закон повторного логарифма]
 \label{theorem:hartman_vinter}
 Пусть случайные величины $ \xi_1, \xi_2, \ldots $ независимы и одинаково распределены, $ \E \xi_i = 0 $, $ \sigma^{2} = \D \xi_i > 0 $. Тогда
 \begin{align*}
  \varlimsup_{n \to \infty} \frac{S_n}{\sqrt{2n \log\log n}} = \sigma,&&
  \varliminf_{n \to \infty} \frac{S_n}{\sqrt{2n \log\log n}} = -\sigma.
 \end{align*}
\end{thm}

\begin{thm}[Штрассен]
 В условиях теоремы \ref{theorem:hartman_vinter} любая точка из $ [-\sigma, \sigma] $ является предельной точкой последовательности
 \begin{align*}
  \left\{\frac{S_n}{\sqrt{2 n \log \log n}} \right\}_{n=1}^{\infty}.
 \end{align*}
\end{thm}

\end{document}
