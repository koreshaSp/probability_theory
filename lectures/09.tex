% 2023.04.11 lecture 09
\documentclass[../main.tex]{subfiles}
\begin{document}

\newpage
\section{Сходимость по распределению.}

Напомним, что \textit{сходимостью по распределению} (или \textit{слабой сходимостью}) случайных величин $ \xi_n $ к величине $ \xi $ называется сходимость функций распределения $ F_{\xi_n}(x) \to F_\xi(x) $ в точках непрерывности $ x $ функции $ F_\xi $.

В этом параграфе основным результатом будет большая теорема о равносильных формулировках сходимости по распределению. Начнём с подготовительных утверждений.

\begin{remrk}
 Пусть cлучайные величины $ \xi $ и $ \eta $ независимы, и $ \eta $ непрерывна. Тогда $ \xi + \eta $ непрерывна.
\end{remrk}
\begin{proof}[\normalfont\textsc{Доказательство}]
 Докажем непрерывность в точке $ a \in \R $, вычислив свёртку распределений:
 \begin{align*}
  P_{\xi+\eta}(\left\{ a \right\}) = P_\xi \ast P_\eta(\left\{ a \right\}) = \int_{\R} P_{\eta}(\left\{ a-x \right\}) dP_\xi(x) = \int_{\R} 0\,dP_\xi = 0, 
 \end{align*} где $ P_\eta(\left\{ y \right\}) = 0 $ по непрерывности $ \eta $.
\end{proof}

Удивительно, что непрерывность второй случайной величины $ \xi $ здесь не требуется.

\begin{remrk}
 \label{remark:open_set_partitioned_into_cells_with_countable_bad_points}
 Пусть $ U \subset \R $ --- открытое множество, а $ D $  --- не более, чем счётное множество <<плохих>> точек. Тогда $ U $ можно разбить на счётное число ячеек
 \begin{align*}
  U = \bigsqcup_{n=1}^{\infty} \left(a_n, b_n\right],
 \end{align*} концы которых <<хорошие>>: $ a_n, b_n \notin D $.
\end{remrk}
\begin{proof}[\normalfont\textsc{Доказательство}]
 Берём ячейки с шагом $ 1 $. Какие-то попали целиком, а какие-то не влезли. Потом берём шаг $ 1 / 2 $ и так далее.

 Если заботимся о концах: делим не совсем пополам, а примерно (так чтобы точка деления не лежала в $ D $).
\end{proof}

\begin{remrk}
 \label{remrk:lim_f_n_using_f_n_a_minnus_f_n_b}
 Пусть $ F_n $ и $ F $ --- функции распределения такие, что для любых $ a, b \in \R $:
 \begin{align*}
  \lim_{n \to \infty} (F_n(b) - F_n(a)) = F(b)  - F(a).
 \end{align*} Тогда для любого $ x \in \R $
 \begin{align*}
  \lim_{n \to \infty} F_n(x) = F(x).
 \end{align*}
\end{remrk}
\begin{proof}
 Возьмём $ x \in\R $. Тогда для любой точки $ a < x $ выполнено
 \begin{align*}
  \lim_{n \to \infty} (F_n(x) - F(x) - (F_n(a)-F(a))) = 0.
 \end{align*} Так как 
\end{proof}
\begin{proof}[\normalfont\textsc{Доказательство}]
 От функций $ F_n $ и $ F $ нужна лишь ограниченность отрезком $ [0,1] $ и существование пределов на бесконечностях, равных $ 0 $ (на $ -\infty $) и $ 1 $ (на $ +\infty $).

 Возьмём любой $ \eps > 0 $. Так как $ F(a) \to 0 $ при $ a \to -\infty $, то существует $ a \in \R $ такое, что $ F(a) < \eps $. Аналогично, так как $ F(b) \to 1 $ при $ b \to +\infty $, то существует $ b \in \R $ такое, что $ F(b) > 1 - \eps $. Зафиксируем эти $ a $, $ b $ раз и навсегда.

 Так как $ F(b) - F(a) > 1 - 2\eps $, то для $ \eps $ существует $ N $ такое, что для всех $ n > N $ выполнено
 \begin{align*}
  \left|(F_n(b) - F_n(a)) - (F(b)-F(a)) \right| < \eps,
 \end{align*} из чего следует $ F_n(b) - F_n(a) > 1 - 3\eps $ и $ F_n(a) < 3\eps $ при $ n > N $.

 Рассмотрим любую точку $ x \in \R $. Тогда для $ \eps $ существует $ N' $ такое, что при всех $ n > N' $ выполнено
 \begin{align*}
  \left|(F_n(x) - F_n(a)) - (F(x)-F(a)) \right| < \eps.
 \end{align*} Следовательно,
 \begin{align*}
  \left|F_n(x) - F(x) \right| \leqslant \left| (F_n(x) - F_n(a)) - (F(x) - F(a)) \right| + F_n(a) + F(a) < 5\eps
 \end{align*} при $ n > \max(N,N') $.
\end{proof}

\begin{prop}
 У функции распределения не более, чем счётно много точек разрыва.
\end{prop}
\begin{proof}[\normalfont\textsc{Доказательство}]
 Уже было пояснено ранее, в доказательстве следствия \ref{corollary:char_func_defines_distribution}. Другое доказательство: в каждом интервале разрыва можно выбрать рациональное число.
\end{proof}

\begin{df}
 Множество $ B \subset \R $ \textit{регулярно} относительно распределения $ P_\xi $, если мера его границы нулевая: $ P_\xi(\mathrm{Cl}\,B \setminus \mathrm{Int}\,B) = 0 $.

 В этом случае $ P(\xi \in A) = P(\xi \in B) $, если $ \mathrm{Int}\,B \subset A \subset \mathrm{Cl}\,B $.

\end{df}

Наконец-то мы готовы приступить к основной теореме параграфа.

\begin{thm}[о сходимости по распределению]
 \label{theorem:weak_convergence}
 Пусть $ \xi, \xi_1, \xi_2, \ldots $ --- последовательность случайных величин, $ F, F_1, F_2, \ldots $ --- их функции распределения, и $ \varphi, \varphi_1, \varphi_2, \ldots $ --- их характеристические функции соответственно. Следующие условия равносильны.
 \begin{enumerate}
  \item \label{i1:weak_convergence} $ \xi_n $ сходится к $ \xi $ по распределению:
   \begin{align*}
    \lim_{n \to \infty} F_n(x) = F(x)
   \end{align*} для любой точки непрерывности $ x $ функции распределения $ F $.
  \item \label{i2:weak_convergence} Для любого открытого множества $ U \subset \R $
   \begin{align*}
    \liminf_{n \to \infty} P(\xi_n \in U) \geqslant P(\xi \in U).
   \end{align*}
  \item \label{i3:weak_convergence} Для любого замкнутого множества $ A \subset \R $
   \begin{align*}
    \limsup_{n \to \infty} P(\xi_n \in A) \leqslant P(\xi \in A).
   \end{align*}
  \item \label{i4:weak_convergence} Для любого регулярного (относительно $ P_\xi $) борелевского множества $ B \subset \R$
   \begin{align*}
    \lim_{n \to \infty} P(\xi_n \in B) = P(\xi \in B).
   \end{align*}
  \item \label{i5:weak_convergence} Для любого регулярного (относительно $ P_\xi $) борелевского множества $ B \subset\R $
   \begin{align*}
    \lim_{n \to \infty} \E \Ind_B(\xi_n) = \E \Ind_B(\xi).
   \end{align*}
  \item \label{i6:weak_convergence} Для любой непрерывной ограниченной вещественнозначной функции $ f \in C(\R) $
   \begin{align*}
    \lim_{n \to \infty} \E f(\xi_n) = \E f(\xi).
   \end{align*}
  \item \label{i7:weak_convergence} Характеристические функции $ \varphi_n(t) $ сходятся к характеристической функции $ \varphi(t) $ всюду в $ \R $:
   \begin{align*}
    \lim_{n \to \infty} \varphi_n(t) = \varphi(t),
   \end{align*} для любой точки $ t \in \R $.
 \end{enumerate}
\end{thm}
\begin{proof}[\normalfont\textsc{Доказательство}]
 Докажем сначала простые следствия и эквивалентности.
 \begin{itemize}
  \item Условие \ref{i2:weak_convergence} равносильно условию \ref{i3:weak_convergence}. Здесь достаточно перейти к дополнению: для открытого $ U \subset \R$ взять замкнутое $ A = \R \setminus U $. Тогда $ P(\xi_n \in A) = 1 - P(\xi_n \in U) $ для всех $ n $, и
   \begin{align*}
    \limsup_{n \to \infty} P(\xi_n \in A) = 1 - \liminf_{n \to \infty} P(\xi_n \in U).
   \end{align*} Тогда если
   \begin{align*}
    \liminf_{n \to \infty} P(\xi_n \in U) \geqslant P(\xi \in U),
   \end{align*} то
   \begin{align*}
    \limsup_{n \to \infty} P(\xi_n\in A) \leqslant 1 - P(\xi \in U) = P(\xi\in A).
   \end{align*} Наоборот, если
   \begin{align*}
    \limsup_{n \to \infty} P(\xi_n \in A) \leqslant P(\xi \in A),
   \end{align*} то
   \begin{align*}
    \liminf_{n \to \infty} P(\xi_n \in U) \geqslant 1 - P(\xi \in A) = P(\xi \in U).
   \end{align*}

  \item Из условий \ref{i2:weak_convergence}, \ref{i3:weak_convergence} следует условие \ref{i4:weak_convergence}. Для регулярного борелевского множества $ B \subset \R $ возьмём открытое множество $ U = \mathrm{Int}\,B $ и замкнутое множество $ A =\mathrm{Cl}\,B $. Пользуясь условиями \ref{i2:weak_convergence}, \ref{i3:weak_convergence}, запишем цепочку неравенств:
   \begin{align*}
    P(\xi \in U) &\leqslant \liminf_{n \to \infty} P(\xi_n \in U) \leqslant \\
    &\leqslant \liminf_{n \to \infty} P(\xi_n \in B) \leqslant \limsup_{n \to \infty} P(\xi_n \in B) \leqslant \\
    &\leqslant \limsup_{n \to \infty} P(\xi_n \in A) \leqslant P(\xi \in A).
   \end{align*} Но так как множество $ B $ регулярное, то $ P(\xi\in U) = P(\xi\in B) = P(\xi\in A) $, и все неравенства выше --- равенства. Тогда верхний предел равен нижнему, и мы получаем
   \begin{align*}
    \lim_{n \to \infty} P(\xi_n \in B) = P(\xi\in B).
   \end{align*}

  \item Условие \ref{i5:weak_convergence} --- это то же условие \ref{i4:weak_convergence}, но по-другому записанное:
   \begin{align*}
    \E \Ind_B (\xi) = P(\xi \in B), && \E \Ind_B(\xi_n) = P(\xi_n \in B).
   \end{align*}

  \item Из условия \ref{i6:weak_convergence} следует условие \ref{i7:weak_convergence}:
   \begin{align*}
    \varphi_n(t) &= \E e^{it\xi_n} = \E \cos(t\xi_n) + i \cdot \E\sin(t\xi_n) \to \\
    &\to \E \cos(t\xi) + i \cdot \E\sin(t\xi) = \E e^{it\xi} = \varphi(t).
   \end{align*} $ x \mapsto \cos(tx) $ и $ x \mapsto \sin(tx) $ --- те самые непрерывные ограниченные вещественнозначные функции.
 \end{itemize}

 Осталось доказать три сложных следования.

 \begin{itemize}
  \item Из условия \ref{i1:weak_convergence} следует условие \ref{i2:weak_convergence}. Обозначим за $ D $ не более чем счётное множество точек разрыва функции распределения $ F $. Пользуясь замечанием  \ref{remark:open_set_partitioned_into_cells_with_countable_bad_points}, разобьём открытое множество $ U $ на счётное число ячеек:
   \begin{align*}
    U = \bigsqcup_{k=1}^{\infty} \left(a_k, b_k\right],
   \end{align*} концы которых являются точками непрерывности $ F $: $ a_n, b_n \notin D $.

   Для каждого конечного  $ m \geqslant 1 $  имеем
   \begin{align*}
    P(\xi_n \in U) &\geqslant P \left[ \xi_n\in \bigsqcup_{k=1}^{m} \left(a_k, b_k\right]   \right] = \\
    &= \sum_{k=1}^{m} P(\xi_n \in \left(a_k, b_k\right]  ) = \sum_{k=1}^{m} (F_n(b_k) - F_n(a_k)).
   \end{align*} Пририсуем пределы с обеих сторон:
   \begin{align*}
    \liminf_{n \to \infty} P(\xi_n \in U) \geqslant \lim_{n \to \infty} \sum_{k=1}^{m}(F_n(b_k) - F_n(a_k)) = \sum_{k=1}^{m} F(b_k)-F(a_k),
   \end{align*} где последнее равенство верно из-за сходимости по распределению и из-за того, что $ a_k,b_k \notin D $. Так как это верно для любого $ m $, то
   \begin{align*}
    \liminf_{n \to \infty} P(\xi_n\in U) \geqslant \sum_{k=1}^{\infty}F(b_k)-F(a_k) = P \left( \xi \in \bigsqcup_{k=1}^{\infty} \left(a_k, b_k\right]   \right) = P(\xi\in U).
   \end{align*}

  \item Из условия \ref{i5:weak_convergence} следует условие \ref{i6:weak_convergence}. Идея доказательства естественная: приблизить непрерывную ограниченную функцию простой функцией --- линейной комбинацией индикаторных функцией, множества которых являются борелевскими и регулярными относительно $ P_\xi $. Однако само доказательство длинновато из-за технических деталей.

   Пусть $ f \in C(\R)$ --- непрерывная ограниченная вещественнозначная функция. Рассмотрим следующее множество плохих точек:
   \begin{align*}
    D = \left\{ x \in \R \mid P(f(\xi) = x) > 0 \right\}.
   \end{align*} Множество $ D $ не более, чем счётно, так как
   \begin{align*}
    \sum_{x \in D} P(f(\xi) = x) \leqslant 1.
   \end{align*}

   Пусть функция $ f $ ограничена числом $ M > 0 $:  $ \left| f \right| < M $, причём $ M \notin D $ и $ -M \notin D $ (если всё-таки $ M $ или $ -M $ попало в $ D $, то увеличим $ M $ чуть-чуть). Разобьем отрезок $ [-M,M] $ на $ m $ маленьких отрезочков с разрезами по точкам
   \begin{align*}
    -M = t_0,\;t_1,\;t_2,\;\ldots,\;t_{m-1},\;t_m=M
   \end{align*} так, чтобы $ t_j \notin D $ для всех $ 0 \leqslant j \leqslant m $, и чтобы длины отрезков были малы:
   \begin{align*}
    t_j - t_{j-1} < \eps = \frac{2M}{m}
   \end{align*} для всех $ 1 \leqslant j \leqslant m $.
   \begin{figure}[ht]
    \centering
    \incfig[0.8]{7_points_theorem_cut_m}
    \caption{Разбиение отрезка $[-M, M]$ на малые отрезки с хорошими концами.}
    \label{fig:7_points_theorem_cut_m}
   \end{figure}

   Разобьем вещественную прямую на следующие $ m $ множеств:
   \begin{align*}
    \R = \bigsqcup_{j=1}^{m} B_j, &&
    B_j=\left\{x\in\R\mid t_{j-1} < f(x) \leqslant t_j\right\}.
   \end{align*} Множества $ B_j $ покрывают всю вещественную прямую, так как $ f $ ограничена числом $ M $. При этом, множества $ B_j $ борелевские как непрерывные прообразы борелевских множеств.

   Проверим, что каждое множество $ B_j $ регулярное относительно распределения $ P_\xi $. Для этого рассмотрим множества
   \begin{align*}
    U_j &= \left\{ x \in \R \mid t_{j-1} < f(x) < t_j \right\},\\
    A_j &= \left\{ x \in \R \mid t_{j-1} \leqslant f(x) \leqslant t_j  \right\}.
   \end{align*} Множество $ U_j $ открыто как непрерывный прообраз открытого, а множество $ A_j $ замкнуто как непрерывный прообраз замкнутого. Поэтому, верна цепочка включений
   \begin{align*}
    U_j \subset \mathrm{Int}\,B_j \subset B_j \subset \mathrm{Cl}\,B_j \subset A_j.
   \end{align*} Тогда
   \begin{align*}
    P_\xi(\mathrm{Cl}\,B_j \setminus \mathrm{Int}\,B_j) \leqslant P_\xi(A_j \setminus U_j) = P(f(\xi) \in \left\{ t_{j-1},t_j \right\}) = 0,
   \end{align*} так как $ t_{j-1},t_j \notin D $. Регулярность $ B_j $ проверена.

   Построим следующую простую функцию
   \begin{align*}
    g(x) = \sum_{k=1}^{m} t_{j-1}\Ind_{B_j}(x),
   \end{align*} приближающую $ f $. По её построению верно
   \begin{align*}
    \left| f(x)-g(x) \right| < \eps
   \end{align*} всюду в $ \R $. Следовательно,
   \begin{align*}
    \E \left| f(\xi_n)-g(\xi_n) \right| < \eps, && \E \left| f(\xi)-g(\xi) \right| < \eps.
   \end{align*}

   Раз все $ B_j $ регулярные и борелевские, то по условию \ref{i5:weak_convergence}
   \begin{align*}
    \lim_{n \to \infty} \E \Ind_{B_j}(\xi_n) = \E \Ind_{B_j}(\xi)
   \end{align*} для каждого $ j $. По линейности математического ожидания
   \begin{align*}
    \lim_{n \to \infty} \E g(\xi_n) = \E g(\xi).
   \end{align*}

   Теперь применим $ 3\eps $-приём:
   \begin{align*}
    &\left| \E f(\xi_n) - \E f(\xi) \right| \leqslant \\
    \leqslant\;&\left| \E f(\xi_n) - \E g(\xi_n) \right| + \left| \E g(\xi_n) - \E g(\xi) \right| + \left| \E g(\xi) - \E f(\xi) \right| < 3\eps
   \end{align*} при достаточно больших $ n $. Устремляя $ m \to \infty $  ($ \eps = \frac{2M}{m} \to 0 $), получаем
   \begin{align*}
    \lim_{n \to \infty} \E f(\xi_n)=\E f(\xi).
   \end{align*}
 \end{itemize}

 Осталось одно, самое сложное следование.

 \begin{itemize}
  \item Из условия \ref{i7:weak_convergence} следует условие \ref{i1:weak_convergence}. 
 \end{itemize}

 $ 7 \implies 1 $. Воспользуемся формулой обращения:
 \begin{align*}
  P(\xi \in (a,b]) = \lim_{T \to +\infty}  \frac{1}{2\pi} \int\limits_{-T}^{T} \frac{e^{-iat} - e^{-ibt}}{it} \varphi_\xi(t)\,dt,
 \end{align*} если $ P(\xi = a) = P(\xi = b) = 0 $. Здесь проблема в том, что интеграл не настоящий, а в смысле главного значения. Решать проблему будем так: изменим случайную величину, так чтобы не сильно поменялись значения, но характеристическая функция улучшилась бы. Возьмём случайную величину $ \eta $, не зависящую от всех $ \xi, \xi_1, \xi_2, \ldots $ такую, что $ \eta \sim \Norm(0, \sigma^{2}) $ (формально это очень занудно). Тогда мы знаем, что
 \begin{align*}
  \varphi_{\xi_n + \eta}(t) = \varphi_{\xi_n}(t) \cdot \varphi_\eta(t) = \varphi_{\xi_n}(t) e^{-\sigma^{2}t^{2} / 2} \to \varphi_\xi(t) e^{-\sigma^{2}t^{2} / 2} = \varphi_{\xi+\eta}(t)
 \end{align*} поточечно. Кроме того, случайные величины $ \xi_n + \eta $ и   $ \xi + \eta $ непрерывны. Поэтому не нужно оговорка. Тогда
 \begin{align*}
  P(\xi_n + \eta \in [a,b]) = \lim_{T \to +\infty}  \frac{1}{2\pi} \int\limits_{-T}^T \underbrace{\frac{e^{-iat} - e^{-ibt}}{it}}_{|\cdot| \leqslant const} \underbrace{\varphi_{\xi_n}(t)}_{|\cdot| \leqslant 1} e^{-\sigma^{2}t^{2} / 2} \,dt
 \end{align*} Но интеграл сходится, так как $\int_{-\infty}^{+\infty} e^{-\sigma^2t^2/2}\,dt$ сходится, а остальные сомножители под интегралом ограничены! Поэтому можно убрать предел:
 \begin{align*}
  = \frac{1}{2\pi} \int\limits_{\R} \frac{e^{-iat} - e^{-ibt}}{it} e^{-\sigma^{2}t^{2} / 2}\varphi_{\xi_n}(t)\,dt
 \end{align*} Устремим $ n \to \infty $. По теореме Лебега с мажорантой $const \cdot e^{\sigma^2t^2/2}$ переходим к пределу под интегралом:  
 \begin{align*}
  \to \frac{1}{2\pi} \int\limits_{\R} \frac{e^{-iat}-e^{-ibt}}{it}  e^{-\sigma^{2}t^{2} / 2}\varphi_{\xi}(t)\,dt
 \end{align*} Но
 \begin{align*}
  = P(\xi+ \eta \in (a,b]).
 \end{align*} Пусть $ G_n $ и $ G $ --- функции распределения для $ \xi_n + \eta $ и $ \xi + \eta $ соответственно. Тогда мы только что показали, что
 \begin{align*}
  G_n(b) - G_n(a) \to G(b) - G(a)
 \end{align*} для любых $ a,b\in\R $. По замечанию \ref{remrk:lim_f_n_using_f_n_a_minnus_f_n_b} $ G_n(x) \to G(x) $ для любого $ x \in \R $.

 Осталось перейти от $ G $ к $ F $. Пусть $ x $ --- точка непрерывности $ F $. Зафиксируем $ \eps  > 0 $. Выберем $ \delta > 0 $ так, что 
 \begin{align*}
  \left| F(x \pm 2\delta) - F(x) \right| < \eps.
 \end{align*}
 \begin{align*}
  \left\{ \xi+\eta \leqslant x  + \delta \right\} \subset \left\{ \xi \leqslant x + 2\delta \right\} \cup \left\{ \left| \eta \right| \geqslant \delta \right\}.
 \end{align*} Запишем неравенство на вероятности:
 \begin{align*}
  G(x+\delta) \leqslant F(x+2\delta) + P(\left| \eta \right| \geqslant \delta) \leqslant F(x+2\delta) + \frac{\D \eta}{\delta^{2}} = F(x + 2\delta) + \frac{\sigma^2}{\delta^2}. 
 \end{align*} 
 Тогда при больших $n$
 \begin{align*}
  F_n(x) = P(\xi_n < x) &\leqslant P(\xi_n + \eta \leqslant x + \delta) + P(|\eta| \geqslant \delta) \\
  &\leqslant G_n(x+\delta) + \frac{\sigma^{2}}{\delta^{2}} \\
  &< G(x+\delta) + \eps + \frac{\sigma^{2}}{\delta^{2}} \\
  &\leqslant F(x+2\delta) + \eps + \frac{2\sigma^{2}}{\delta^{2}} \\
  &< F(x) + 2\eps + \frac{2\sigma^{2}}{\delta^{2}}.
 \end{align*}

 Напишем похожую штуку в другую сторону:
 \begin{align*}
  \left\{ \xi +\eta \leqslant x - \delta \right\} \supset \left\{ \xi \leqslant x - 2\delta \right\} \setminus \left\{ \left| \eta \right| \geqslant \delta \right\}.
 \end{align*} Для вероятностей:
 \begin{align*}
  G(x-\delta) \geqslant F(x - 2\delta) - \frac{\sigma^{2}}{\delta^{2}}.
 \end{align*}  Аналогично получаем оценку снизу 
 \begin{align*}
  F_n(x) \geqslant G_n(x - \delta) - \frac{\sigma^{2}}{\delta^{2}} > G(x-\delta) - \eps - \frac{\sigma^{2}}{\delta^{2}} \geqslant F(x - 2\delta) - \eps - \frac{2\sigma^{2}}{\delta^{2}}  > F(x) - 2\eps - \frac{2\sigma^{2}}{\delta^{2}}.
 \end{align*}

 Тогда
 \begin{align*}
  \left| F_n(x) - F(x) \right| < 2\eps + 2 \frac{\sigma^{2}}{\delta^{2}}
 \end{align*} Выбираем $ \eps > 0$, по $ \eps $ выбираем $ \delta $,  а затем по $ \delta $ выбираем $ \sigma $, и получаем нужное $ N $. Так и построим $ N $ по $ \eps $.

\end{proof}

\begin{thm}
 Пусть функции $ F_n, F \colon\, \R \to [0,1] $(они возрастают), $ F \in C(\R) $ и $ F_n \to F $ поточечно. Тогда $ F_n \rightrightarrows F $ равномерно.
\end{thm}
\begin{proof}[\normalfont\textsc{Доказательство}]
 Найдём $ t_j $ такое, что $ F(t_j) = \frac{j}{m} $, где $ j = 1, 2, \ldots, m - 1 $. Мы знаем, что для всех $ j $ верно $ F_n(t_j) \to F(t_j) $, поэтому при больших $ n $ верно
 \begin{align*}
  \left| F_n(t_j) - F(t_j) \right| < \eps
 \end{align*}

 \begin{figure}[ht]
  \centering
  \incfig[0.5]{distribution_function_points_to_evenlly}
  \caption{Выбор $t_1, \dots, t_{m-1}$. }
  \label{fig:distribution_function_points_to_evenlly}
 \end{figure}

 Тогда возьмём $ t_{j-1} < t \leqslant t_j $. Тогда при больших $ n $:
 \begin{align*}
  F_n(t) \leqslant F_n(t_j) < F(t_j) + \eps = \frac{j}{m} + \eps = F(t_{j-1}) + \frac{1}{m} + \eps \leqslant F(t) + \eps + \frac{1}{m}.
 \end{align*} В другую сторону аналогично:
 \begin{align*}
  F_n(t) \geqslant F_n(t_{j-1}) > F(t_{j-1}) - \eps = \frac{j-1}{m} - \eps = F(t_j) - \frac{1}{m} - \eps \geqslant F(t) - \eps - \frac{1}{m}.
 \end{align*} Значит берём $ \frac{1}{m} < \eps $ и успех.
\end{proof}

\newpage
\section{Центральная предельная теорема.}

\begin{thm}[ЦПТ в форме Поля Лев\'{и}]
 \label{theorem:central_limit_theorem_Levi}
 Пусть случайные величины $ \xi_1, \xi_2, \ldots $  одинаково распределены и независимы. Пусть $ a = \E \xi_i $ --- их матожидание, и  $ \sigma^{2} = \D \xi_i > 0 $  --- их дисперсия,
 \begin{align*}
  S_n = \xi_1 + \xi_2 + \ldots + \xi_n
 \end{align*} --- их частичная сумма. Тогда
 \begin{align*}
  P\left(\frac{S_n - \E S_n}{\sqrt{\D S_n}} \leqslant x\right) = P \left( \frac{S_n - na}{\sigma \sqrt n} \leqslant x \right) \rightrightarrows \Phi(x),
 \end{align*} где
 \begin{align*}
  \Phi(x) = \frac{1}{\sqrt{2\pi}} \int\limits_{-\infty}^{x} e^{-t^{2} / 2}\,dt.
 \end{align*}
\end{thm}
\begin{remrk*}
 Поль Лев\'{и}(1886 - 1971) --- не тот, что и Л\'{е}ви(Беппо Леви, 1875 - 1961) из теоремы в теории меры о перестановке знака предела и интеграла Лебега.
\end{remrk*}
\begin{proof}[\normalfont\textsc{Доказательство}]
 Разложим
 \begin{align*}
  \varphi(t) = \varphi_{\xi_1 - a}(t) = 1 - \frac{\sigma^{2}t^{2}}{2} + o(t^{2})
 \end{align*} Положим
 \begin{align*}
  \varphi_n(t) &:= \varphi_{\frac{S_n - na}{\sigma \sqrt n}}(t) = \varphi_{S_n - na} \left( \frac{t}{\sigma \sqrt n} \right) = \prod_{k=1}^{n} \varphi_{\xi_k - a} \left( \frac{t}{\sigma \sqrt n} \right) = \\
  &= \left(\varphi \left( \frac{t}{\sigma \sqrt n} \right) \right)^{n} = \left( 1- \frac{\sigma^{2} \frac{t^{2}}{\sigma^{2} n}}{2} + o \left( \frac{t^{2}}{\sigma^{2} n} \right) \right)^{n} = \\
  &= \left( 1 - \frac{t^{2}}{2n} + o \left( \frac{t^{2}}{n} \right) \right)^{n} \to e^{-t^{2} / 2}.
 \end{align*} Осталось проверить поточечную сходимость характеристических функций:

 \begin{align*}
  \left(1 - \frac{t^2}{2n} + o\left(\frac{t^2}{n}\right)\right)^{n} \to e^{-t^2/2} \Longleftrightarrow \\
  n \log \left(1 - \frac{t^2}{2n} + o\left(\frac{t^2}{n}\right)\right) \to - \frac{t^2}{2} \Longleftrightarrow \\
  n \left(-\frac{t^2}{2n} + o\left(\frac{t^2}{n}\right)\right) \to -\frac{t^2}{2} \Longleftrightarrow \\
  -\frac{t^2}{2} \to -\frac{t^2}{2}
 \end{align*}. Получили что и требовалось.

\end{proof}

\begin{crly}[Интегральная теорема Муавра-Лапласа]
 Пусть $ S_n  $ --- количество успехов в схеме Бернулли в схеме с  $ n $  испытаниями и вероятностью успеха $ p $ . Тогда
 \begin{align*}
  P \left( \frac{S_n - np}{\sqrt{npq}} \leqslant x \right) \rightrightarrows \Phi(x).
 \end{align*}
\end{crly}
\begin{proof}[\normalfont\textsc{Доказательство}]
 Пусть
 \begin{align*}
  \xi_k = \begin{cases}
   1, \text{ с вероятностью } p, \\
   0, \text{ с вероятностью } q = 1 - p.
  \end{cases}
 \end{align*} Тогда $ \E \xi_k = p $, $ \D \xi_k = pq $, $ \xi_k $ независимы. Применим ЦПТ.
\end{proof}

\begin{thm}[Пуассона]
 Пусть $ P(\xi_{nk} = 1) = p_{nk} $, $ P(\xi_{nk} = 0) = 1-p_{nk} $. Много бернуллевских случайных величин. Пачка $ \xi_{n 1}, \xi_{n 2}, \ldots, \xi_{n n} $ независимы. Пусть
 \begin{align*}
  \max_{k=1}^{n} p_{nk} \to 0
 \end{align*} при $ n \to \infty $. Пусть также
 \begin{align*}
  p_{n 1}  + p_{n 2} + \ldots + p_{n n} \to \lambda > 0.
 \end{align*} Пусть $ S_n = \xi_{n 1} + \xi_{n 2} + \ldots + \xi_{n n} $. Тогда
 \begin{align*}
  P(S_n = k) \to \frac{e^{-\lambda}\lambda^{k}}{k!}.
 \end{align*}
\end{thm}
В стандартной теореме Пуассона $ p_{n 1} = p_{n 2 } = \ldots = p_{n n} $.
\begin{proof}[\normalfont\textsc{Доказательство}]
 Посмотрим на характеристические функции $ \xi_{n k} $:
 \begin{align*}
  \varphi_{\xi_{n k}}(t) = \E e^{it \xi_{n k}} = (1 - p_{n k}) + e^{it} p_{n k} = 1 + (e^{it} - 1)p_{n k},
 \end{align*} ведь случайная величина принимает лишь два значения.
 \begin{align*}
  \varphi_{S_n}(t) = \prod_{k=1}^{n} \varphi_{\xi_{n k}}(t) = \prod_{k=1}^{n} (1 + (e^{it} - 1)p_{n k}).
 \end{align*} Запишем характеристическую функцию Пуассона:
 \begin{align*}
  \exp((e^{it} - 1)\lambda).
 \end{align*} Проверим: прологарифмируем:
 \begin{align*}
  \log \varphi_{S_n}(t) = \sum_{k=1}^{n}\log (1 + (e^{it} - 1)p_{nk}) \overset{?}{\to} (e^{it} - 1) \lambda
 \end{align*} Но
 \begin{align*}
  \log(1 + (e^{it} - 1)p_{n k}) = (e^{it} - 1)p_{n k} + O(p_{n k}^{2})
 \end{align*} Тогда сумма
 \begin{align*}
  = \sum_{k=1}^{n} \left( (e^{it}-1)p_{nk} +O(p_{nk}^{2}) \right) = (e^{it} - 1) \underbrace{\sum_{k=1}^{n}p_{nk}}_{\to \lambda} + \sum_{k = 1}^{n} O(p_{nk} ^{ 2}).
 \end{align*} Оценим остаток:
 \begin{align*}
  \sum_{k=1}^{n}p_{nk}^{2} \leqslant \sum_{k=1}^{n} p_{nk} \cdot \max p_{n k} \to \lambda \cdot \underbrace{\max p_{nk}}_{\to 0} \to 0.
 \end{align*} Мы поняли, что $ \varphi_{S_n}(t) \to \exp((e^{it}-1)\lambda) $. Тогда есть сходимость по распределению: $ S_n $ сходится по распределению к Пуассону. Так как всё дискретно, то условие <<точки непрерывности>> --- не проблема.
\end{proof}
\begin{remrk*}
 Вычислим характеристическую функцию распределения Пуассона:
 \begin{align*}
  \varphi(t) &= \E e^{it\xi} = \sum_{n=0}^{\infty} e^{itn} P(\xi = n) = \sum_{n=0}^{\infty} (e^{it})^{n} \cdot \frac{e^{-\lambda} \lambda^{n}}{n!} = e^{-\lambda} \sum_{n = 0}^{\infty} \frac{(e^{it}\lambda)^{n}}{n!} = \\
  &= e^{-\lambda} \cdot e^{\lambda e^{it}} = \exp((e^{it} - 1)\lambda).
 \end{align*}
\end{remrk*}

\begin{thm}[%
 ЦПТ в форме Линденберга]
 Пусть случайные величины $ \xi_1, \xi_2, \ldots $  независимы, $ a_k = \E \xi_k $,  $ \sigma_k^{2} = \D \xi_k > 0 $ . Пусть $ S_n = \xi_1 + \ldots + \xi_n $, и пусть
 \begin{align*}
  DS_n = D_n^{2} = \sum_{k=1}^{n} \sigma_k^{2}.
 \end{align*} Рассмотрим функцию $ f(x) := x^{2} \Ind_{\left\{ \left| x \right| \geqslant \eps D_n \right\}} $. Обозначим
 \begin{align*}
  \mathrm{Lind}(\eps, n) := \frac{1}{D_n^{2}} \sum_{k=1}^{n} \E f(\xi_k - a_k).
 \end{align*} Пусть для любого $ \eps > 0 $ верно
 \begin{align*}
  \lim_{n \to \infty} \mathrm{Lind}(\eps, n) = 0
 \end{align*} --- \textit{условие Линденберга}. Тогда верно заключение ЦПТ:
 \begin{align*}
  P \left( \frac{S_n - \E S_n}{\sqrt{\D S_n}} \leqslant x \right) \rightrightarrows \Phi(x).
 \end{align*}
\end{thm}

Доказывать не будем (много считать).

\begin{exercs*}
 Проверить условие Линденберга для независимых одинаково распределённых случайных величин с конечной дисперсией.
\end{exercs*}

\begin{thm}[%
 ЦПТ в форме Ляпунова]
 Пусть $ \xi_1, \xi_2, \ldots $ независимы. $ a_k = \E \xi_k $, $ \sigma_k^{2} = \D \xi_k > 0 $, пусть
 \begin{align*}
  L(\delta, n) = \frac{1}{D_n^{2+\delta}} \sum_{k=1}^{n} \E \left| \xi_k - a_k \right|^{2 + \delta} \to 0
 \end{align*}, где $ D_n $ и $ S_n $ --- то же самое. для некоторого $ \delta  > 0 $. Тогда верно заключение ЦПТ:
 \begin{align*}
  P \left( \frac{S_n - \E S_n}{\sqrt{\D S_n}} \leqslant x \right) \rightrightarrows \Phi(x).
 \end{align*}
\end{thm}
\begin{proof}[\normalfont\textsc{Вывод из ЦПТ в форме Линденберга}]
 \begin{align*}
  \mathrm{Lind}(\eps, n) &= \frac{1}{D_n^{2}} \sum_{k=1}^{n} \E \left( (x_k-a_k)^{2}\Ind_{\left\{ \left| x \right| \geqslant \eps D_n\right\}}(\xi_k - a_k) \right).
 \end{align*} Заметим, что $ \Ind_{\left\{ \left| x \right| \geqslant \eps D_n \right\}} \leqslant \left(\frac{\left| x \right|}{\eps D_n} \right)^{\delta} $. Подставим вместо индикатора эту дробь:
 \begin{align*}
  \leqslant \frac{1}{D_n^{2}} \sum_{k=1}^{n} \E \left( (\xi_k-a_k)^{2} \cdot \frac{\left| \xi_k-a_k \right|^{\delta}}{\eps^{\delta} D_n^{\delta}} \right) = \frac{L(\delta, n)}{\eps^{\delta}}.
 \end{align*} 
\end{proof}
\begin{thm}[Оценка для Ляпунова]
 \begin{align*}
  \left| P \left( \frac{S_n-\E S_n}{\sqrt{\D S_n}} \leqslant x\right) - \Phi(x) \right| \leqslant C_\delta L(\delta, n),
 \end{align*} где $ C_\delta $ --- константа, зависящая от $ \delta $, для любого $ \delta \in (0, 1] $.
\end{thm}
\begin{crly}
 Если $ \xi_1, \xi_2, \ldots $  одинаково распределены и независимы, то
 \begin{align*}
  \left| P \left( \frac{S_n - na}{\sigma \sqrt n} \leqslant x \right) - \Phi(x) \right| \leqslant C_\delta,
 \end{align*} $ D_n^{2} = n\sigma^{2} $. Тогда
 \begin{align*}
  L(\delta, n) = \frac{1}{n^{1 + \frac{\delta}{2}}} \cdot \frac{1}{\sigma^{2 + \delta}} n \E \left| \xi_1 - a \right|^{2 + \delta}
 \end{align*}
 \begin{align*}
  \leqslant C_\delta \frac{\E \left| \xi_1 - a \right|^{2+\delta}}{n^{\delta / 2} \sigma^{2 + \delta}}.
 \end{align*}
\end{crly}

\begin{thm}[Берри-Эссеена]
 Пусть $ \xi_1, \xi_2, \ldots $ независимы и одинаково распределены.  Тогда
 \begin{align*}
  \left| P \left( \frac{S_n - na}{\sigma \sqrt n} \leqslant x \right) - \Phi(x) \right| \leqslant C \frac{\E \left| \xi_1 - a \right|^{3}}{\sqrt n \cdot \sigma^{3}}.
 \end{align*}
\end{thm}

Что известно про константы?

\begin{itemize}
 \item Эссеен (1956): $ C \geqslant \frac{3 + \sqrt{10}}{6 \sqrt{2\pi}} \approx 0.4097 $.
 \item Шевцова (2014): $ C \leqslant 0.469 $.
\end{itemize}

Для общего случая известно $ C_1 \leqslant 0.5583 $.

Для схемы Бернулли $ C \leqslant 0.4099 $ (не хватает $ 0.002 $ до нижней оценки!) (2018)

Для схемы Бернулли  с $ p = \frac{1}{2} $: $ C = \frac{1}{\sqrt{2\pi}} $.

Чисто для справки приведём ещё два результата.

\begin{thm}[Хартмана-Винтнера, закон повторного логарифма]
 Пусть случайные величины $ \xi_1, \ldots $ независимы и одинаково распределены, $ \E \xi_i = 0 $, $ \sigma^{2} = \D \xi_i > 0 $. Тогда
 \begin{align*}
  \varlimsup_{n \to \infty} \frac{S_n}{\sqrt{2n \log\log n}} = \sigma,
 \end{align*} и
 \begin{align*}
  \varliminf_{n \to \infty} \frac{S_n}{\sqrt{2n \log\log n}} = -\sigma.
 \end{align*}
\end{thm}

\begin{thm}[Штрассена]
 Любая точка из $ [-\sigma, \sigma] $ является предельной точкой последовательности
 \begin{align*}
  \frac{S_n}{\sqrt{2 n \log \log n}}.
 \end{align*}
\end{thm}

\end{document}
