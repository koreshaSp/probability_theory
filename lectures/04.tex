% 2023.03.07: lecture 04

Отступление в теорию меры: интеграл по мере, имеющей плотность.

\begin{thm}[%
]
 Пусть $ w \geqslant 0 $ --- измеримая, неотрицательная функция такая, что
 \begin{align*}
  \nu(A) = \int\limits_{A} w\,d\mu 
 \end{align*} --- мера, где $ \mu $ --- мера. Тогда
 \begin{align*}
  \int\limits_{X} f\,d\nu = \int\limits_{X} f \omega \, d\mu,
 \end{align*} если $ f \geqslant 0 $ или если $ f\omega $ суммируема относительно $ \mu $.
\end{thm}
\begin{proof}[\normalfont\textsc{Доказательство}]\
 \begin{itemize}
  \item Пусть $ f = \chi_A $ --- индикаторная функция. Тогда
   \begin{align*}
    \int\limits_{X} f\, d\nu = \int\limits_{X} \chi_A \, d\nu = \nu(A) = \int\limits_{A} w\,d\mu = \int\limits_{X} \chi_A w\,d\mu.    
   \end{align*} 
  \item По линейности верно для простых неотрицательных функций $ f $.
  \item Пусть $ f $ неотрицательна и измерима. По теореме об аппроксимации можно взять последовательных простых неотрицательных функций $ f_n \geqslant 0 $, которые поточечно возрастают к $ f $. Тогда
   \begin{align*}
    \int\limits_{X} f_n \,d\nu = \int\limits_{X} f_n w \,d\mu.  
   \end{align*} По теореме Леви левая часть стремится к $ \int_{X} f\,d\nu  $, а правая --- к $ \int_{X} fw\,d\mu $, ведь функции $ f_nw $ тоже неотрциательные и возрастают к $ fw $.
 \end{itemize}
\end{proof}

Рассмотрим несколько примеров вероятностных распределений.

\begin{exmpl}[биномиальное распределение]
 $ \xi \colon\, \Omega \to \left\{ 0, 1, \ldots, n \right\} $. Обозначение: $ \xi \sim \mathrm{Binom}(n,p) $, где $ 0 \leqslant p \leqslant 1 $.
 \begin{align*}
  P[\xi = k] = \binom n k p^{k}(1-p)^{k}.
 \end{align*} 
\end{exmpl}
\begin{exmpl}[распределение Пуассона]
$ \xi \colon\, \Omega \to \left\{ 0, 1, 2, \ldots \right\} $. Обозначение: $ \xi \sim \mathrm{Poisson}(\lambda) $, где $ \lambda > 0 $.
  \begin{align*}
  P[\xi = k] = \frac{\lambda^{k}}{k!}e^{-\lambda}.
 \end{align*} 
\end{exmpl}
\begin{exmpl}[%
геометрическое распределение]
$ \xi \colon \Omega \to \left\{ 1,2,3, \ldots \right\} $, $ \xi \sim \mathrm{Geom}(p) $, где $ 0 < p \leqslant 1 $. Здесь
\begin{align*}
 P[\xi = k] = p(1 - p)^{k-1}.
\end{align*} Это вероятность того, что первый успех в схеме Бернулли случился на шаге $ k $.

 Иногда удобнее вычитать на единицу (но формулы чуть поменяются).
\end{exmpl}

\begin{exmpl}[%
дискретное равномерное распределение]
 $ \xi \colon\, \Omega \to \left\{ 1,2,\ldots,n \right\} $,
 \begin{align*}
  P[\xi = k] = \frac{1}{n}.
 \end{align*} Общепринятого обозначения нет.
\end{exmpl}
\begin{exmpl}[непрерывное равномерное распределение]
 $ \xi\colon\,\Omega\to[a,b] $. Обозначение: $ \xi \sim U[a,b] $. Здесь
 \begin{align*}
  p_{\xi}(t) = \frac{1}{b-a} \chi_{[a,b]}(t).
 \end{align*} 
\end{exmpl}
\begin{exmpl}[%
нормальное распределение]
$ \xi \colon\,\Omega \to \R $, $ \xi \sim N(a,\sigma^{2}) $, где $ a \in \R $, $ \sigma > 0 $. Здесь
\begin{align*}
	p_{\xi}(t) = \frac{1}{\sigma\sqrt{2\pi}}e^{-\frac{1}{2}\left(\frac{t-a} {\sigma}\right)^2}
.\end{align*} $ N(0,1) $ называется \textit{cтандартным} нормальным распределением.
\end{exmpl}
\begin{exmpl}[%
экспоненциальное распределение]
$ \xi \colon\,\Omega\to[0,+\infty) $, $ \xi \sim \mathrm{Exp}(\lambda) $, $ \lambda > 0 $.
\begin{align*}
 p_{\xi}(t) = \begin{cases}
   \lambda e^{-\lambda t}, \text{ при } t \geqslant 0;  \\
  0, \text{ иначе. }
 \end{cases} 
\end{align*} 
\end{exmpl}

\section{Совместное распределение}

\begin{df}
 \textit{Совместное} (оно же \textit{многомерное}) распределение. Есть много случайных величин $ \xi_1, \xi_2, \ldots, \xi_n \colon\, \Omega\to\R $.
 \begin{align*}
  \vec \xi = (\xi_{1}, \ldots, \xi_n) \colon\,\Omega \to \R^{n}.
 \end{align*}
 \begin{align*}
  P_{\vec\xi}(A) := P[\vec\xi \in A]
 \end{align*} для борелевских множеств $ A $. Распределение векторозначной величины. 
\end{df}
\begin{remrk}
 $ P_{\vec\xi} $ однозначно определяет распределения координат $ P_{\xi_1}, \ldots, P_{\xi_n} $, но не наоборот. Действительно,
 \begin{align*}
  P_{\xi_j}(A) = P_{\vec\xi}(A \times \R^{n-1})
 \end{align*} 
\end{remrk}
\begin{exmpl}
 $ \xi $ и $ \eta \colon\,\Omega \to \left\{ 0,1 \right\}$  с вероятностью $ \frac{1}{2} $. Если они независимы, то $ (\xi,\eta) \to \left\{ \left\{ 0,0 \right\}, \left\{ 0,1 \right\}, \left\{ 1,0 \right\}, \left\{ 1,1 \right\} \right\} $ с равной вероятностью.

 Если же $ \xi = \eta $, то $ (\xi,\eta) \colon\,\Omega \to \left\{ \left\{ 0,0 \right\}, \left\{ 1,1 \right\} \right\} $ с вероятностью $ \frac{1}{2} $. При это распределение координат такое же!
\end{exmpl}

\begin{df}
 Случайные величины $ \xi_1, \xi_2, \ldots, \xi_n \colon\, \Omega \to \R $ называются \textit{независимыми}, если для любых борелевских множеств $ A_1, \ldots, A_n \subset \R $ верно
 \begin{align*}
  P[\xi_1 \in A_1] \cdot \ldots \cdot P[\xi_n \in A_n] = P[\xi_1 \in A_1, \ldots, \xi_n \in A_n].
 \end{align*} Иными словами, события $ \xi_1 \in A_1, \ldots, \xi_n \in A_n $ независимы (можно пропускать множители за счёт выбора $ A_j  = \R $).
\end{df}

\begin{thm}
 $ \xi_1, \ldots, \xi_n $ --- независимые случайные величины тогда и только тогда, когда
 \begin{align*}
  P_{\vec\xi} = P_{\xi_1} \times P_{\xi_2} \times \ldots \times P_{\xi_n}.
 \end{align*} 
\end{thm}
\begin{proof}[\normalfont\textsc{Доказательство}]
 Нужно проверить равенство мер. Проверим совпадение на множествах вида $ A_1 \times \ldots \times A_n $:
 \begin{align*}
  P_{\vec\xi}(A_1 \times \ldots \times A_n) &= P[\xi_1 \in A_1, \ldots, \xi_n \in A_n] =  \\
  &= P[\xi_1 \in A_1] \cdot \ldots \cdot P[\xi_n \in A_n] = \\
  &= P_{\xi_1}(A_1) \cdot \ldots \cdot P_{\xi_n}(A_n).
 \end{align*} Этого достаточно по единственности стандартного продолжения.
\end{proof}

\begin{df}
 Совместная функция распределения $ F_{\vec\xi}(\vec x) $, где $ \vec x \in \R^{n} $, $ F_{\vec\xi} \colon \R^{n} \to [0,1] $:
 \begin{align*}
  F_{\vec\xi}(\vec x) := P[\xi_1 \leqslant x_1, \ldots, \xi_n \leqslant x_n].
 \end{align*} 
\end{df}

\begin{prop}[cвойства совместной функции распределения]\
 \begin{enumerate}
  \item $ F_{\vec\xi} $ возрастает по каждой координате.
  \item \begin{align*}
    \lim_{x_1, \ldots, x_n \to +\infty}  F_{\vec\xi}(\vec x) = 1.
  \end{align*} 
 \item \begin{align*}
   \lim_{x_i \to -\infty} F_{\vec\xi}(\vec x) = 0.
 \end{align*} 
\item \begin{align*}
  \lim_{x_i \to +\infty} F_{\vec\xi}(\vec x) = F_{\xi_1, \ldots, \xi_{i-1}, \xi_{i+1}, \ldots, \xi_n}(x_1, \ldots, x_{i-1},x_{i+1},\ldots,x_n).
\end{align*} 
 \end{enumerate}
\end{prop}

\begin{df}
 Совместная(многомерная) плотность распределения $ p_{\vec\xi}(\vec t) \geqslant 0 $  измеримая, такая что:
 \begin{align*}
  F_{\vec\xi}(\vec x) = \int\limits_{-\infty}^{x_1}  \ldots \int\limits_{-\infty}^{x_n} p_{\vec\xi}(\vec t) dt_n \ldots dt_1.
 \end{align*} 
 $\vec \xi$ имеет абсолютно непрерывное распределение если у него есть плотность.
\end{df}
\begin{crly}
	$ \xi_1, \ldots, \xi_n $ независ. $ \iff $ $ F_{\vec\xi}(\vec x) = F_{\xi_1}(x_1) \ldots F_{\xi_n}(x_n), \, \forall \vec x \in \R^n $.
\end{crly}
\begin{proof}[\normalfont\textsc{Доказательство}]
 Будем проверять $ P_{\vec\xi} = P_{\xi_1} \times \ldots \times P_{\xi_n} \iff F_{\vec\xi}(\vec x) = F_{\xi_1}(x_1) \ldots F_{\xi_n}(x_n) $.

$\implies F_{\vec \xi}(\vec x) = P_{\vec \xi}((-\infty, \vec x]) = P_{\xi_1}((-\infty, x_1]) \dots P_{\xi_n}((-\infty, x_n]) = F_{\xi_1}(x_1) \dots F_{\xi_n}(x_n)$ 

 $ \impliedby $ : картинка $ n=2 $. Надо показать на ячейках. 
 \begin{align*}
  P_{\xi} (a_1,b_1] = F_{\xi}(b_1) - F_{\xi}(a_1) \\
  P_{\eta} (a_2,b_2] = F_{\eta}(b_2) - F_{\eta}(a_2).
 \end{align*} Тогда
 \begin{align*}
	 P_{\xi, \eta} \left( (a_1, b_1] \times (a_2, b_2] \right) 
	 = F_{\xi, \eta} (b_1, b_2) + F_{\xi, \eta} (a_1, a_2) - F_{\xi, \eta} (a_1, b_2) - F_{\xi, \eta}(a_2, b_1) = \\ 
	 = F_{\xi} (b_1) F_{\eta} (b_2) + F_{\xi}(a_1) F_{\eta} (a_2) - F_{\xi}(a_1) F_{\eta} (b_2) - F_{\xi} (a_2) F_{\eta} (b_1) = \\
	 = \left(F_{\xi} (b_1) - F_{\xi} (a_1) \right) \left( F_{\eta} (b_2) - F_{\eta} (a_2) \right) = \\
	 = P_\xi (a_1, b_1] P_\eta (a_2, b_2]
 \end{align*}

 При $n > 2$ формулы будут сложнее, но идейно то же самое.

\end{proof}
\begin{crly}
 Если $ \xi_1, \ldots, \xi_n $ абс. непр. то они независимы $ \iff p_{\vec\xi}(\vec t) = p_{\xi_1}(t_1) \ldots p_{\xi_n}(t_n) $.

 В частности, если $ \xi_1, \ldots, \xi_n $ независимы, то $ \vec\xi $ абс. непр.
\end{crly}
\begin{proof}[\normalfont\textsc{Доказательство}]
 Проверим, что
 \begin{align*}
  F_{\vec\xi}(\vec x) = F_{\xi_1} \ldots F_{\xi_n} \iff p_{\vec\xi}(\vec t) = p_{\xi_1}(t_1) \ldots p_{\xi_n}(t_n)
 \end{align*}
 $ \implies $ Покажем, что $ p_{\xi_1}(t_1) \ldots p_{\xi_n}(t_n) $ подходит под определение плотности.
 \begin{align*}
  F_{\vec\xi}(\vec x) = F_{\xi_1}(x_1) \ldots F_{\xi_n}(x_n) = \int\limits_{-\infty}^{x_1} p_{\xi_1}(t_1)\,dt_1 \ldots \int\limits_{-\infty}^{x_n} p_{\xi_n}(t_n)\,dt_n = \\
  = \int\limits_{-\infty}^{x_1} \ldots \int\limits_{-\infty}^{x_n} p_{\xi_1}(t_1) \ldots p_{\xi_n}(t_n)\,dt_n\ldots dt_1
 \end{align*} Это и есть плотность. В обратную сторону то же самое.
\end{proof}

Ещё одно отступление: свёртка мер.

Есть две меры $ \mu $ и $ \nu $ --- конечные на борелевских подмножествах $ R $ ($ \mathcal B_1 $ ). Для $ A \in \mathcal B_1 $ определим
 \begin{align*}
 \mu \ast \nu(A) := \int\limits_{-\infty}^{+\infty} \mu(A-x)\,d\nu(x)
\end{align*}

где $A - x := \{ a - x \ \vert \ a \in A \}$

Свойства:
\begin{enumerate}
 \item $ \mu \ast \nu(A) = \int_{\R^{2}} \chi_{A}(x + y) \, d\mu(x)\, d\nu(y)   $
\begin{proof}[\normalfont\textsc{Доказательство}]
 Мы знаем, что
 \begin{align*}
 \mu \ast \nu(A) = \int\limits_{\R} \mu(A-y) \,d\nu(y) = \int\limits_{\R} \int\limits_{\R} \chi_{A-y}(x)\,d\mu(x)\,d\nu(y) = \\
 =\int\limits_{\R} \int\limits_{\R}  \chi_A(x+y)\,d\mu(x)\,d\nu(y)
 =\int\limits_{\R^{2}}  \chi_A(x+y)\,d\mu(x)\,d\nu(y).
 \end{align*} 
\end{proof}

\begin{remrk*}
	Почти все последующие свойства являются тривиальными следствиями 1-го свойства.
\end{remrk*}

 \item Есть коммутативность $ \mu \times \nu = \nu \times \mu $.
 \item 
\begin{align*}
 \mu_1 \ast \ldots \ast \mu_n (A) = \int\limits_{\R^{n}}\chi_A(x_1 + x_2 + \ldots + x_n)\,d\mu_1(x_1) \ldots d\mu_n(x_n).   
\end{align*}
\item Ассоциативность: $(\mu_1 \ast \mu_2) \ast \mu_3 = \mu_1 \ast (\mu_2 \ast \mu_3)$ .
\item $ (c\mu) \ast \nu = c \cdot \mu \ast \nu $.
\item Линейность: $ (\mu_1 + \mu_2) \ast \nu = \mu_1 \ast \nu + \mu \ast \nu $.
\item $ \mu \ast \delta_0 = \mu $, где $\delta_0$~--- мера с единичной нагрузкой в нуле.

\end{enumerate}
\begin{df}
	$ \delta_x(A) = \begin{cases} 1 & x \in A \\ 0 & x \notin A \end{cases} $~--- мера с единичной нагрузкой в точке $ x $.
\end{df}
\begin{proof}[\normalfont\textsc{Доказательство}]
		
	\begin{align*}
	\mu \ast \delta_0(A) = \delta_0 \ast \mu (A) = \int\limits_{\R} \delta_0(A - x) d \mu (x) = \\ 
	= \left[ \delta_0(A - x) = 1 \Leftrightarrow 0 \in A - x \Leftrightarrow x \in A \right] = \\ 
	= \int\limits_{\R} \chi_A(x) d \mu (x) = \mu (A)
	\end{align*}
	
\end{proof}

А что будет, если эти меры будут с плотностями?

\begin{thm}
 Если меры $ \mu $ и $ \nu $ имеют плотности $ p_{\mu} $ и $ p_{\nu} $, тогда $ \mu \ast \nu $ имеет плотность $ p(t) := \int_{\R} p_{\mu}(t-u)p_{\nu}(u)\,du  $ --- \textit{cвёртка функций}.
\end{thm}
\begin{proof}[\normalfont\textsc{Доказательство}]
 \begin{align*}
  \int\limits_{A} p(t)\,dt = \int\limits_{A} \int\limits_{\R} p_{\mu}(t-u)p_{\nu}(u)\,du\,dt = \int\limits_{\R} \int\limits_{\R} \chi_A(t)p_{\mu} (t-u)p_{\nu}(u)\,du\,dt = \\
  = [v = t - u] = \int\limits_{\R} \int\limits_{\R} \chi_{A}(u+v) p_{\mu}(v) p_{\nu}(u) \, du\,dv = \\
  = \int\limits_{\R} \int\limits_{\R} \chi_A(u+v)\,d\mu(v)\,p_{\nu}(u)\,du = \int\limits_{\R^{2}} \chi_A(u+v)\,d\mu(v)\,d\nu(u) = \mu \ast \nu(A).   
 \end{align*} 
\end{proof}

\begin{thm}
 Если $ \xi $ и $ \eta $ независимы, то $ P_{\xi + \eta} = P_\xi \ast P_\eta $.
\end{thm}
\begin{proof}[\normalfont\textsc{Доказательство}]
	$B := \{(x, y) \mid x + y \in A \}$ 

 \begin{align*}
  P_{\xi+\eta}(A) &= P[\xi + \eta \in A] = P[(\xi,\eta)\in B] = P_{(\xi,\eta)}(B) = \\
  &= \int\limits_{\R^{2}} \chi_B(x,y) \,dP_{(\xi,\eta)}(x,y) = \\
  &= \int\limits_{\R^{2}} \chi_B(x,y)\,dP_\xi(x)dP_\eta(y) = \\ 
  &= \int\limits_{\R^{2}} \chi_A(x+y) \,dP_\xi(x)dP_\eta(y) = P_{\xi} \ast P_{\eta}(A).
 \end{align*} 
\end{proof}

\begin{exmpl}
 Свёртка с дискретным распределением.
  \begin{align*}
   \nu = \sum_{k=1}^{\infty} p_k \delta_{x_k}, \qquad p_k > 0
 \end{align*} 
 \begin{align*}
  \mu \ast \nu(A) = \int\limits_{\R} \mu(A - x) d\nu(x) = \sum_{k=1}^{\infty} p_k \int\limits_{\R} \mu(A-x) d \delta_{x_k}(x) = \\
  \sum_{k=1}^{\infty}p_k\mu(A-x_k).
 \end{align*} 
\end{exmpl}
\begin{exmpl}
 Свернём два Пуассона: $ \xi_i \sim \mathrm{Poisson}(\lambda_i) $ назавис. $ \xi_1 + \xi_2 \sim ? $

 \begin{align*}
  P_{\xi_1 + \xi_2}(\left\{ n \right\}) = P_{\xi_1} \ast P_{\xi_2}(\left\{ n \right\})
 \end{align*} Ответ 
  
\end{exmpl}

Упражнения

\begin{itemize}
 \item $ \xi_1 $ и $ \xi_2 $ независимы $ \sim \mathrm{Exp}(1)$. Найти $ P_{\xi_1 + \xi_2} $.
 \item $ \xi_i \sim N(a_i,\sigma_i^{2}) $ независимы. Доказать, что $ \xi_1 + \xi_2 \sim N(a_1 + a_2, \sigma_1^{2} + \sigma_2^{2}) $.
\end{itemize}

